\chapter{System Memory Map}
\label{cha:memory-map}
\section{Introduction}

The MEGA65 computer has a large 28-bit address space, which allows it
to address up to 256MB of memory and memory-mapped devices.
This memory map has several different views, depending on which mode
the computer is operating in. Broadly, there are five main modes:
(1) Hypervisor mode; (2) C64 compatibility mode; (3) C65 compatibility mode; (4) UltiMAX
compatibility mode; and (5) MEGA65-mode, or one of the other modes,
where the programmer has made use of MEGA65 enhanced features.

It is important to understand that, unlike the C128, the C65 and
MEGA65 allow access to all enhanced features from C64-mode, if the
programmer wishes to do so.  This means that while we frequently talk
about ``C64-mode,'' ``C65-mode'' and ``MEGA65-mode,'' these are simply
terms of convenience for the MEGA65 with its memory map (and sometimes
other features) configured to provide an environment that matches
the appropriate mode.  The heart of this is the MEGA65's flexible
memory map.

In this appendix, we will begin by describing the MEGA65's native
memory map, that is, where all of the memory, I/O devices and other
features appear in the 28-bit address space. We will then explain how
C64 and C65 compatible memory maps are accessed from this 28-bit
address space.

\newpage

\section{MEGA65 Native Memory Map}

\subsection{The First Sixteen 64KB Banks}

The MEGA65 uses a similar memory map to that of the C65 for the first
MB of memory, i.e., 16 memory banks of 64KB each.
This is because the C65's 4510 CPU can access only 1MB
of address space.  These banks can be accessed from BASIC 65 using the
\stw{BANK}\index{BASIC 65 Commands!BANK},
\stw{DMA}\index{BASIC 65 Commands!DMA}, \stw{PEEK} and \stw{POKE}
commands.  The following table summarises the contents of the first
16 banks:

\setlength{\tabcolsep}{3pt}
\begin{longtable}{|L{1cm}|L{1.5cm}|L{2cm}|p{6cm}|}
\hline
{\bf{HEX}} & {\bf{DEC}} & {\bf{Address}} & {\bf{Contents}} \\
\hline
\endfirsthead
\multicolumn{4}{l@{}}{\ldots continued}\\
\hline
{\bf{HEX}} & {\bf{DEC}} & {\bf{Address}} & {\bf{Contents}} \\
\endhead
\multicolumn{4}{l@{}}{continued \ldots}\\
\endfoot
\hline
\endlastfoot
\hline
\small 0 & \small 0 & \$0xxxx & \multicolumn{1}{p{6cm}|}{First 64KB RAM. This is the RAM visible in C64-mode.}\\
\hline
\small 1 & \small 1 & \$1xxxx & \multicolumn{1}{p{6cm}|}{Second 64KB RAM. This is the 2nd 64KB of RAM present on a C65.}\\
\hline
\small 2 & \small 2 & \$2xxxx & \multicolumn{1}{p{6cm}|}{First half of C65 ROM (C64-mode and shared components) {\em or} RAM}\\
\hline
\small 3 & \small 3 & \$3xxxx & \multicolumn{1}{p{6cm}|}{Second half of C65 ROM (C65-mode components) {\em or} RAM}\\
\hline
\small 4 & \small 4 & \$4xxxx & \multicolumn{1}{p{6cm}|}{Additional RAM (384KB or larger chip-RAM models)}\\
\hline
\small 5 & \small 5 & \$5xxxx & \multicolumn{1}{p{6cm}|}{Additional RAM (384KB or larger chip-RAM models)}\\
\hline
\small 6 & \small 6 & \$6xxxx & \multicolumn{1}{p{6cm}|}{Additional RAM (512KB or larger chip-RAM models)}\\
\hline
\small 7 & \small 7 & \$7xxxx & \multicolumn{1}{p{6cm}|}{Additional RAM (512KB or larger chip-RAM models)}\\
\hline
\small 8 & \small 8 & \$8xxxx & \multicolumn{1}{p{6cm}|}{Additional RAM (1MB or larger chip-RAM models)}\\
\hline
\small 9 & \small 9 & \$9xxxx & \multicolumn{1}{p{6cm}|}{Additional RAM (1MB or larger chip-RAM models)}\\
\hline
\small A & \small 10 & \$Axxxx & \multicolumn{1}{p{6cm}|}{Additional RAM (1MB or larger chip-RAM models)}\\
\hline
\small B & \small 11 & \$Bxxxx & \multicolumn{1}{p{6cm}|}{Additional RAM (1MB or larger chip-RAM models)}\\
\hline
\small C & \small 12 & \$Cxxxx & \multicolumn{1}{p{6cm}|}{Additional RAM (1MB or larger chip-RAM models)}\\
\hline
\small D & \small 13 & \$Dxxxx & \multicolumn{1}{p{6cm}|}{Additional RAM (1MB or larger chip-RAM models)}\\
\hline
\small E & \small 14 & \$Exxxx & \multicolumn{1}{p{6cm}|}{Additional RAM (1MB or larger chip-RAM models)}\\
\hline
\small F & \small 15 & \$Fxxxx & \multicolumn{1}{p{6cm}|}{Additional RAM (1MB or larger chip-RAM models)}\\
\hline
\end{longtable}

The key features of this address space are the 128KB of RAM in the first two banks, which is also present on
the C65. If you intend to write programs which can also run on a C65, you should only use these two banks
of RAM.

On all models it is possible to use all or part of the 128KB of ``ROM'' space as RAM. To do this, you must first
request that the Hypervisor removes the read-only protection on this area, before you will be able to change
its contents.  If you are writing a program which will start from C64-mode, or otherwise switch to using the C64
part of the ROM, instead of the C65 part), then the second half of that space, i.e., BANK 3, can be safely used
for your programs. This gives a total of 192KB of RAM, which is available on all models of the MEGA65.

On models that have 384KB or more of chip RAM, BANK 4 and 5 are also available.  Similarly, models which provide
1MB or more of chip RAM will have BANK 6 through 15 also available, giving a total of 896KB (or 960KB, if only
the C64 part of the ROM is required) of RAM available for your programs.  Note that the MEGA65's built-in
freeze cartridge currently freezes only the first 384KB of RAM.

\subsection{Colour RAM}

The MEGA65's VIC-IV video controller supports much larger screens than the VIC-II or VIC-III. For this reason, it
has access to a separate colour RAM, similar to on the C64.  For compatibility with the C65, the first two kilo-bytes
of this are accessible at \$1F800 -- \$1FFFF.  The full 32KB or 64KB of colour RAM is located at \$FF80000.
This is most easily access through the use of advanced DMA operations, or the 32-bit base-page indirect addressing
mode of the processor.

At the time of writing, the \stw{BANK} and \stw{DMA} commands cannot be used to access the rest of the colour RAM, because the
colour RAM is not located in the first mega-byte of address space.  This may be corrected in a future revision of
the MEGA65, allowing access to the full colour RAM via BANK 15 or an equivalent DMA job.

\subsection{28-bit Address Space}

In addition to the C65-style 1MB address space, the MEGA65 extends this
to 256MB, by using 28-bit addresses.  The following shows the high-level
layout of this address space.

\setlength{\tabcolsep}{3pt}
\begin{longtable}{|L{1.5cm}|L{2cm}|L{1.1cm}|p{6cm}|}
\hline
{\bf{HEX}} & {\bf{DEC}} & {\bf{Size}} & {\bf{Contents}} \\
\hline
\endfirsthead
\multicolumn{3}{l@{}}{\ldots continued}\\
\hline
{\bf{HEX}} & {\bf{DEC}} & {\bf{Size}} & {\bf{Contents}} \\
\endhead
\multicolumn{3}{l@{}}{continued \ldots}\\
\endfoot
\hline
\endlastfoot
\hline
\small 0000000 & \small 0 & 1 & \multicolumn{1}{p{6cm}|}{CPU I/O Port Data
  Direction Register}\\
\hline
\small 0000001 & \small 1 & 1 & \multicolumn{1}{p{6cm}|}{CPU I/O Port Data}\\
\hline
\small 0000002 -- 005FFFF & \small 2 -- 384KB & 384KB &
\multicolumn{1}{p{6cm}|}{Fast chip RAM (40MHz)}\\
\hline
\small 0060000 -- 0FFFFFF & \small 384KB -- 16MB  & 15.6MB &
\multicolumn{1}{p{6cm}|}{Reserved for future chip RAM expansion}\\
\hline
\small 1000000 -- 3FFFFFF & \small 16MB -- 64MB & 48MB &
\multicolumn{1}{p{6cm}|}{Reserved}\\
\hline
\small 4000000 -- 7FFFFFF & \small 64MB -- 128MB & 64MB &
\multicolumn{1}{p{6cm}|}{Cartridge port and other devices on the slow bus
  (1 -- 10 MHz)}\\
\hline
\small 8000000 -- 87FFFFF & \small 128MB -- 135MB & 8MB &
\multicolumn{1}{p{6cm}|}{8MB ATTIC RAM (selected models only)}\\
\hline
\small 8800000 -- 8FFFFFF & \small 135MB -- 144MB & 8MB &
\multicolumn{1}{p{6cm}|}{8MB CELLAR RAM (selected models only)}\\
\hline
\small 9000000 -- EFFFFFF & \small 144MB -- 240MB & 96MB &
\multicolumn{1}{p{6cm}|}{Reserved for future expansion RAM}\\
\hline
\small F000000 -- FF7DFFF & \small 240MB -- 255.49MB & 15.49MB &
\multicolumn{1}{p{6cm}|}{Reserved for future I/O expansion}\\
\hline
\small FF7E000 -- FF7EFFF & \small 255.49MB -- 255.49MB & 4KB &
\multicolumn{1}{p{6cm}|}{VIC-IV Character ROM (write only)}\\
\hline
\small FF80000 -- FF87FFF & \small 255.5MB -- 255.53MB & 32KB &
\multicolumn{1}{p{6cm}|}{VIC-IV Colour RAM (32KB colour RAM models)}\\
\hline
\small FF88000 -- FF8FFFF & \small 255.53MB -- 255.57MB & 32KB &
\multicolumn{1}{p{6cm}|}{Additional VIC-IV Colour RAM (64KB colour RAM models only)}\\
\hline
\small FF90000 -- FFCAFFF & \small 255.53MB -- 255.80MB & 216KB &
\multicolumn{1}{p{6cm}|}{Reserved}\\
\hline
\small FFCB000 -- FFCBFFF & \small 255.80MB -- 255.80MB & 4KB &
\multicolumn{1}{p{6cm}|}{Emulated C1541 RAM}\\
\hline
\small FFCC000 -- FFCFFFF & \small 255.80MB -- 255.81MB & 16KB &
\multicolumn{1}{p{6cm}|}{Emulated C1541 ROM}\\
\hline
\small FFD0000 -- FFD0FFF & \small 255.81MB -- 255.81MB & 4KB &
\multicolumn{1}{p{6cm}|}{C64 \$Dxxx I/O Personality}\\
\hline
\small FFD1000 -- FFD1FFF & \small 255.81MB -- 255.82MB & 4KB &
\multicolumn{1}{p{6cm}|}{C65 \$Dxxx I/O Personality}\\
\hline
\small FFD2000 -- FFD2FFF & \small 255.82MB -- 255.82MB & 4KB &
\multicolumn{1}{p{6cm}|}{MEGA65 \$Dxxx Ethernet I/O Personality}\\
\hline
\small FFD3000 -- FFD3FFF & \small 255.82MB -- 255.82MB & 4KB &
\multicolumn{1}{p{6cm}|}{MEGA65 \$Dxxx Normal I/O Personality}\\
\hline
\small FFD4000 -- FFD5FFF & \small 255.82MB -- 255.83MB & 8KB &
\multicolumn{1}{p{6cm}|}{Reserved}\\
\hline
\small FFD6000 -- FFD67FF & \small 255.83MB -- 255.83MB & 2KB &
\multicolumn{1}{p{6cm}|}{Hypervisor scratch space}\\
\hline
\small FFD6000 -- FFD6BFF & \small 255.83MB -- 255.83MB & 3KB &
\multicolumn{1}{p{6cm}|}{Hypervisor scratch space}\\
\hline
\small FFD6C00 -- FFD6DFF & \small 255.83MB -- 255.83MB & 512 &
\multicolumn{1}{p{6cm}|}{F011 floppy controller sector buffer}\\
\hline
\small FFD6E00 -- FFD6FFF & \small 255.83MB -- 255.83MB & 512 &
\multicolumn{1}{p{6cm}|}{SD Card controller sector buffer}\\
\hline
\small FFD7000 -- FFD70FF & \small 255.83MB -- 255.83MB & 256 &
\multicolumn{1}{p{6cm}|}{MEGAphone r1 I2C peripherals}\\
\hline
\small FFD7100 -- FFD71FF & \small 255.83MB -- 255.83MB & 256 &
\multicolumn{1}{p{6cm}|}{MEGA65 r2 I2C peripherals}\\
\hline
\small FFD7200 -- FFD72FF & \small 255.83MB -- 255.83MB & 256 &
\multicolumn{1}{p{6cm}|}{MEGA65 HDMI I2C registers (only for models fitted
  with the ADV7511 HDMI driver chip)}\\
\hline
\small FFD7300 -- FFD7FFF & \small 255.83MB -- 255.84MB & 3.25KB &
\multicolumn{1}{p{6cm}|}{Reserved for future I2C peripherals}\\
\hline
\small FFD8000 -- FFDBFFF & \small 255.83MB -- 255.86MB & 16KB &
\multicolumn{1}{p{6cm}|}{Hypervisor ROM (only visible in Hypervisor Mode)}\\
\hline
\small FFDC000 -- FFDDFFF & \small 255.86MB -- 255.87MB & 8KB &
\multicolumn{1}{p{6cm}|}{Reserved for Hypervisor Mode ROM expansion}\\
\hline
\small FFDE000 -- FFDE7FF & \small 255.87MB -- 255.87MB & 2KB &
\multicolumn{1}{p{6cm}|}{Reserved for Ethernet buffer expansion}\\
\hline
\small FFDE800 -- FFDEFFF & \small 255.87MB -- 255.87MB & 2KB &
\multicolumn{1}{p{6cm}|}{Ethernet frame read buffer (read only) and
  Ethernet frame write buffer (write only)}\\
\hline
\small FFDF000 -- FFDFFFF & \small 255.87MB -- 255.87MB & 4KB &
\multicolumn{1}{p{6cm}|}{Virtual FPGA registers (selected models only)}\\
\hline
\small FFE0000 -- FFFFFFF & \small 255.87MB -- 256MB & 128KB &
\multicolumn{1}{p{6cm}|}{Reserved}\\
\hline
\end{longtable}

\section{\$D000 -- \$DFFF I/O Personalities}
\label{sec:iopersonalities}

The MEGA65 supports four different I/O personalities.  These are
selected by writing the appropriate values to the \$D02F KEY register,
which is visible in all four I/O personalities.  There is more information in
\bookvref{cha:modes} about the use of the KEY
register.

The following table shows which I/O devices are visible in each of
these I/O modes, as well as the KEY register values that are used to
select the I/O personality.

\newpage
\setlength{\tabcolsep}{3pt}
\begin{longtable}{|p{2.2cm}|C{2cm}|C{2cm}|C{2cm}|C{2cm}|}
\hline
{\bf{HEX}} & {\bf{C64}} & {\bf{C65}} & {\bf{MEGA65 ETHERNET}} & {\bf{MEGA65}} \\
\hline
\endfirsthead
\multicolumn{3}{l@{}}{\ldots continued}\\
\hline
{\bf{HEX}} & {\bf{C64}} & {\bf{C65}} & {\bf{MEGA65 ETHERNET}} & {\bf{MEGA65}} \\
\endhead
\multicolumn{3}{l@{}}{continued \ldots}\\
\endfoot
\multicolumn{5}{p{10.2cm}}{{$^1$} In the C64 I/O personality, \$D030 behaves as on C128, allowing toggling
  between 1MHz and 2MHz CPU speed.}\\
\multicolumn{5}{p{10.2cm}}{{$^2$} The additional MEGA65 SIDs are visible in
  all I/O personalities.}\\
\multicolumn{5}{p{10.2cm}}{{$^3$} Some models may replace the repeated images
  of the first four SIDs with four additional SIDs, for a total of 8 SIDs.}\\
\endlastfoot
\hline
\small KEY & \small \$00 & \$A5, \$96 & \$45, \$54  & \$47, \$53 \\
\hline
\small \$D000 -- \$D02F & \small VIC-II & VIC-II & VIC-II & VIC-II \\
\hline
\small \$D030 -- \$D07F & \small VIC-II{$^1$} & VIC-III & VIC-III & VIC-III \\
\hline
\small \$D080 -- \$D08F & \small VIC-II & F011 & F011 & F011 \\
\hline
\small \$D090 -- \$D09F & \small VIC-II & -- & SD Card & SD Card \\
\hline
\small \$D0A0 -- \$D0FF & \small VIC-II & RAM EXPAND CONTROL & -- & -- \\
\hline
\small \$D100 -- \$D1FF & \small VIC-II & RED Palette & RED Palette &
RED Palette \\
\hline
\small \$D200 -- \$D2FF & \small VIC-II & GREEN Palette & GREEN Palette &
GREEN Palette \\
\hline
\small \$D300 -- \$D3FF & \small VIC-II & BLUE Palette & BLUE Palette &
BLUE Palette \\
\hline
\small \$D400 -- \$D41F & \small SID Right \#1 & SID Right \#1 & SID Right \#1 &
SID Right \#1 \\
\hline
\small \$D420 -- \$D43F & \small SID Right \#2 & SID Right \#2 & SID Right \#2 &
SID Right \#2 \\
\hline
\small \$D440 -- \$D45F & \small SID Left \#1 & SID Left \#1 & SID Left \#1 &
SID Left \#1 \\
\hline
\small \$D460 -- \$D47F & \small SID Left \#2 & SID Left \#2 & SID Left \#2 &
SID Left \#2 \\
\hline
\small \$D480 -- \$D49F & \small SID Right \#1 & SID Right \#1 & SID Right \#1 &
SID Right \#1 \\
\hline
\small \$D4A0 -- \$D4BF & \small SID Right \#2 & SID Right \#2 & SID Right \#2 &
SID Right \#2 \\
\hline
\small \$D4C0 -- \$D4DF & \small SID Left \#1 & SID Left \#1 & SID Left \#1 &
SID Left \#1 \\
\hline
\small \$D4E0 -- \$D4FF & \small SID Left \#2 & SID Left \#2 & SID Left \#2 &
SID Left \#2 \\
\hline
\small \$D500 -- \$D5FF & \small SID images & -- & Reserved & Reserved \\
\hline
\small \$D600 -- \$D63F & \small -- & UART & UART & UART \\
\hline
\small \$D640 -- \$D67F & \small -- & UART images & HyperTrap
Registers & HyperTrap Registers \\
\hline
\small \$D680 -- \$D6FF & \small -- & -- & MEGA65 Devices & MEGA65 Devices \\
\hline
\small \$D700 -- \$D7FF & \small -- & -- & MEGA65 Devices & MEGA65 Devices \\
\hline
\small \$D800 -- \$DBFF & \small COLOUR RAM & COLOUR RAM & ETHERNET Buffer & COLOUR RAM \\
\hline
\small \$DC00 -- \$DDFF & \small CIAs & CIAs / COLOUR RAM & ETHERNET Buffer & CIAs / COLOUR RAM \\
\hline
\small \$DE00 -- \$DFFF & \small CART I/O & CART I/O & ETHERNET Buffer & CART I/O / SD SECTOR \\
\hline
\end{longtable}

\section{CPU Memory Banking}
\label{sec:membanking}

The 45GS10 processor, like the 6502, can only ``see'' 64KB of memory
at a time. Access to additional memory is via a selection of
bank-switching mechanisms.  For backward-compatibility with the C64
and C65, the memory banking mechanisms for both of these computers
existing the MEGA65:
\begin{enumerate}
\item C65-style MAP instruction banking
\item C65-style \$D030 banking
\item C64-style cartridge banking
\item C64-style \$00 / \$01 banking
\end{enumerate}

It is important to understand that these different banking modes have
a priority order: If a higher priority form of banking is being used,
it takes priority over a lower priority form.  The C65 banking methods
take priority of the C64-mode banking methods.  So, for example, if
the 45GS10 MAP instruction has been used to provide a particular
memory layout, the C64-style \$00 / \$01 banking will not be visible.

This makes the overall banking scheme more complex than on the C64.
Thus to understand what the actual memory layout will be, you should
start by considering the effects of C64 memory banking, and then if
any C65 MAP instruction memory banking is enabled, using that to override the
C64-style memory banking. Then if any C65 \$D030 memory banking is
used, that overrides both the C64 and C65 MAP instruction memory
banking. Finally, if I/O is banked, or if there are any cartridges
inserted and active, their effects are made.

The following diagram shows the different types of banking that can
apply to the different areas of the 64KB that the CPU can see.  The
higher layers take priority over the lower layers, as described in the
previous paragraph.
\begin{center}
\begin{tabular}{rccccccc}
\cline{4-5} \cline{7-8}
\rowcolor[HTML]{C0C0C0} I/O/CART &                                 & \multicolumn{1}{c|}{\cellcolor[HTML]{C0C0C0}}                               & \multicolumn{1}{c|}{\cellcolor[HTML]{FE0000}\begin{tabular}[c]{@{}c@{}}CART\\ ROMLO\end{tabular}} & \multicolumn{1}{c|}{\cellcolor[HTML]{FE0000}\begin{tabular}[c]{@{}c@{}}CART\\ ROMHI\end{tabular}} & \multicolumn{1}{c|}{\cellcolor[HTML]{C0C0C0}}                                                      & \multicolumn{1}{c|}{\cellcolor[HTML]{F8A102}I/O}                                                 & \multicolumn{1}{c|}{\cellcolor[HTML]{FE0000}\begin{tabular}[c]{@{}c@{}}CART \\ ROMHI\end{tabular}} \\ \cline{4-8}
\rowcolor[HTML]{EFEFEF}
C65                                                 &                                 & \multicolumn{1}{c|}{\cellcolor[HTML]{EFEFEF}}                               & \multicolumn{1}{c|}{\cellcolor[HTML]{F8FF00}BASIC}                                                & \multicolumn{1}{c|}{\cellcolor[HTML]{F8FF00}BASIC}                                                & \multicolumn{1}{c|}{\cellcolor[HTML]{F8FF00}\begin{tabular}[c]{@{}c@{}}INTER-\\ FACE\end{tabular}} & \multicolumn{1}{c|}{\cellcolor[HTML]{EFEFEF}}                                                   & \multicolumn{1}{c|}{\cellcolor[HTML]{F8FF00}KERNAL}                                                \\ \cline{2-8}
\rowcolor[HTML]{34FF34}
\multicolumn{1}{r|}{\cellcolor[HTML]{C0C0C0}MAP}    & \multicolumn{2}{c|}{\cellcolor[HTML]{34FF34}\begin{tabular}[c]{@{}c@{}}MAP LO\\ (4 x 8KB slabs)\end{tabular}} & \multicolumn{5}{c|}{\cellcolor[HTML]{34FF34}\begin{tabular}[c]{@{}c@{}}MAP HI\\ (4 x 8KB slabs)\end{tabular}}                                                                                                                                                                                                                                                                                                                                                                                                     \\ \cline{2-8}
\rowcolor[HTML]{EFEFEF}
C64                                                 &                                 &                                                                             & \multicolumn{1}{c|}{\cellcolor[HTML]{EFEFEF}}                                                     & \multicolumn{1}{c|}{\cellcolor[HTML]{00D2CB}BASIC}                                                & \multicolumn{1}{c|}{\cellcolor[HTML]{EFEFEF}}                                                      & \multicolumn{1}{c|}{\cellcolor[HTML]{00D2CB}\begin{tabular}[c]{@{}c@{}}CHAR\\ ROM\end{tabular}} & \multicolumn{1}{c|}{\cellcolor[HTML]{00D2CB}KERNAL}                                                \\ \cline{2-8}
\rowcolor[HTML]{9698ED}
\multicolumn{1}{r|}{\cellcolor[HTML]{C0C0C0}RAM}   & \multicolumn{2}{c|}{\cellcolor[HTML]{9698ED}RAM*}                                                             & \multicolumn{1}{c|}{\cellcolor[HTML]{9698ED}RAM}                                                  & \multicolumn{1}{c|}{\cellcolor[HTML]{9698ED}RAM}                                                  & \multicolumn{1}{c|}{\cellcolor[HTML]{9698ED}RAM}                                                   & \multicolumn{1}{c|}{\cellcolor[HTML]{9698ED}RAM}                                                & \multicolumn{1}{c|}{\cellcolor[HTML]{9698ED}RAM}                                                   \\ \cline{2-8}
\rowcolor[HTML]{EFEFEF}
                                                    & \multicolumn{2}{c}{\cellcolor[HTML]{EFEFEF}\begin{tabular}[c]{@{}c@{}}\$0000 --\\ \$7FFF\end{tabular}}        & \begin{tabular}[c]{@{}c@{}}\$8000 --\\ \$9FFF\end{tabular}                                        & \begin{tabular}[c]{@{}c@{}}\$A000 --\\ \$BFFF\end{tabular}                                        & \begin{tabular}[c]{@{}c@{}}\$C000 --\\ \$CFFF\end{tabular}                                         & \begin{tabular}[c]{@{}c@{}}\$D000 --\\ \$DFFF\end{tabular}                                      & \begin{tabular}[c]{@{}c@{}}\$E000 --\\ \$FFFF\end{tabular}
\end{tabular}
\end{center}

(There are actually a few further complications. For example, if the
cartridge selects the UltiMAX(tm) game mode, then only the first 4KB
of RAM will be visible, and the remaining address space will be
un-mapped, and able to be supplied by the cartridge.)

For example, using \$D030 to bank in C65
ROM at \$A000, this will take priority over the C64 BASIC 2 ROM at the
same address.

\section{C64/C65 ROM Emulation}

The C64 and C65 use ROM memories to hold the KERNAL and BASIC system.
The MEGA65 is different: It uses 128KB of its 384KB fast chip RAM at
\$20000 - \$3FFFF (banks 2 and 3) to
hold these system programs. This makes it possible to change or upgrade the
``ROM'' that the MEGA65 is running, without having to open the
computer. It is even possible to use the MEGA65's Freeze Menu to
change the ``ROM'' being used while a program is running.

The C64 and C65 memory banking methods use this 128KB of area when
making ROM banks visible.  When the RAM banks are mapped, they are
always read-only.  However, if the MAP instruction or DMA is used to
access that address area, it is possible to write to it. For improved
backward compatibility, the whole 128KB region of memory is normally
set to read-only.

A program can, however, request read-write access to this
128KB area of memory, so that it can make full use of the MEGA65's
384KB of chip RAM.  This is accomplished by triggering the {\em Toggle
  Rom Write-protect} system trap of the hypervisor.  The following
code-fragment demonstrates how to do this. Calling it a second time
will re-activate the write-protection.

\begin{screenoutput}
  LDA #$70
  STA $D640
  NOP
\end{screenoutput}

This fragment works by
calling sub-function \$70 (toggle ROM write-protect) of Hypervisor
trap \$00. Note that the \screentext{NOP} is mandatory. The MEGA65
I/O personality must be first selected, so that the \$D640 register is
un-hidden.

The current write-protection
state can be tested by attempting to write to this area of memory.
Also, you can examine and toggle the current state from in the MEGA65
Freeze Menu.

NOTE: If you are starting your program from C65-mode, you must first make
sure that the I/O area is visible at \$D000-\$DFFF.  The simplest way to do
this is to use the {\tt MAP} instruction with all zero values in the registers.
The following fragment demonstrates this, and also makes sure that the MEGA65 I/O
context is active, so that the hypervisor trap will be able to trigger:

\begin{screenoutput}

  ; Clear C65 memory map
  LDA #$00
  TAX
  TAY
  TAZ
  MAP
  ; Bank I/O in via C64 mechanism
  LDA #$35
  STA $01
  ; Do MEGA65 / VIC-IV I/O knock
  LDA #$47
  STA $D02F
  LDA #$53
  STA $D02F
  ; End MAP sequence, thus allowing interrupts to occur again
  EOM
  ; Do Hypervisor call to un-write-protect the ROM area
  LDA #$70
  STA $D640
  NOP
\end{screenoutput}


\subsection{C65 Compatibility ROM Layout}

The layout of the C65 compatibility 128KB ROM area is identical to that of the C65:

{\ttfamily
\setlength{\tabcolsep}{3pt}
\begin{tabular}{|l|l|}
\hline
{\bf{HEX}} & {\bf{Contents}} \\
\hline
\$3E000 -- \$3FFFF & C65 KERNAL \\
\hline
\$3C000 -- \$3DFFF & RESERVED \\
\hline
\$38000 -- \$3BFFF & C65 BASIC GRAPHICS ROUTINES \\
\hline
\$32000 -- \$37FFF & C65 BASIC \\
\hline
\$30000 -- \$31FFF & MONITOR (gets mapped at \$6000 -- \$7FFF) \\
\hline
\$2E000 -- \$2FFFF & C64 KERNAL \\
\hline
\$2D000 -- \$2DFFF & C64 CHARSET \\
\hline
\$2C000 -- \$2CFFF & INTERFACE \\
\hline
\$24000 -- \$27FFF & RESERVED \\
\hline
\$20000 -- \$23FFF & DOS (gets mapped at \$8000 -- \$BFFF) \\
\hline
\end{tabular}
}

The INTERFACE program is a series of routines that are used by the C65
to switch between C64-mode, C65-mode and the C65's built-in DOS.  The
DOS is located in the lower-eighth of the ROM.




