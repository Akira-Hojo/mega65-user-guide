\chapter{System Memory Map}
\label{cha:memory-map}
\section{Introduction}

The MEGA65 computer has a large 28-bit address space, which allows it
to address upto 256MiB of memory and memory-mapped devices. 
This memory map has several different views, depending on which mode
the computer is operating in. Broadly, there are five main modes: 
(1) Hypervisor mode; (2) C64 compatibility mode; (3) C65 compatibility mode; (4) UltiMAX
compatibility mode; and (5) MEGA65 mode, or one of the other modes,
where the programmer has made use of MEGA65 enhanced features. 

It is important to understand that, unlike the C128, the C65 and
MEGA65 allow access to all enhanced features from C64 mode, if the
programmer wishes to do so.  This means that while we frequently talk
about ``C64 Mode,'' ``C65 Mode'' and ``MEGA65 Mode,'' these are simply
terms of convenience for the MEGA65 with its memory map (and sometimes
other features) configured to provide an environment that matches
the appropriate mode.  The heart of this is the MEGA65's flexibly
memory map.

In this appendix, we will begin by describing the MEGA65's native
memory map, that is, where all of the memory, IO devices and other
features appear in the 28-bit address space. We will then explain how
C64 and C65 compatible memory maps are accessed from this 28-bit
address space.

\section{MEGA65 Native Memory Map}

\setlength{\tabcolsep}{3pt}
\begin{longtable}{|L{1.2cm}|L{1.1cm}|c|c|c|c|c|c|c|c|}
\hline
{\bf{HEX}} & {\bf{DEC}} & {\bf{Contents}} \\
\hline
\endfirsthead
\multicolumn{3}{l@{}}{\ldots continued}\\
\hline
{\bf{HEX}} & {\bf{DEC}} & {\bf{Contents}} \\
\endhead
\multicolumn{3}{l@{}}{continued \ldots}\\
\endfoot
\hline
\endlastfoot

\hline
\end{longtable}
