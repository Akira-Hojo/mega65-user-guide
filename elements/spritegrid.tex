% This is an element for displaying a sprite in a grid, just like page 70 of the
% commodore manual. This version can be easily expanded. For now it will suffice.
% In order to display a hi-res mono sprite grid use:
%
%	\spritegrid{
%	\hline
%	\spritecells{---------ooooooo--------}
%	\spritecells{-------ooooooooooo------}
%	\spritecells{------ooooooooooooo-----}
%	\spritecells{------ooooo--oooooo-----}
%	\spritecells{-----ooooo-oo--ooooo----}
%	\spritecells{-----ooooo-ooooooooo----}
%	\spritecells{-----ooooo-oo--ooooo----}
%	\spritecells{------ooooo--oooooo-----}
%	\spritecells{------ooooooooooooo-----}
%	\spritecells{------ooooooooooooo-----}
%	\spritecells{------o-ooooooooo-o-----}
%	\spritecells{-------o-ooooooo-o------}
%	\spritecells{-------o--ooooo--o------}
%	\spritecells{--------o--ooo--o-------}
%	\spritecells{--------o--ooo--o-------}
%	\spritecells{---------o--o--o--------}
%	\spritecells{---------o--o--o--------}
%	\spritecells{----------ooooo---------}
%	\spritecells{----------ooooo---------}
%	\spritecells{----------ooooo---------}
%	\spritecells{-----------ooo----------}
%	}
%
% For a multicolour sprite:
%
%	\spritegrid{
%	\hline
%	\spritecells{------------------------}
%	\spritecells{------------------------}
%	\spritecells{------------------------}
%	\spritecells{------------------------}
%	\spritecells{--------llllll----------}
%	\spritecells{------llllllggll--------}
%	\spritecells{------llllllllgg--------}
%	\spritecells{----llllllgggggggg------}
%	\spritecells{----llllggeeeellll------}
%	\spritecells{----lloollllllggee------}
%	\spritecells{----llooggggooggee------}
%	\spritecells{----llooggggooggee------}
%	\spritecells{----eeeeggggooeeee------}
%	\spritecells{----ggeeeeeeeeoo--------}
%	\spritecells{------ggooooooee--------}
%	\spritecells{------eeggeeeeee--------}
%	\spritecells{--------eeeeee----------}
%	\spritecells{------------------------}
%	\spritecells{------------------------}
%	\spritecells{------------------------}
%	\spritecells{------------------------}
%	}

\usepackage{tabulary} %Removes spacing from tabulars
\usepackage[table,dvipsnames]{xcolor}
\usepackage{xstring} % for string substitution
\usepackage{xparse} % used for unpacking the sprite characters
% \renewcommand{\familydefault}{\sfdefault} % default sans font

\usepackage{graphicx} % for resizing the tabular used by spritegrid
\usepackage{subcaption} % used for the left hand subtable of row numbers
\usepackage{multirow} % used for the ``Row'' column
\usepackage{rotating} % used by the rotating ``Row'' word

\newcommand{\spritebytecolumn}[1]{
   %\framebox[4mm]{#1}
   \makebox[4mm]{#1}
}

\setlength\tabcolsep{0.3mm} % the indivdual cell width and height

% The byte numbers at the top of the grid in two jaged rows. 
\newcommand{\spritetopcolumnbytenumbers}{
  \spritebytecolumn{128} & 
  \spritebytecolumn{ } & 
  \spritebytecolumn{32} & 
  \spritebytecolumn{ } &

  \spritebytecolumn{8} & 
  \spritebytecolumn{ } & 
  \spritebytecolumn{2} & 
  \spritebytecolumn{} %\\[-2pt]
}

\newcommand{\spritebottomcolumnbytenumbers}{
  \spritebytecolumn{ } & 
  \spritebytecolumn{64} & 
  \spritebytecolumn{ } & 
  \spritebytecolumn{16} &
  
  \spritebytecolumn{ } & 
  \spritebytecolumn{4} & 
  \spritebytecolumn{ } & 
  \spritebytecolumn{1} 
}


% Cell colour list. Can be expanded for other colours in the sprite grid
\def\blk{\cellcolor{black}}
\def\wht{\cellcolor{white}}
\def\grn{\cellcolor{ForestGreen}}
\def\lgrn{\cellcolor{YellowGreen}}
\def\gry{\cellcolor{Gray}}

\newcounter{lettercounter} % counter for detecting the last cell

% Collect the spritecell list and send it to \ProcessSpriteCell for turning into cells
\NewDocumentCommand{\spritecells}{%
>{\SplitList{}} m }{%
  \ProcessList{#1}{\ProcessSpriteCell}%
}

\NewDocumentCommand{\ProcessSpriteCell}{m}{%
  \stepcounter{lettercounter}% 
    \IfStrEqCase{#1}{
	{o}{\blk}
	{-}{\wht}
	{g}{\grn}
	{l}{\lgrn}
	{e}{\gry}
   }%
   \IfStrEq{\thelettercounter}{24}{\setcounter{lettercounter}{0} \\ \hline}{&}%
}

% Start of the actual spritegrid definition
\newenvironment{spritegrid}[1]{
\begin{table}[h]
\centering
\begin{subtable}{8mm}
\vspace{8mm}
\scalebox{0.76}{
\begin{tabular}{c}
\multirow{24}{12mm}{%
\begin{turn}{90}%
\bfseries\uppercase{Row}%
\end{turn}}\\
 1\\
 2\\
 3\\
 4\\
 5\\
 6\\
 7\\
 8\\
 9\\
 10\\
 11\\
 12\\
 13\\
 14\\
 15\\
 16\\
 17\\
 18\\
 19\\
 20\\
 21
\end{tabular}
}
\end{subtable}%
\begin{subtable}{.8\textwidth}

\setlength{\arrayrulewidth}{1pt}
\scalebox{0.7}{%
\begin{tabular}{  *{3}{p{30mm} }  }
  \center\uppercase{Series\\1} &
  \center\uppercase{Series\\2} &
  \center\uppercase{Series\\3}  
\end{tabular}%
}

\scalebox{0.7}{
\begin{tabular}{  *{24}{p{3.35mm} }  }
	\spritetopcolumnbytenumbers & 
	\spritetopcolumnbytenumbers & 
	\spritetopcolumnbytenumbers \\
	\spritebottomcolumnbytenumbers &
	\spritebottomcolumnbytenumbers &
	\spritebottomcolumnbytenumbers
\end{tabular} 
}

\scalebox{0.7}{
\begin{tabular}{ | *{24}{p{3mm} |}  }
#1
\end{tabular}
}

\scalebox{0.7}{
\begin{tabular}{ *{24}{p{3.35mm}}  }
	1 & & & & 5 & & & & & 10 & & & & & 15 & & & & & 20 & & & & 24 \\
	\multicolumn{24}{c}{\bfseries\uppercase{Column}}
\end{tabular}
}
\end{subtable}
\end{table}
}
% End of the actual spritegrid definition