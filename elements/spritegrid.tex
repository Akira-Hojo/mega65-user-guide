% This is an element for displaying a sprite in a grid, just like page 70 of the
% commodore manual. This version can be easily expanded. For now it will suffice.
% In order to display a hi-res mono sprite grid use:
%
%	\spritegrid{
%	\hline
%	\spritecells{---------ooooo----------}
%	\spritecells{-------ooooooooo--------}
%	\spritecells{------ooooooooooo-------}
%	\spritecells{------ooo--o---oo-------}
%	\spritecells{-----ooo-ooo-ooooo------}
%	\spritecells{-----ooo-ooo-ooooo------}
%	\spritecells{-----ooo---o---ooo------}
%	\spritecells{-----ooo-o-ooo-ooo------}
%	\spritecells{-----ooo-o-ooo-ooo------}
%	\spritecells{-----ooo---o--oooo------}
%	\spritecells{------ooooooooooo-------}
%	\spritecells{------ooooooooooo-------}
%	\spritecells{-------ooooooooo--------}
%	\spritecells{-------o-ooooo-o--------}
%	\spritecells{--------o-o-o-o---------}
%	\spritecells{--------o--o--o---------}
%	\spritecells{---------o-o-o----------}
%	\spritecells{---------o-o-o----------}
%	\spritecells{---------ooooo----------}
%	\spritecells{---------ooooo----------}
%	\spritecells{----------ooo-----------}
%	}
%
% For a multicolour sprite:
%
%	\spritegrid{
%	\hline
%	\spritecells{------------------------}
%	\spritecells{------------------------}
%	\spritecells{------------------------}
%	\spritecells{------------------------}
%	\spritecells{llllllllllllllllllllllll}
%	\spritecells{llllllllllllllllllllllll}
%	\spritecells{lllllleeeeeelleeeeeeeell}
%	\spritecells{lllloooooooollooooooooll}
%	\spritecells{llllooggggggllooggggggll}
%	\spritecells{lllloolllllllloollllllll}
%	\spritecells{llllooeeeellllooeeeellll}
%	\spritecells{lllloooooooollooooooooll}
%	\spritecells{llllooggggoollggggggooll}
%	\spritecells{lllloolllloollllllllooll}
%	\spritecells{llllooeeeeoolleeeeeeooll}
%	\spritecells{llllggooooggllooooooggll}
%	\spritecells{llllllggggllllggggggllll}
%	\spritecells{llllllllllllllllllllllll}
%	\spritecells{------------------------}
%	\spritecells{------------------------}
%	\spritecells{------------------------}
%	}

\usepackage{tabulary} %Removes spacing from tabulars
\usepackage{xstring} % for string substitution
\usepackage{xparse} % used for unpacking the sprite characters
% \renewcommand{\familydefault}{\sfdefault} % default sans font

%\usepackage{graphicx} % for resizing the tabular used by spritegrid
\usepackage{subcaption} % used for the left hand subtable of row numbers
\usepackage{multirow} % used for the ``Row'' column
\usepackage{rotating} % used by the rotating ``Row'' word

\newcommand{\spritebytecolumn}[1]{
   %\framebox[4mm]{#1}
   \makebox[4mm]{#1}
}

\setlength\tabcolsep{0.3mm} % the indivdual cell width and height


% Cell colour list. Can be expanded for other colours in the sprite grid
\def\blk{\cellcolor{black}}
\def\wht{\cellcolor{white}}
\def\grn{\cellcolor{ForestGreen}}
\def\lgrn{\cellcolor{YellowGreen}}
\def\gry{\cellcolor{Gray}}

\newcounter{lettercounter} % counter for detecting the last cell

% Collect the spritecell list and send it to \ProcessSpriteCell for turning into cells
\NewDocumentCommand{\spritecells}{%
>{\SplitList{}} m }{%
  \ProcessList{#1}{\ProcessSpriteCell}%
}

\NewDocumentCommand{\ProcessSpriteCell}{m}{%
  \stepcounter{lettercounter}%
    \IfStrEqCase{#1}{
	{o}{\blk}
	{-}{\wht}
	{g}{\grn}
	{l}{\lgrn}
	{e}{\gry}
   }%
   \IfStrEq{\thelettercounter}{24}{\setcounter{lettercounter}{0} \\ \hline}{&}%
}

% Start of the actual spritegrid definition
\newcommand{\spritegrid}[1]{
\begin{table}[h!]
\centering
\begin{subtable}{28mm}
\vspace{8mm}
\scalebox{0.76}{
\begin{center}
\begin{tabular}{p{25mm} p{4mm} c}
\multirow{21}{*}{ } &
\multirow{21}{*}{%
\begin{turn}{90}%
\bfseries\uppercase{Row}%
\end{turn}} &
 1\\
& & 2\\
& & 3\\
& & 4\\
& & 5\\
& & 6\\
& & 7\\
& & 8\\
& & 9\\
& & 10\\
& & 11\\
& & 12\\
& & 13\\
& & 14\\
& & 15\\
& & 16\\
& & 17\\
& & 18\\
& & 19\\
& & 20\\
& & 21
\end{tabular}
\end{center}
}
\end{subtable}%
\begin{subtable}{.8\textwidth}

\setlength{\arrayrulewidth}{1pt}
\scalebox{0.7}{
\begin{center}
\begin{tabular}{  *{3}{p{30mm} }  }
  \center\uppercase{Series\\1} &
  \center\uppercase{Series\\2} &
  \center\uppercase{Series\\3}
\end{tabular}
\end{center}
}

\scalebox{0.7}{
\begin{center}
\begin{tabular}{p{5.8mm} *{11}{C{7.3mm}}}
   128 & 32 & 8 & 2 & 128 & 32 & 8 & 2 & 128 & 32 & 8 & 2
\end{tabular}
\end{center}
}
\\[-1.5mm]
\scalebox{0.7}{
\begin{center}
\begin{tabular}{p{1.7mm} *{12}{C{7.3mm}}}
  & 64 & 16 & 4 & 1 & 64 & 16 & 4 & 1 & 64 & 16 & 4 & 1
\end{tabular}
\end{center}
}

\scalebox{0.7}{
\begin{center}
\begin{tabular}{ | *{24}{p{3mm} |}  }
#1
\end{tabular}
\end{center}
}

\scalebox{0.7}{
\begin{center}
\begin{tabular}{ *{24}{p{3.35mm}}  }
	\spritebytecolumn{1} & & & &
	\spritebytecolumn{5} & & & & &
	\spritebytecolumn{10} & & & & &
	\spritebytecolumn{15} & & & & &
	\spritebytecolumn{20} & & & &
	\spritebytecolumn{24} \\
	\multicolumn{24}{c}{\bfseries\uppercase{Column}}
\end{tabular}
\end{center}
}
\end{subtable}
\end{table}
}

% End of the actual spritegrid definition

