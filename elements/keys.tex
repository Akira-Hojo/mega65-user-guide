% This is a `keys' element for displaying a Mega65 keyboard key
% using a black filled label with rounded edges.
% In order to display a key as a title, use:
%
%     \megakey[title]{Run/Stop}
%
% For displaying a key as a part of the normal document flow, simply use:
%
%    \megakey{Shift}
%
%
% If you get warnings on special characters, mathematical characters etc, use $, eg:
%
%    \megakey{$\leftarrow$}
%
% Other sizes are supported, as part of tcolorbox:
% http://mirror.aarnet.edu.au/pub/CTAN/macros/latex/contrib/tcolorbox/tcolorbox.pdf#subsubsection.4.7.5 however, only `title' and the default: `small' are proposed for use in this manual.
%
% The second macro available here is the megasymbolkey.
% This will display the MEGA symbol as white on a black key box. Simply use:
%
%		 \megasymbolkey
%
% Some MEGA65 keys contain two lines of text like "RUN/STOP"
% You can use the specialkey macro for this:
%
%    \specialkey{SHIFT LOCK}%

\usepackage{tcolorbox}

\newtcbox{\megakeyinner}[1][small]{colback=black, coltext=white, size=#1, fontupper=\bfseries, nobeforeafter,box align=bottom,baseline=3pt,text height=7pt}
\newcommand{\megakey}[2][small]{\megakeyinner[#1]{\uppercase{#2}}}

% Previous version of megasymbolkey
%\newtcbox{\megasymbolkeyinner}{colback=black, coltext=white, clip title=false. fontupper=\symbolfont, box align=bottom,baseline=3pt,text height=7pt}
%\newcommand{\megasymbolkey}{\megakeyinner{\megasymbol[white]}\ }

\newtcolorbox{megasymbolkeyinner}
{colback=black,coltext=white,size=small,fontupper=\small\bfseries,
width=0.65cm, height=0.55cm, box align=base,
nobeforeafter, halign=flush left, left=0mm,top=0.3mm,bottom=0mm,right=0mm
,boxsep=0.5mm,baseline=4pt, enlarge right by = 1mm
}
\newcommand{\megasymbolkey}{
\begin{megasymbolkeyinner}%
\megasymbol[white]%
\end{megasymbolkeyinner}%
}

\newtcolorbox{specialkeyinner}
{colback=black,coltext=white,size=small,fontupper=\tiny\bfseries,
width=0.65cm, height=0.55cm, box align=base,
nobeforeafter, halign=flush left, left=0mm,top=0.3mm,bottom=0mm,right=0mm
,boxsep=0.5mm,baseline=4pt
}
\newcommand{\specialkey}[1]{
\begin{specialkeyinner}%
#1%
\end{specialkeyinner}%
}
