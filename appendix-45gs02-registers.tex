\chapter{45GS02 Microprocessor}

\section{Introduction}

The 45GS02 is an enhanced version of the processor portion of the CSG4510
and of the F018 ``DMAgic'' DMA controller used in the Commodore 65 computer prototypes.  The 4510 is, in turn,
an enhanced version of the 65CE02.  
The reader is referred to
the considerable documentation available for the 6502 and 65CE02 processors
for the backwards-compatibile operation of the 45GS02.

This chapter will
focus on the differences between the 45GS02 and the earlier 6502-class
processors, and the documentation of the many built-in memory-mapped IO
registers of the 45GS02.

\section{Differences to Earlier 6502-Class Processors}

The 45GGS02 has a number of key differences to earlier 6502-class processors:

\subssection{6502 Illegal Opcodes}

The 65C02, 65CE02 and CSG4510 processors extended the original 6502 processor
by using previously unallocated opcodes of the 6502 to provide additional
instructions.  All software that followed the official documentation of the 6502
processor will therefore work on these newer processors, possibly with different
instruction timing.  However, the common practice on the C64 and other home computers
of using undefined opcodes (often called ``illegal opcodes'', although there is no
law against using them), means that many existing programs will not work on these
newer processors.

To alieviate this problem the 45GS02 has the ability to switch processor personalities
between the 4510 and 6502.  The effect is that in 6502 mode, none of the new opcodes of
the 65C02, 65CE02, 4510 or 45GS02 are available, and are replaced with the original,
often strange, behaviour of the undefined opcodes of the 6502.

Status: Incomplete.  Most undocumented 6502 opcodes do not operate correctly when the 6502
personality is enabled.

\subsection{Read-Modify-Write Instruction Bug Compatibility}

The 65CE02 processor optimised a group of instructions called the Read-Modify-Write (RMW)
instructions.  For such instructions, such as INC, that increments the contents of a memory
location, the 6502 would read the original value and then write it back unchanged, before
writing it back with the new increased value.  For most purposes, this did not cause any
problems. However, it turned out to be a fast way to acknowledge VIC-II interrupts, because
writing the original value back (which the instruction doesn't need to do) acknowledges
the interrupt.  This method is faster and uses fewer bytes than any alternative, and so
became widely used in C64 software.

The problem came with the C65 with its 65CE02 derived CSG4510 that didn't do this extra write
during the RMW instructions.  This made the RMW instructions one cycle faster, which made
software run slightly faster. Unfortunately, it also meant that a lot of existing C64 software
simply won't run on a C65, unless the interrupt acknowledgement code in each program is patched
to work around this problem. This is the single most common reason why many C64 games and other
software titles won't run on a C65.

Because this problem is so common, the MEGA65's 45GS02 includes bug compatibility with this
commonly used feature of the original 6502.  It does this by checking if the target of an RMW
instruction is \$D019, i.e., the interrupt status register of the VIC-II.  If it is, then
the 45GS02 performs the dummy write, allowing many C64 software titles to run unmodified on the
MEGA65, that do not run on a C65 prototype.  By only performing the dummy write if the address
is \$D019, the MEGA65 maintains C64 compatibility, without sacrificing the speed improvement
for all other uses of these instructions.

\subsection{Variable CPU Speed}

The 45GSG02 is able to run at ~1MHz, ~2MHz, ~3.5MHz and 40MHz, to support running software
designed for the C64, C128 in C64 mode, C65 and MEGA65.  It is also possible to more smoothly
vary the CPU speed using the {\bf SPEEDBIAS} register located at \$F7FA, when MEGA65 IO mode
is enabled.

\subsubsection{Slow (1MHz -- 3.5MHz) Operation}
In these modes, the 45GS02 processor slows down, so that the same number of instructions
per video frame are executed as on a PAL or NTSC C64, C128 in C64 mode or C65 prototype.
This is to allow existing software to run on the MEGA65 at the correct speed, and with
minimal display problems.  The VIC-IV video controller provides cycle indication pulses
to the 45GS02 that are used to keep time.

In these modes, opcodes take the same number of cycles as an 6502.  However memory accesses within an
instruction are not guaranteed to occur in the same cycle as on a 1MHz 6502.  Normally
the effect is that instructions complete faster, and the processor idles until the
correct number of cycles have passed. This means that timing may be incorrect by upto
7 micro-seconds.  This is not normally a problem, and even many C64 fast loaders will
function correctly. For example, the GEOS(tm) Graphical Operating System for the C64
can be booted and used from a 1541 connected to the MEGA65's serial port.

However, some advanced VIC-II graphics tricks, such as Variable Screen Position (VSP) are
highly unlikely to work correctly, due to the uncertainty in timing of the memory write
cycles of instructions.  However, in most cases such problems can be easily solved by using
the advanced features of the MEGA65's VIC-IV video controller.  For example, VSP is unnecessary
on the MEGA65, because you can set the screen RAM address to any location in memory.

\subsubsection{Full Speed (40MHz) Instruction Timing}

When the MEGA65's processor is operating at full speed (currently 40MHz), the instruction
timing no longer exactly mirrors the 6502: Instructions that can be executed in fewer cycles
will do so. For example, branches are typically require fewer instructions on the 45GS02.
There are also some instructions that require more cycles on the 45GS02, in particular the
LDA, LDX, LDY and LDZ instructions. Those instructions typically require one additional cycle.
However as the processor is running at 40MHz, these instructions still execute much more quickly
than on even a C65 or C64 with an accelerator.

\section{C64 CPU Memory Mapped Registers}

\input{regtable_CPU.C64}

\section{MEGA65 CPU Memory Mapped Registers}

\input{regtable_CPU.MEGA65}

\section{F018 ``DMAgic'' DMA Controller}

\input{regtable_DMA.C65}

\section{MEGA65 DMA Controller Extensions}

\input{regtable_DMA.MEGA65}

\section{MEGA65 CPU Math Unit Registers}

\input{regtable_MATH.MEGA65}

\section{MEGA65 Hypervisor Mode Memory Mapped Registers}

\input{regtable_HCPU.MEGA65}
