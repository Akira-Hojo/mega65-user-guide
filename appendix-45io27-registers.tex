\chapter{45IO27 Multi-Function IO Controller}

\section{Overview}

The 45IO27 is a multi-purpose IO controller that incorporates the functions of the
C65's F011 floppy controller, together with the MEGA65's SD card controller interface,
and a number of other miscellation IO functions.

Each of these major functions is covered in a separate section of this chapter

\section{F011-compatible Floppy Controller}

The MEGA65 computer is one of very few modern computers that still
includes first-class support for magnetic floppy drives.  It includes
a floppy controller that is backwards compatible with the C65's F011D
floppy drive controller.

However, unlike the F011D, the MEGA65's
floppy disk controller supports HD and ED media, and similar to the
1541 floppy drive, it also supports variable data rates, so that a
determined user could develop disk formats that store more data,
includ robust copy protection schemes, or both.

GCR encoding is not currently supported, but may be supported by a
future revision of the controller.  It may also be possible with some
creativity and effort to use the debug register interface to read
double-density GCR formatted media.  This is because there are debug
registers that can be queried to indicate the gap between each
successive magnetic domain -- which is sufficient to decode any disk
format. 

\subsection{Multiple Drive Support}

XXX - two interal drives
XXX - provisional support for 8 real drives via a future C1565-style
interface.
XXX - Drive swap support

\subsection{Reading From Disks}

XXX - byte-by-byte
XXX - sector-buffer
XXX - auto-tune
XXX - example code to read a sector

\subsection{Writing To Disks}

The current revision of the 45io27 does not support writing to disks.
This will be corrected in a future revision.  However, the determined
user may still be able to write to disks by manipulating the WGATE and
WDATA signals of the \$D6A0 register.  However, this is not
recommended.

\subsection{Using HD, ED Drives and/or variable Data Rates}

The mechanism for using high-density (HD) and extended-density (ED)
media and variable data rates are all identical on the 45IO27:  You
simply set the interval rate (DATARATE) in the \$D6A2 register.

The register is used as a divisor of the system bus frequency
(40.5MHz).  Thus for 1581-compatible double-density (DD) media which
uses a data rate of 500KHz, this register should contain:

$40,500,000 / 500,000 = 81$

High-density media uses a 1MHz data rate, and thus should contain 40,
and extended-density media with its 2MHz data rate should have this
field set to 20.  Depending on the HD or ED drive used, it may be
necessary to clear or assert the DENSITY signal (\$D6A0 bit 7), or
take other specific ations.  However, in our experience, most HD
drives do not require any manual manipulation of this signal: Instead
simply set the DATARATE register to the correct value, and begin
accessing the disk.

Note that the DATARATE register takes effect instantaneously.  This
means that it is possible to change the data rate for different
tracks, sectors, or even bytes within sectors.

This means that it
would be possible to, for example, use standard 500KHz DD encoding for
the directory track and one data track, and then switch to HD encoding for the other 79
tracks of a disk. The result would be a disk that could contain a
boot-loader programme and directory that can be read in a 1581, and
that could be used to switch to the faster and higher-density HD
encoding for the remaining data tracks.  This could even be done for
half of a disk, so that when used in a 1581, it loads at the DD speed,
but when inserted in a MEGA65, it uses the HD data tracks, allowing
the programme to load twice as fast, and fit twice as much data.

You are really only limited by your imagination, available time, and
the limited number of people who are still interested in inserting a
floppy disk into their computer!

\section{SD Card Controller and F011 Virtualisation Functions}

For those situations where you do not wish to use real floppy disks,
the 45io27 supports two complementary alternative modes:

\begin{itemize}
\item SD-Card Based Disk Image Access.
\item Virtualised Disk Image Access.
\end{itemize}

\subsection{SD-Card Based Disk Image Access}

XXX - D81 and ``MEGA Disk'' images (800KiB and 64MiB)

\subsection{F011 Virtualisation}

In addition to allowing automatic read and write access to SD-card
based D81 images, it is possible to connect a programme to the serial
monitor interface that provides and accepts data as though it were the
floppy disk.

This is commonly used in a cross-development
environment, where you wish to frequently modify a disk image that is
used by a programme you are developing -- without the need to
continually push new versions of the disk image on the MEGA65's
SD-Card first. It also has the added benefit that it allows you to
easily visualise which sectors are being read from and written to,
which can help speed up development and debugging of your programme.

This function operates together with the MEGA65's Hypervisor by
triggering hyperrupts (that is, interrupts that activate the
Hypervisor).  There is then special code in the Hypervisor that
communicates with the {\tt m65} programme via the serial monitor
interface.

If that all sounds rather complex, all you need to know is that to use
this function, you run the {\tt m65} utility with arguments like
{\tt -d image.d81}.  This should automatically establish the link with
the MEGA65.  If the BASIC interprettor stops responding, press the
reset button (not the power switch) on the left side of your MEGA65,
and it should return to the BASIC's {\tt READY.} prompt -- and if your
supplied disk image has a C65 auto-boot function, then it should
automatically start booting.

This function works very well if the host computer runs Linux, and
will allow loading at a speed of around 60KiB per second.  However, it
may be much slower on Windows or Apple OSX-based systems.

Of course to use this, you will also need an interface module and/or
cable to connect your cross-development system to the MEGA65's serial
monitor interface. This is most easily done using a Trenz TE0790-03
JTAG adapter and mini-USB cable.

More information on using this interface and the {\tt m65} tool can be
found in \bookvref{cha:transfer-and-debug-tools}.

\section{Touch Panel Interface}

\section{Audio Support Functions}

\section{Miscellaneous IO Functions}

