\chapter{How Computers Work}

Computers can do amazing things, and you can make them do amazing things for you, too.
But to do that, you need to understand how computers work.  This can be hard to find
out these days, because computers are now so complicated, that it isn't obvious how
computers work anymore, just by looking at them and using them.  The MEGA65 is designed
to be simple enough that you can learn how computers work as you use it.  But we don't
want to leave you to have to work out everything for yourself.  That is why we have
written this chapter, so that you can learn how computers work, and then use that knowledge
to help you make computers do what \emph{you} want them to do.

\phantomsection
\section{Computers are just a pile of switches}

What are computers \emph{really}? Well, the answer to that question is quite simple, if a little
surprising: Computers really are just made of lots of switches.  These switches work very similar
to the switches you use to turn a light on or off.  Light switches connect or disconnect the
power supply to a light.  The switches in computers connect or disconnect circuits in the computer
to power. But computers can do a lot more than just turn on and off, so something else must be
going on. That something else is also a bit of a surprise: A computer can turn its own switches
on and off by itself.  Let's explore how this simple idea of switches that can turn themselves on
and off makes a computer.

As we have just heard, computers are full of switches.  Obviously those switches must be tiny, and they
are.  When you hear people talking about microns and nanometres with regard to computers, they are often
talking about the size of the little switches that make up the computer chips.  These switches are called
transistors. The switches in the main
chip of the MEGA65, for example, are 28 nanometres long.  That's about 100,000 times skinnier than a single
strand of hair.  This is good, because a computer might need millions or billions of switches in its design.

