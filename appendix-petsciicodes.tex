
\chapter{Special Keyboard Controls and Sequences}


\section{PETSCII Codes and CHR\$}

\label{appendix:asciicodes}

You can use the \screentext{PRINT CHR\$(X)} statement to print a character.
Below is the full table of PETSCII codes you can print by index.  For example, by
using index 65 from the table below as: \screentext{PRINT CHR\$(65)} you will
print the letter \screentext{A}.

You can also do the reverse with the ASC statement.  For example:
\screentext{PRINT ASC("A")} will output \screentext{65}, which matches with the
code in the table.

\begin{adjustwidth}{}{-2cm}
\begin{multicols}{4}
\begin{description}[align=left,labelwidth=0.2cm]
    \item [0]
    \item [1]
    \item [2]   \small{UNDERLINE ON}
    \item [3]
    \item [4]
    \item [5]   \small{WHITE}
    \item [6]
    \item [7]   \small{BELL}
    \item [8]
    \item [9]
    \item [10]  \small{LINEFEED}
%   \item [11]  \footnotesize{DISABLE \specialkey{SHIFT}\megasymbolkey}
    \item [11]  DISABLE \\ \specialkey{SHIFT}\megasymbolkey
%   \item [12]  \footnotesize{ENABLE \specialkey{SHIFT}\megasymbolkey}
    \item [12]  ENABLE \\ \specialkey{SHIFT}\megasymbolkey
    \item [11]
    \item [12]
    \item [13]  \megakey{RETURN}
    \item [14]  \small{LOWER CASE}
    \item [15]  \small{BLINK ON}
    \item [16]
    \item [17]  \megakey{$\downarrow$}
    \item [18]  \specialkey{RVS ON}
    \item [19]  \specialkey{CLR HOME}
    \item [20]  \specialkey{INST DEL}
    \item [21]
    \item [22]
    \item [23]
    \item [24]
    \item [25]
    \item [26]
    \item [27]  \small{ESCAPE}
    \item [28]  \small{RED}
    \item [29]  \megakey{$\rightarrow$}
    \item [30]  \small{GREEN}
    \item [31]  \small{BLUE}
    \item [32]  \megakey{SPACE}
    \item [33]  !
    \item [34]  "
    \item [35]  \#
    \item [36]  \$
    \item [37]  \%
    \item [38]  \&
    \item [39]  '
    \item [40]  (
    \item [41]  )
    \item [42]  *
    \item [43]  +
    \item [44]  ,
    \item [45]  -
    \item [46]  .
    \item [47]  /
    \item [48]  0
    \item [49]  1
    \item [50]  2
    \item [51]  3
    \item [52]  4
    \item [53]  5
    \item [54]  6
    \item [55]  7
    \item [56]  8
    \item [57]  9
    \item [58]  :
    \item [59]  ;
    \item [60]  <
    \item [61]  =
    \item [62]  >
    \item [63]  ?
    \item [64]  @
    \item [65]  A
    \item [66]  B
    \item [67]  C
    \item [68]  D
    \item [69]  E
    \item [70]  F
    \item [71]  G
    \item [72]  H
    \item [73]  I
    \item [74]  J
    \item [75]  K
    \item [76]  L
    \item [77]  M
    \item [78]  N
    \item [79]  O
    \item [80]  P
    \item [81]  Q
    \item [82]  R
    \item [83]  S
    \item [84]  T
    \item [85]  U
    \item [86]  V
    \item [87]  W
    \item [88]  X
    \item [89]  Y
    \item [90]  Z
    \item [91]  [
    \item [92]  \pounds
    \item [93]  ]
    \item [94]  $\uparrow$
    \item [95]  $\leftarrow$
    \item [96]  \graphicsymbol{C}
    \item [97]  \graphicsymbol{A}
    \item [98]  \graphicsymbol{B}
    \item [99]  \graphicsymbol{C}
    \item [100] \graphicsymbol{D}
    \item [101] \graphicsymbol{E}
    \item [102] \graphicsymbol{F}
    \item [103] \graphicsymbol{G}
    \item [104] \graphicsymbol{H}
    \item [105] \graphicsymbol{I}
    \item [106] \graphicsymbol{J}
    \item [107] \graphicsymbol{K}
    \item [108] \graphicsymbol{L}
    \item [109] \graphicsymbol{M}
    \item [110] \graphicsymbol{N}
    \item [111] \graphicsymbol{O}
    \item [112] \graphicsymbol{P}
    \item [113] \graphicsymbol{Q}
    \item [114] \graphicsymbol{R}
    \item [115] \graphicsymbol{S}
    \item [116] \graphicsymbol{T}
    \item [117] \graphicsymbol{U}
    \item [118] \graphicsymbol{V}
    \item [119] \graphicsymbol{W}
    \item [120] \graphicsymbol{X}
    \item [121] \graphicsymbol{Y}
    \item [122] \graphicsymbol{Z}
    \item [123] \graphicsymbol{+}
    \item [124] \graphicsymbol{-}
    \item [125] \graphicsymbol{B}
    \item [126] \graphicsymbol{\textbackslash}
    \item [127] \graphicsymbol{]}
    \item [128]
    \item [129] \small{ORANGE}
    \item [130] \small{UNDERLINE OFF}
    \item [131]
    \item [132]
    \item [133] F1
    \item [134] F3
    \item [135] F5
    \item [136] F7
    \item [137] F2
    \item [138] F4
    \item [139] F6
    \item [140] F8
    \item [141] \specialkey{SHIFT}\megakey{RETURN}
    \item [142] \small{UPPERCASE}
    \item [143] \small{BLINK OFF}
    \item [144] \small{BLACK}
    \item [145] \megakey{$\uparrow$}
    \item [146] \specialkey{RVS OFF}
    \item [147] \specialkey{CLR HOME}
    \item [148] \specialkey{INST DEL}
    \item [149] \small{BROWN}
    \item [150] \small{LT. RED}
    \item [151] \small{DK. GRAY}
    \item [152] \small{GRAY}
    \item [153] \small{LT. GREEN}
    \item [154] \small{LT. BLUE}
    \item [155] \small{LT. GRAY}
    \item [156] \small{PURPLE}
    \item [157] \megakey{$\leftarrow$}
    \item [158] \small{YELLOW}
    \item [159] \small{CYAN}
    \item [160] \megakey{SPACE}
    \item [161] \graphicsymbol{k}
    \item [162] \graphicsymbol{i}
    \item [163] \graphicsymbol{t}
    \item [164] \graphicsymbol{[}
    \item [165] \graphicsymbol{g}
    \item [166] \graphicsymbol{=}
    \item [167] \graphicsymbol{m}
    \item [168] \graphicsymbol{/}
    \item [169] \graphicsymbol{?}
    \item [170] \graphicsymbol{v}
    \item [171] \graphicsymbol{q}
    \item [172] \graphicsymbol{d}
    \item [173] \graphicsymbol{z}
    \item [174] \graphicsymbol{s}
    \item [175] \graphicsymbol{n}
    \item [176] \graphicsymbol{a}
    \item [177] \graphicsymbol{e}
    \item [178] \graphicsymbol{r}
    \item [179] \graphicsymbol{w}
    \item [180] \graphicsymbol{h}
    \item [181] \graphicsymbol{j}
    \item [182] \graphicsymbol{l}
    \item [183] \graphicsymbol{y}
    \item [184] \graphicsymbol{u}
    \item [185] \graphicsymbol{p}
    \item [186] \graphicsymbol{\{}
    \item [187] \graphicsymbol{f}
    \item [188] \graphicsymbol{c}
    \item [189] \graphicsymbol{x}
    \item [190] \graphicsymbol{v}
    \item [191] \graphicsymbol{b}
\end{description}
\end{multicols}
\end{adjustwidth}
NOTE: Codes for 192 to 223 are the equal to 96-127. Codes 224 to 254 equal to 160-190 and code 255 equal to 126.
\newpage



\section{Control codes}
\label{appendix:controlcodes}

\begin{center}
\setlength{\def\arraystretch{1.5}\tabcolsep}{6pt}
\begin{longtable}{c|L{5.5cm}}
	\textbf{Keyboard Control} & \textbf{Function}\\
   \hhline{==}
	\endhead

  \multicolumn{2}{l}{\textbf{Colours}} \\
  \hhline{==}
\megakey{CTRL} + \megakey{1} to \megakey{8} &
Choose from the first range of colours.\\
\hline
\megasymbolkey + \megakey{1} to \megakey{8} &
Choose from the second range of colours.\\
\hline
\megakey{CTRL} + \megakey{E} &
Restores the colour of the cursor back to the default white.\\
  \hhline{==}
  \multicolumn{2}{l}{\textbf{Tabs}} \\
  \hhline{==}
\megakey{CTRL} + \megakey{Z} &
Tabs the cursor to the left. When there are no more tab positions, the cursor will remain at the start of the line.\\
\hline
\megakey{CTRL} + \megakey{I} &
Tabs the cursor to the right. When there are no more tab positions, the cursor will remain at the end of the line.\\
\hline
\megakey{CTRL} + \megakey{X} &
Sets or clears the current screen column as a tab position.
 Use \megakey{CTRL} + \megakey{I} and \megakey{Z} to jump to all positions set with \megakey{X}.\\
  \hhline{==}
  \multicolumn{2}{l}{\textbf{Movement}} \\
  \hhline{==}
\megakey{CTRL} + \megakey{Q} &
Moves the cursor down one line at a time. This is the same function produced by the \megakey{$\downarrow$} key.\\
\hline
\megakey{CTRL} + \megakey{J} &
Moves the cursor down a row. If you are on a long line of BASIC code that has extended to two lines, then the cursor will move down two positions to point to the next line.\\
\hline
\megakey{CTRL} + \megakey{]} &
The same function as \megakey{$\rightarrow$}.\\
\hline
\megakey{CTRL} + \megakey{T} &
Backspace the character immediately to the left and to shift all rightmost characters one position to the left. This is the same function as the \specialkey{INST DEL} key.\\
\hline
\megakey{CTRL} + \megakey{U} &
Moves cursor back to the start of the previous word, or unbroken string of characters. If there are no characters between the current cursor position and the start of the line, the cursor will move to the first column of the current line.\\
\hline
\megakey{CTRL} + \megakey{W} &
Advances cursor forward to the start of the next word, or unbroken string of characters. If there are no characters between the current cursor position and the end of the line, the cursor will move to the first column of the next line.\\
\hline
\megakey{CTRL} + \megakey{P} &
Scroll BASIC listing down one line, equivalent to \megakey{F9} key.\\
\hline
\megakey{CTRL} + \megakey{V} &
Scroll BASIC listing up one line, equivalent to \megakey{F11} key.\\
\hline
\megakey{CTRL} + \megakey{M} &
Performs a carriage return, the same function as the \megakey{Return} key.\\
  \hhline{==}
  \multicolumn{2}{l}{\textbf{Miscellaneous}} \\
  \hhline{==}
\hline
\megakey{CTRL} + \megakey{G} &
Produces a bell tone.\\
\hline
\megakey{CTRL} + \megakey{B} &
Turns on underline text mode. Turn off underline mode by pressing \megakey{ESC} then \megakey{O}.\\
\hline
\megakey{CTRL} + \megakey{N} &
Changes the text case mode from uppercase to lowercase.\\
\hline
\megakey{CTRL} + \megakey{K} &
Locks the uppercase/lowercase mode switch usually performed with \megasymbolkey and \megakey{Shift} keys.\\
\hline
\megakey{CTRL} + \megakey{L} &
Enables the uppercase/lowercase mode switch that is performed with the \megasymbolkey and \megakey{Shift} keys.\\
\hline
\megakey{CTRL} + \megakey{[} &
Equivalent to pressing the \megakey{ESC} key.\\
\hline
\megakey{CTRL} + \megakey{*} &
Enters the Matrix Mode Debugger.\\
\hline

\end{longtable}
\end{center}


\section{Shifted codes}
\label{appendix:shiftedcodes}

\begin{center}
\setlength{\def\arraystretch{1.5}\tabcolsep}{6pt}
\begin{longtable}{c|L{5.5cm}}
	\textbf{Keyboard Control} & \textbf{Function}\\
  \hhline{==}
	\endhead

\megakey{Shift} + \specialkey{INST DEL} &
Insert a character in the current cursor position and move all characters to the right by one position.\\
\hline
\megakey{Shift} + \specialkey{HOME} &
Clear home, clear the entire screen and move the cursor to the home position.\\
\hline

\end{longtable}
\end{center}


\newpage



\section{Escape Sequences}
\label{appendix:escapesequences}

To perform an Escape Sequence, press and release the \megakey{ESC} key. Then press one of the following keys to perform the sequence:

\begin{center}
\setlength{\def\arraystretch{1.5}\tabcolsep}{6pt}
\begin{longtable}{c|L{5.5cm}}
	\textbf{Key} & \textbf{Sequence}\\
  \hhline{==}
	\endhead

\megakey{ESC} + \megakey{X} &
Clears the screen and toggles between 40 and 80 column modes.\\
\hline
\megakey{ESC} + \megakey{@} &
Clears the screen starting from the cursor to the end of the screen.\\
\hline
\megakey{ESC} + \megakey{A} &
Enables the auto-insert mode. Any keys pressed will insert before other characters.\\
\hline
\megakey{ESC} + \megakey{C} &
Disables auto-insert mode, going back to overwrite mode.\\
\hline
\megakey{ESC} + \megakey{T} &
Set top-left window area of the screen at the cursor position. All typed characters and screen activity will be restricted to the area. Also see \megakey{ESC} then \megakey{B}.\\
\hline
\megakey{ESC} + \megakey{B} &
Sets the bottom-right window area of the screen at the cursor position. All typed characters and screen activity will be restricted to the area. Also see \megakey{ESC} then \megakey{T}.\\
\hline
\megakey{ESC} + \megakey{I} &
Inserts an empty line in the current cursor position and moves all subsequent lines down one position.\\
\hline
\megakey{ESC} + \megakey{D} &
Deletes the current line and moves other lines up one position.\\
\hline
\megakey{ESC} + \megakey{E} &
Sets the cursor to non-flashing mode.\\
\hline
\megakey{ESC} + \megakey{F} &
Sets the cursor to regular flashing mode.\\
\hline
\megakey{ESC} + \megakey{G} &
Enables the bell which can be sounded using \megakey{CTRL} and \megakey{G}.\\
\hline
\megakey{ESC} + \megakey{H} &
Disable the bell so that pressing \megakey{CTRL} and \megakey{G} will have no effect.\\
\hline
\megakey{ESC} + \megakey{J} &
Moves the cursor to start of current line.\\
\hline
\megakey{ESC} + \megakey{K} &
Move to end of the last non-white-space character on the current line.\\
\hline
\megakey{ESC} + \megakey{L} &
Enables scrolling when the cursor down key is pressed at the bottom of the screen.\\
\hline
\megakey{ESC} + \megakey{M} &
Disables scrolling. When pressing the cursor down key at the bottom on the screen, the cursor will move to the top of the screen. The cursor is restricted at the top of the screen with the Cursor up key.\\
\hline
\megakey{ESC} + \megakey{O} &
Cancels the quote, reverse, underline and flash modes.\\
\hline
\megakey{ESC} + \megakey{P} &
Erases all characters from the cursor to the start of current line.\\
\hline
\megakey{ESC} + \megakey{Q} &
Erases all characters from the cursor to the end of current line.\\
\hline
\megakey{ESC} + \megakey{S} &
Switches the VIC-IV to colour range 16-31. These colours can be accessed with \megakey{CTRL} and keys \megakey{1} to \megakey{8} or \megasymbolkey and keys \megakey{1} to \megakey{8}.\\
\hline
\megakey{ESC} + \megakey{U} &
Switches the VIC-IV to colour range 0-15. These colours can be accessed with \megakey{CTRL} and keys \megakey{1} to \megakey{8} or \megasymbolkey and keys \megakey{1} to \megakey{8}.\\
\hline
\megakey{ESC} + \megakey{V} &
Scrolls the entire screen up one line.\\
\hline
\megakey{ESC} + \megakey{W} &
Scrolls the entire screen down one line.\\
\hline
\megakey{ESC} + \megakey{Y} &
Set the default tab stops (every 8 spaces) for the entire screen.\\
\hline
\megakey{ESC} + \megakey{Z} &
Clears all the tab stops. Any tabbing with \megakey{CTRL} and \megakey{I} will move the cursor to the end of the line.\\
\hline
\megakey{ESC} + $\uparrow$ &
Saves the current cursor position. Use \megakey{ESC} + $\leftarrow$ key (next to 1 key) to move back to saved position.\\
\hline
\megakey{ESC} + $\leftarrow$ &
Restores the cursor position to the position stored via a prior press of the \megakey{ESC} + $\uparrow$ key (next to RESTORE key).\\
\hline
\end{longtable}
\end{center}
