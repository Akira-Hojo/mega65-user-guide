\newenvironment{hyppotrap}[3]
{
  \newcommand{\availablefrom}[1]{Available since Hyppo ##1}
  \newcommand{\deprecatedfrom}[1]{
    \textcolor{Orange}{
      \textbf{DEPRECATED} since Hyppo ##1.
      This service may be removed in a future version. New programs should
      not use this service.
    }
  }
  \newcommand{\errordesc}[3]{
    \index{Hyppo Error Codes!\$##1}
    \textbf{\$##1 ##2} ##3\par
  }
  \newcommand{\hypporef}[1]{
    \StrDel{hyppo:##1}{\_}[\reflbl]
    \nameref{\reflbl}
  }
  \newcommand{\notimplemented}{\item[Remarks:]\textbf{NOT IMPLEMENTED}\par}
  \newcommand{\register}[2]{\textbf{##1} ##2\par}
  \newcommand{\TODO}{\textbf{\color{red}TODO}}
  \titleformat*{\subsection}{\normalfont\huge\bfseries\color{blue}}
  \subsection{hyppo\_#1}
  { \StrDel{hyppo:#1}{\_}[\reflbl] \label{\reflbl} }
  \index{Hyppo Services!\$#2 \$#3}
  \begin{description}[leftmargin=2.7cm,style=nextline]
  \item [Trap:] LDA \#\$#3 : STA \$#2 : CLV
}
{
  \end{description}
}

\DeclareTCBInputListing{\acmelisting}{ m }{
  bottom=0mm,
  breakable,
  colback=black,
  coltext=lightgray,
  enhanced,
  frame empty,
  fontupper=\codefont,
  listing engine=listings,
  listing file={#1},
  listing only,
  listing options={
    basewidth=0.3em,
    breakatwhitespace,
    breakindent=0pt,
    breaklines,
    columns=fullflexible,
    commentstyle=\sffamily\footnotesize\color{Dandelion},
    includerangemarker=false,
    inputpath="examples/appendix-hypervisor-calls",
    keywordstyle={*\color{white}},
    keywordstyle={[2]\color{GreenYellow}},
    language={[acme]Assembler},
    linerange=EXAMPLE\ BEGINS-EXAMPLE\ ENDS,
    postbreak={;\space},
    rangeprefix=;\ >>>\ ,
    showstringspaces=false,
  },
  top=0mm,
  underlay first and middle={
    \node at ([yshift=-3mm]frame.south)
      {\ldots{} continues on the next page \ldots} ;
  }
}


\chapter{MEGA65 Hyppo Services}
\index{Hyppo Services}
\index{Registers!\$D640}
\index{Registers!\$D641}
\index{Registers!\$D642}
\index{Registers!\$D643}
\index{Registers!\$D644}
\index{Registers!\$D645}
\index{Registers!\$D646}
\index{Registers!\$D647}
\index{Registers!\$D648}
\index{Registers!\$D649}
\index{Registers!\$D64A}
\index{Registers!\$D64B}
\index{Registers!\$D64C}
\index{Registers!\$D64D}
\index{Registers!\$D64E}
\index{Registers!\$D64F}
\index{Registers!\$D650}
\index{Registers!\$D651}
\index{Registers!\$D652}
\index{Registers!\$D653}
\index{Registers!\$D654}
\index{Registers!\$D655}
\index{Registers!\$D656}
\index{Registers!\$D657}
\index{Registers!\$D658}
\index{Registers!\$D659}
\index{Registers!\$D65A}
\index{Registers!\$D65B}
\index{Registers!\$D65C}
\index{Registers!\$D65D}
\index{Registers!\$D65E}
\index{Registers!\$D65F}
\index{Registers!\$D660}
\index{Registers!\$D661}
\index{Registers!\$D662}
\index{Registers!\$D663}
\index{Registers!\$D664}
\index{Registers!\$D665}
\index{Registers!\$D666}
\index{Registers!\$D667}
\index{Registers!\$D668}
\index{Registers!\$D669}
\index{Registers!\$D66A}
\index{Registers!\$D66B}
\index{Registers!\$D66C}
\index{Registers!\$D66D}
\index{Registers!\$D66E}
\index{Registers!\$D66F}
\index{Registers!\$D670}
\index{Registers!\$D671}
\index{Registers!\$D672}
\index{Registers!\$D673}
\index{Registers!\$D674}
\index{Registers!\$D675}
\index{Registers!\$D676}
\index{Registers!\$D677}
\index{Registers!\$D678}
\index{Registers!\$D679}
\index{Registers!\$D67A}
\index{Registers!\$D67B}
\index{Registers!\$D67C}
\index{Registers!\$D67D}
\index{Registers!\$D67E}
\index{Registers!\$D67F}

\section{Introduction}
A part of the MEGA65 is the system program called Hyppo that:
\begin{itemize}
  \item Boots the MEGA65.
  \item Loads the ROMs and other files from the SD card.
  \item Makes memory banks 2 and 3 ROM-like by protecting them from being
        written to.
  \item Virtualises the floppy disk controller so you can use disk images.
  \item Launches various utilities like the freezer and the Matrix Mode
        Debugger.
  \item Provides services specific to the MEGA65 that you can use in your
        programs.
\end{itemize}

If you know about hypervisors and virtual machines, Hyppo is a very limited
hypervisor. Don't expect to be able to run multiple virtual machines
concurrently with full isolation. Hyppo runs things that are more akin to the
task and processes of a modern operating system than the virtual machines of a
hypervisor as you might know it.

Hyppo provides 3 operating modes.

\begin{itemize}
  \item \textbf{The C64-like operating mode} runs C64 programs and MEGA65
        programs that run in the MEGA65's C64 mode. When you boot with
        \megasymbolkey pressed or use the \textbf{GO64} command, Hyppo
        starts a process in the C64-like operating mode to run BASIC 2 or the
        MEGA65 program.
  \item \textbf{The C65-like operating mode} is the MEGA65's normal operating
        mode. This is where regular MEGA65 program run, including BASIC 65
        programs.
  \item \textbf{The MEGA65 operating mode} runs the MEGA65's system programs
        like the freezer, the configuration utility and the Matrix Mode
        Debugger. Maybe surprisingly, normal MEGA65 programs do not run in the
        MEGA65 operating mode. They run in the C65-like operating mode. The
        MEGA65 operating mode is designed solely for the MEGA65 and does not
        attempt to be compatible with or even be similar to previous systems.
\end{itemize}

Unlike on the C128, it is possible for a program to effectively change the
operating mode while it's is running, by simply enabling or disabling the
various hardware features.

\filbreak
Hyppo provides very limited virtualisation of the MEGA65's hardware. It can
virtualise the floppy controller. There are plans to virtualise the serial bus
so the MEGA65 can use disk images for units like the 1541.

There are some parts of the hardware that only Hyppo can access. It is the only
component that can directly access the internal and external SD cards. You need
to use Hyppo's services if you want to access the files and directories on the
SD cards from within your programs.

\subsection{Terminology}

When you start to learn about Hyppo, there can be some terminology that might be
confusing if you already know about other parts of the MEGA65.

On the SD card there is likely to be a file called HICKUP.M65. This file updates
Hyppo to new versions without having to install an upgraded core. You might find
occasions where Hyppo might be called Hickup because of this strong association.

There are 3 distinct disk operating systems in the MEGA65.

\begin{itemize}
  \item Inside Hyppo is Hyppo DOS, or HDOS for short. HDOS is for accessing
        the FAT32 file system on the SD cards. HDOS does not know anything
        about Commodore file systems. It can attach an image of a Commodore
        file system, but it does not understand what is inside the image.
  \item Inside the Kernal is CBDOS. CBDOS is for accessing 1581-like file
        systems. CBDOS uses the 45IO27 multi-function I/O controller to access
        the sectors of a physical disk. CBDOS does not know anything about SD
        cards and the FAT32 file system on them. Hyppo virtualises part of the
        45IO27 so CBDOS can access disk images like they're physical disks.
  \item The external disk units attached to the serial bus each have their own
        DOS. They are used for accessing the file systems on their respective
        physical disks.
\end{itemize}

The word drive means different things for each of these DOS's.

\begin{itemize}
  \item The drives in Hyppo are the partitions of the internal and external SD
        cards. When the MEGA65 boots, Hyppo assigns numbers to the partitions
        it can read.
  \item The drives in CBDOS are the physical disk drives attached to the 45IO27
        multi-function I/O controller --- such as the internal disk drive ---
        or the disk images attached to the virtualised 45IO27. The CBDOS drives
        are normally seen as units 8 and 9.
  \item The drives in an external unit attached to the serial bus are the
        disk drives inside that unit.
\end{itemize}

\filbreak
\subsection{Versions}

This chapter describes the services available in Hyppo 1.2.

New Hyppo services may become available and existing Hyppo services may change
or be deprecated. A robust program will use the \nameref{hyppo:getversion}
service to check whether it is compatible with the Hyppo in the MEGA65 it's
running on.

% Jimbo - Do we commit to semantic versioning? Do we have a deprecation policy?

\subsection{Using}
When you want to use a Hyppo service, you don't use JSR. This is because
Hyppo exists in a space that's separate from regular code. In order to
access it, the CPU needs to switch into its hypervisor mode.

At addresses \$D640 -- \$D67F are a set of hypervisor traps. Writing to these
addresses are not like writing to other addresses. Instead of writing to memory
or I/O, the CPU switches into the hypervisor mode and starts a Hyppo service.
How the CPU does this is described in \bookvref{cha:cpu}.

Which Hyppo service starts depends on what value from the A register you
write and which trap you write to. Each of the services described in this
chapter tells you what value to write and which trap to use. You have to use the
A register when triggering a trap. Writing the same value from another register
won't work.

When the Hyppo service finishes, the CPU will switch back to your program.
Except for the registers a service uses to return values, the registers are
otherwise preserved.

\textbf{Important} The CPU may or may not execute the next byte in your
program after the Hyppo service finishes. Put a CLV instruction after the STA.
The CPU executing the CLV or not shouldn't matter to your program. If your
program does rely on the V flag, you can use the NOP instruction instead. When
you use NOP you must be mindful of when the CPU interprets the NOP as a prefix
for the following instruction. For this reason, you should prefer using CLV
over NOP.

\subsection{Errors}
\index{Hyppo Error Codes}

If the service was successful, it will set the C flag.

If the service was unsuccessful, it will clear the C flag and put an error code
in the A register. There is a table of error codes in the description for
\nameref{hyppo:geterrorcode}.



% ==============================================================================
% General Services
% ==============================================================================
\newpage
\section{General Services}


% ******************************************************************************
% geterrorcode
% ******************************************************************************
\begin{hyppotrap}{geterrorcode}{D640}{38}
\index{Hyppo Error Codes}
\item [Service:]
  Returns the current error code from Hyppo.
\item [Precondition:]
  The previous service used cleared the C flag.
\item [Outputs:]
  \register{A}{The error code of the previously failed service.}
\item [History:]
  \availablefrom{1.2}
\item [Remarks:]
  The error code is only valid if the previous Hyppo service cleared the C flag.
  If the C flag was set there was no error and the Hyppo error code is
  undefined.

  The meanings here are generic. See the sections for the services for more
  specific meanings.
\item [Error codes:] This is possibly not an exhaustive list.
{
  \setlength{\def\arraystretch{1.5}\tabcolsep}{3pt}
  \begin{longtable}{|c|r|l|p{8cm}|}
    \hline
    \textbf{Hex} & \textbf{Dec} & \textbf{Name} & \textbf{General meaning}\\
    \hline
    \endhead
    \index{Hyppo Error Codes!\$01}
    \$01 & 1 & \makecell[tl]{partition not \\ interesting} &
    The partition is not of a supported type.
    \\\hline
    \index{Hyppo Error Codes!\$02}
    \$02 & 2 & bad signature &
    The signature bytes at the end of a partition table or of the first sector
    of a partition were missing or incorrect.
    \\\hline
    \index{Hyppo Error Codes!\$03}
    \$03 & 3 & is small FAT &
    This is partition is FAT12 or FAT16 partition. Only FAT32 is supported.
    \\\hline
    \index{Hyppo Error Codes!\$04}
    \$04 & 4 & \makecell[tl]{too many reserved\\clusters} &
    The partition has more than 65,535 reserved sectors.
    \\\hline
    \index{Hyppo Error Codes!\$05}
    \$05 & 5 & not two FATs &
    The partition does not have exactly two copies of the FAT structure.
    \\\hline
    \index{Hyppo Error Codes!\$06}
    \$06 & 6 & too few clusters &
    The partition contains too few clusters.
    % Jimbo - What is the minimum?
    \\\hline
    \index{Hyppo Error Codes!\$07}
    \$07 & 7 & read timeout &
    It took to long to read from the SD card.
    % Jimbo - Is there a write timeout error code?
    \\\hline
    \index{Hyppo Error Codes!\$08}
    \$08 & 8 & partition error &
    An unspecified error occurred while handling a partition.
    \\\hline
    \index{Hyppo Error Codes!\$10}
    \$10 & 16 & invalid address &
    An invalid address was supplied in an argument.
    \\\hline
    \index{Hyppo Error Codes!\$11}
    \$11 & 17 & illegal value &
    An illegal value was supplied in an argument.
    \\\hline
    \index{Hyppo Error Codes!\$20}
    \$20 & 32 & read error &
    An unspecified error occurred while reading.
    \\\hline
    \index{Hyppo Error Codes!\$21}
    \$21 & 33 & write error &
    An unspecified error occurred while writing.
    \\\hline
    \index{Hyppo Error Codes!\$80}
    \$80 & 128 & no such drive &
    The supplied Hyppo drive number does not exist.
    \\\hline
    \index{Hyppo Error Codes!\$81}
    \$81 & 129 & {name too long} &
    The supplied filename was too long.
    \\\hline
    \index{Hyppo Error Codes!\$82}
    \$82 & 130 & not implemented &
    The Hyppo service is not implemented.
    \\\hline
    \index{Hyppo Error Codes!\$83}
    \$83 & 131 & file too long &
    The file is larger than 16MB.
    \\\hline
    \index{Hyppo Error Codes!\$84}
    \$84 & 132 & \makecell[tl]{too many\\open files} &
    All of the file descriptors are in use.
    \\\hline
    \index{Hyppo Error Codes!\$85}
    \$85 & 133 & invalid cluster &
    The supplied cluster number is invalid.
    \\\hline
    \index{Hyppo Error Codes!\$86}
    \$86 & 134 & is a directory &
    An attempt was made to operate on a directory, where a normal file was
    expected.
    \\\hline
    \index{Hyppo Error Codes!\$87}
    \$87 & 135 & not a directory &
    An attempt was made to operate on a normal file, where a directory was
    expected.
    \\\hline
    \index{Hyppo Error Codes!\$88}
    \$88 & 136 & file not found &
    The file could not be located in the current directory of the current drive.
    \\\hline
    \index{Hyppo Error Codes!\$89}
    \$89 & 137 & \makecell[tl]{invalid file\\descriptor} &
    An invalid or closed file descriptor was supplied.
    \\\hline
    \index{Hyppo Error Codes!\$8A}
    \$8A & 138 & \makecell[tl]{image wrong\\length} &
    The disk image file has the wrong length.
    \\\hline
    \index{Hyppo Error Codes!\$8B}
    \$8B & 139 & image fragmented &
    The disk image is not stored contiguously on the SD card.
    \\\hline
    \index{Hyppo Error Codes!\$8C}
    \$8C & 140 & no space &
    The SD card has no free space for the requested operation.
    \\\hline
    \index{Hyppo Error Codes!\$8D}
    \$8D & 141 & file exists &
    A file already exists with the given name.
    \\\hline
    \index{Hyppo Error Codes!\$8E}
    \$8E & 142 & directory full &
    The directory cannot accommodate any more entries.
    \\\hline
    \index{Hyppo Error Codes!\$FF}
    \$FF & 255 & eof &
    The end of a file or directory was encountered.
    \\\hline
    \index{Hyppo Error Codes!\$FF}
    \$FF & 255 & no such trap &
    There is no Hyppo service available for the trap. The program may be
    incompatible with this version of Hyppo.
    \\\hline
  \end{longtable}
}
\end{hyppotrap}


% ******************************************************************************
% getversion
% ******************************************************************************
\newpage
\begin{hyppotrap}{getversion}{D640}{00}
\item [Service:]
  Returns the version of Hyppo and HDOS.
\item [Outputs:]
  \register{A}{The major version number of Hyppo}
  \register{X}{The major version number of Hyppo}
  \register{Y}{The minor version number of HDOS}
  \register{Z}{The major version number of HDOS}
\item [History:]
  \availablefrom{1.2}
\item [Remarks:]
  The HDOS in Hyppo is not related to the CBDOS inside the Kernal or the
  DOS in the disk drive units attached to the serial port.
\item [Example:]
  Tests if Hyppo's version is $\geq$ 1.2 and $<$ 2.0.
  \acmelisting{getversion.asm}
\end{hyppotrap}


% ******************************************************************************
% setup_transfer_area
%
% Jimbo - What are the post-conditions for this? What services are affected by
%         this?
%       - Is this now redudant? So far everything I've used has the transfer
%         area as a parameter.
% ******************************************************************************
\newpage
\begin{hyppotrap}{setup\_transfer\_area}{D640}{3A}
\item [Service:]
  Sets up the area Hyppo uses to transfer data to and from your program.
\item [Inputs:]
  \register{Y}{The MSB of the transfer area's address}
\item [Errors:]
  \errordesc{10}{invalid address}{The transfer area address in Y $>$ \$7E}
\item [History:]
  \availablefrom{1.2}
\item [Remarks:]
  The transfer area must be between \$0000 and \$7E00. It must also begin on a
  page boundary. The LSB of its address must be \$00.

  The transfer area is 256 bytes long for most services.

  The transfer area is indicated using the CPU's current memory mapping at
  the time that a service is used. However, it is good practice to always
  place it in the bottom 32KB of bank 0.
\item [Example:]
  Reserves 256 bytes on a page boundary and sets it up as the transfer area.
  \acmelisting{setup_transfer_area.asm}
\end{hyppotrap}



% ==============================================================================
% Disk/storage services
% ==============================================================================
\newpage
\section{Drive/Storage Services}

% Jimbo - These should be named as what they are, partitions.
%         This would resolve potential confusion when we do start referring
%         to the virtual F011's drive 0 and 1 around mounting disk images.

In Hyppo, drives are the partitions of the internal and external SD cards.
They are not the drive 0 and drive 1 of the F011 floppy controller. They
are also not the drive 0 and drive 1 of dual-drive units attached to the serial
bus.


% ******************************************************************************
% chdir
% ******************************************************************************
\begin{hyppotrap}{chdir}{D640}{0C}
\item [Service:]
  Changes the current working directory.
\item [Preconditions:]
  The FAT dir entry for the directory you want to change to has been
  found. \hypporef{findfile} is typically used to find a FAT dir entry.
  \hypporef{findfirst}, \hypporef{findnext} and \hypporef{readdir} can also be
  used.
\item [Errors:]
  \errordesc{87}{not a directory}{The FAT dir entry last found isn't for a
  directory. Bit 4 of the FAT dir entry's attribute byte is set for
  directories.}
\item [History:]
  \availablefrom{1.2}
\item [Remarks:]
  You can move up to the parent directory by finding the .. FAT dir entry.

  You cannot move up or down more than one directory at a time.

  Use \hypporef{cdrootdir} to directly change back to the root directory.
\item [Example:]
  Changes to an arbitrary path on the current drive. Call with Y:X being the
  address of the path. The last character of each component in the path needs
  to have bit 7 set. The whole path is terminated with a \$00. For example,
  to change into DIR1 and then DIR2 the path would be
  {\codefont !text "DIR", '1'+\$80, "DIR" '2'+\$80, 0}.

  If successful, returns with the C flag set. If some part of the path doesn't
  exist, returns with the C flag cleared and the current working directory will
  be whatever directory was last successfully navigated to.
  \acmelisting{chdir.asm}
\end{hyppotrap}


% ******************************************************************************
% closeall
% ******************************************************************************
\newpage
\begin{hyppotrap}{closeall}{D640}{22}
\item [Service:]
  Closes all the file descriptors.
\item [Postconditions:]
  Using any file descriptor with \hypporef{closedir} or \hypporef{closefile}
  succeeds.

  Using any file descriptor with \hypporef{readdir} or \hypporef{readfile}
  fails.

  \hypporef{opendir} and \hypporef{openfile} reuse the file descriptor.
\item [History:]
  \availablefrom{1.2}
\item [Remarks:]
  You can also close individual file descriptors using \hypporef{closedir} or
  \hypporef{closefile}.
\end{hyppotrap}


% ******************************************************************************
% closedir
% ******************************************************************************
\newpage
\begin{hyppotrap}{closedir}{D640}{16}
\item [Service:]
  Closes a file descriptor for a directory.
\item [Preconditions:]
  The file descriptor given in the X register was opened using
  \hypporef{opendir}.
\item [Inputs:]
  \register{X}{The file descriptor for the directory}
\item [Postconditions:]
  Using the file descriptor again with \hypporef{closedir} succeeds.

  Using the file descriptor again with \hypporef{readdir} fails.

  \hypporef{opendir} and \hypporef{openfile} reuse the file descriptor.
\item [History:]
  \availablefrom{1.2}
\item [Remarks:]
  You can also close all the open file descriptors using \hypporef{closeall}.
\item [Example:]
  See the examples in \hypporef{opendir} and \hypporef{readdir}.
\end{hyppotrap}


% ******************************************************************************
% closefile
% ******************************************************************************
\newpage
\begin{hyppotrap}{closefile}{D640}{20}
\item [Service:]
  Closes a file descriptor for a file.
\item [Preconditions:]
  The file descriptor given in the X register was opened using
  \hypporef{openfile}.
\item [Inputs:]
  \register{X}{The file descriptor for the file}
\item [Postconditions:]
  Using the file descriptor again with \hypporef{closefile} succeeds.

  Using the file descriptor again with \hypporef{readfile} fails.

  \hypporef{opendir} and \hypporef{openfile} reuse the file descriptor.
\item [History:]
  \availablefrom{1.2}
\item [Remarks:]
  You can also close all the open file descriptors using \hypporef{closeall}.
\item [Example:]
  See the example in \hypporef{openfile}.
\end{hyppotrap}


% ******************************************************************************
% filedate
% ******************************************************************************
\newpage
\begin{hyppotrap}{filedate}{D640}{2C}
\item [Service:]
  Sets time stamp of a file.
\notimplemented
\end{hyppotrap}


% ******************************************************************************
% findfile
%
% Jimbo - Does this service work for files that don't have an LFN?
% ******************************************************************************
\newpage
\begin{hyppotrap}{findfile}{D640}{34}
\item [Service:]
  Finds the first file whose filename matches the current Hyppo filename.
\item [Preconditions:]
  The current Hyppo filename has been set using \hypporef{setname}.
\item [Postconditions:]
  No additional file descriptors are open.
\item [Errors:]
  \errordesc{88}{file not found}{A matching file was not found in the current
  directory of the current drive.}
\item [History:]
  \availablefrom{1.2}
\item [Remarks:]
  % Jimbo - Why does Hyppo do this and create this restriction?
  Hyppo will only find files whose long filename is all in uppercase.
  Hyppo converts the current filename to ASCII uppercase before trying
  to match it. Bytes \$61 -- \$7B change to \$41 -- \$5A.

  Hyppo does not yet support the wildcard characters * and ?. Support
  for that is planned in a future version.

  This only finds the first matching file. You can find multiple matches by
  using \hypporef{findfirst} and \hypporef{findnext}.
\item [Example:]
  See the example in \hypporef{openfile}.
\end{hyppotrap}


% ******************************************************************************
% findfirst
%
% Jimbo - Does this service work for files that don't have an LFN?
% ******************************************************************************
\newpage
\begin{hyppotrap}{findfirst}{D640}{30}
\item [Service:]
  Finds the first file whose filename matches the current Hyppo filename.
\item [Preconditions:]
  The current Hyppo filename has been set using \hypporef{setname}.
\item [Outputs:]
  \register{A}{The file descriptor for reading the current working directory.
  You might be responsible for closing this file descriptor using
  \hypporef{closedir}. See the remarks.}
\item [Postconditions:]
  \hypporef{findnext} find the next matching file or fails with a file not
  found error.
\item [Side effects:]
  Sets the current file descriptor.
\item [Errors:]
  \errordesc{88}{file not found}{A matching file was not found in the current
  directory of the current drive.}
\item [History:]
  \availablefrom{1.2}
\item [Remarks:]
  If Hyppo finds an initial matching file, it will set the C flag and
  return a file descriptor in the A register. This is a file descriptor for
  reading the current working directory. You are responsible for closing it
  using \hypporef{closedir}. It's a standard directory file descriptor. You
  can use \hypporef{readdir} to read the FAT dir entries after the file that
  was found.

  If Hyppo doesn't find any matching files, it will fail with a file
  not found error. In this case Hyppo will have already closed the
  file descriptor and you don't have to close it.

  % Jimbo - Why does Hyppo do this and create this restriction?
  Hyppo will only find files whose long filename is all in uppercase.
  Hyppo converts the current filename to ASCII uppercase before trying
  to match it. Bytes \$61 -- \$7B change to \$41 -- \$5A.

  Hyppo does not yet support the wildcard characters * and ?. Support
  for that is planned in a future version.

  If you are only interested in the first match, you can use \hypporef{findfile}
  instead. \hypporef{findfile} always closes the file descriptor for you. But
  you can't use it to find multiple matching files.
\item [Example:]
  See the example in \hypporef{findnext}.
\end{hyppotrap}


% ******************************************************************************
% findnext
%
% Jimbo - Does this service work for files that don't have an LFN?
% ******************************************************************************
\newpage
\begin{hyppotrap}{findnext}{D640}{32}
\item [Service:]
  Finds a subsequent file whose filename matches the current Hyppo filename.
\item [Preconditions:]
  The current Hyppo filename has been set using \hypporef{setname}.

  The first matching file has already been found successfully using
  \hypporef{findfirst}.
\item [Postconditions:]
  Using \hypporef{findnext} again finds the next matching file or fails with a
  file not found error.
\item [Errors:]
  \errordesc{88}{file not found}{A subsequent matching file was not found in
  the current directory of the current drive.}
\item [History:]
  \availablefrom{1.2}
\item [Remarks:]
  If Hyppo doesn't find a subsequent matching file, it will fail with a
  file not found error. Hyppo will also close the file descriptor it
  output in \hypporef{findfirst}.

  If you don't exhaust the search by using \hypporef{findnext} until it fails
  with a file not found error, you are required to close the file descriptor
  yourself using \hypporef{closedir}.
\item [Example:]
  Returns with X register containing the number of files matching the current
  Hyppo filename in the current working directory of the current drive. While
  a directory can in theory have multiple files with an indentical name, this
  example will be more useful once Hyppo supports * and ? wildcards.
  \acmelisting{findnext.asm}
\end{hyppotrap}


% ******************************************************************************
% fstat
% ******************************************************************************
\newpage
\begin{hyppotrap}{fstat}{D640}{28}
\item [Service:]
  Returns information about a file.
\notimplemented
\end{hyppotrap}


% ******************************************************************************
% getcurrentdrive
% ******************************************************************************
\newpage
\begin{hyppotrap}{getcurrentdrive}{D640}{04}
\item [Service:]
  Returns the number of the currently selected drive (SD card partition).
\item [Outputs:]
  \register{A}{The current drive number}
\item [History:]
  \availablefrom{1.2}
\item [Remarks:]
  \hypporef{selectdrive} changes the current drive number. \hypporef{cdrootdir}
  can also change it.
\item [Example:]
  Prints the number of the currently selected drive in the top-left of the
  screen. This example assumes that there aren't more than 10 drives (drives
  0 to 9). It also assumes the screen memory hasn't been moved from \$000800.
  \acmelisting{getcurrentdrive.asm}
\end{hyppotrap}


% ******************************************************************************
% getcwd
% ******************************************************************************
\newpage
\begin{hyppotrap}{getcwd}{D640}{0A}
\item [Service:]
  Returns information on the currently selected directory or sub-directory.
\notimplemented
\end{hyppotrap}


% ******************************************************************************
% getdefaultdrive
% ******************************************************************************
\newpage
\begin{hyppotrap}{getdefaultdrive}{D640}{02}
\item [Service:]
  Returns the drive number (SD card partition) Hyppo selected while
  booting.
\item [Outputs:]
  \register{A}{The default drive number}
\item [History:]
  \availablefrom{1.2}
\item [Example:]
  Selects the default drive.
  \acmelisting{getdefaultdrive.asm}
\end{hyppotrap}


% ******************************************************************************
% getdrivesize
% ******************************************************************************
\newpage
\begin{hyppotrap}{getdrivesize}{D640}{08}
\item [Service:]
  Returns information on the size of the currently selected drive (SD card
  partition).
\notimplemented
\end{hyppotrap}


% ******************************************************************************
% loadfile
% ******************************************************************************
\newpage
\begin{hyppotrap}{loadfile}{D640}{36}
\item [Service:]
  Loads a file into chip memory.
\item [Preconditions:]
  The name of the file to load has been set using \hypporef{setname}.
\item [Inputs:]
  \register{X}{The LSB of the address to start loading from}
  \register{Y}{The middle byte of the address to start loading from}
  \register{Z}{The MSB of the address to start loading from}
\item [Postconditions:]
  No additional file descriptors are open.
\item [Errors:]
  \errordesc{84}{too many open files}{\hypporef{loadfile} uses one file
  descriptor internally, but all the file descriptors are in use. Close some or
  all of the file descriptors using \hypporef{closedir}, \hypporef{closefile}
  or \hypporef{closeall}.}
  \errordesc{88}{file not found}{The file was not found in the current directory
  of the current drive.}
\item [History:]
  \availablefrom{1.2}
\item [Remarks:]
  This service can load files up to 16MB in size into the first 16MB of chip
  memory. Chip memory is the 384KB or more of memory inside the CPU module.

  Loading will start at 28-bit address \$00ZZYYXX. If loading tries to go beyond
  \$00FFFFFF, it wraps around and continue at \$00000000.

  You can use \hypporef{loadfile\_attic} to load a file into hyper memory. The
  hyper memory is the 8MB or more of memory in the external RAM chips.
\item [Example:]
  Loads the file SHOOTEMUP-001.LEVEL into memory starting at \$48000.
  \acmelisting{loadfile.asm}
\end{hyppotrap}


% ******************************************************************************
% loadfile_attic
% ******************************************************************************
\newpage
\begin{hyppotrap}{loadfile\_attic}{D640}{3E}
\item [Service:]
  Loads a file into hyper memory.
\item [Preconditions:]
  The name of the file to load has been set using \hypporef{setname}.
\item [Inputs:]
  \register{X}{The LSB of the address to start loading from}
  \register{Y}{The middle byte of the address to start loading from}
  \register{Z}{The MSB of the address to start loading from}
\item [Postconditions:]
  No additional file descriptors are open.
\item [Errors:]
  \errordesc{84}{too many open files}{hos\_loadfile\_attic uses one file
  descriptor internally, but all the file descriptors are in use. Close some or
  all of the file descriptors using \hypporef{closedir}, \hypporef{closefile}
  or \hypporef{closeall}.}
  \errordesc{88}{file not found}{The file was not found in the current directory
  of the current drive.}
\item [History:]
  \availablefrom{1.2}
\item [Remarks:]
  This service can load files up to 16MB in size into the first 16MB of hyper
  memory. Hyper memory is the 8MB or more of memory in the external RAM chips.

  Loading will start at 28-bit address \$08ZZYYXX. If loading tries to go beyond
  \$08FFFFFF, the loading will wrap around and continue at \$08000000.

  You can use \hypporef{loadfile} to load a file into chip memory. The chip
  memory is the 384KB or more of memory inside the CPU module.
\end{hyppotrap}


% ******************************************************************************
% mkdir
% ******************************************************************************
\newpage
\begin{hyppotrap}{mkdir}{D640}{0E}
\item [Service:]
  Creates a sub-directory.
\item [Errors:]
  \errordesc{8D}{file exists}{A sub-directory or file already exists with the
  current Hyppo filename in the current working directory of the current
  drive.}
\notimplemented
\end{hyppotrap}


% ******************************************************************************
% mkfile
% ******************************************************************************
\begin{hyppotrap}{mkfile}{D640}{1E}
\item [Service:]
  Creates a file.
\item [Errors:]
  \errordesc{8D}{file exists}{A sub-directory or file already exists with the
  current Hyppo filename in the current working directory of the current
  drive.}
\notimplemented
\end{hyppotrap}


% ******************************************************************************
% opendir
% ******************************************************************************
\newpage
\begin{hyppotrap}{opendir}{D640}{12}
\item [Service:]
  Opens the current working directory for reading the file entries in it.
\item [Preconditions:]
  The drive and directory you want to read have already been set up using
  \hypporef{selectdrive} and \hypporef{chdir} if necessary.
\item [Outputs:]
  \register{A}{The file descriptor for reading the directory. You are
  responsible for closing this file descriptor using \hypporef{closedir}.}
\item [Postconditions:]
  \hypporef{readdir} reads the first FAT dir entry in the directory.
\item [Errors:]
  \errordesc{84}{too many open files}{All the file descriptors are in use.
  \hypporef{opendir} and \hypporef{openfile} share the same very small pool of
  file descriptors. Close some or all of the file descriptors using
  \hypporef{closedir}, \hypporef{closefile} or \hypporef{closeall}.}
  \errordesc{87}{not a directory}{The FAT dir entry last found is for a file.
  Use \hypporef{openfile} for files.}
\item [History:]
  \availablefrom{1.2}
\item [Example:]
  Calls processdirentry for each FAT dir entry in the current working directory.
  processdirentry is assumed to be defined elsewhere.

  \TODO
\end{hyppotrap}


% ******************************************************************************
% openfile
% ******************************************************************************
\newpage
\begin{hyppotrap}{openfile}{D640}{18}
\item [Service:]
  Opens a file on a drive.
\item [Preconditions:]
  The file has already been found. Files can be found using \hypporef{findfile},
  \hypporef{findfirst} and \hypporef{findnext}. \hypporef{readdir} can also be
  used to find a file.
\item [Outputs:]
  \register{A}{The file descriptor for accessing the file. You are responsible
  for closing this file descriptor using \hypporef{closefile}.}
\item [Postconditions:]
  Using \hypporef{readfile} with this file descriptor reads the first sector of
  the file.
\item [Side effects:]
  Sets the current file to the newly opened file. \hypporef{readfile} reads
  from the current file.
\item [Errors:]
  \errordesc{84}{too many open files}{All the file descriptors are in use.
  \hypporef{opendir} and \hypporef{openfile} share the same very small pool of
  file descriptors. Close some or all of the file descriptors using
  \hypporef{closedir}, \hypporef{closefile} or \hypporef{closeall}.}
  \errordesc{86}{is a directory}{The FAT dir entry last found is for a
  directory. Use \hypporef{opendir} for directories.}
\item [History:]
  \availablefrom{1.2}
\item [Remarks:]
  You cannot use this to open a file inside a disk image. To do that you use
  \hypporef{d81attach0} or \hypporef{d81attach1} to attach the disk image and
  then use either use the Kernal to read the file or program the virtualised
  F011 floppy controller.
\item [Example:]
  Finds, opens and then closes the file CONFIG.DAT.
  \acmelisting{openfile.asm}
\end{hyppotrap}


% ******************************************************************************
% readdir
% ******************************************************************************
\newpage
\begin{hyppotrap}{readdir}{D640}{14}
\item [Service:]
  Reads the next FAT dir entry into a destination area.
\item [Preconditions:]
  The file descriptor given in the X register was opened using
  \hypporef{opendir} and \hypporef{closedir} hasn't since been used to close it.

  The destination area is on a page boundary between \$0000 and \$7E00 and is
  at least 87 bytes.
\item [Inputs:]
  \register{X}{The file descriptor for the directory.}
  \register{Y}{The MSB of the destination area.}
\item [Outputs:]
  Starting at \$YY00, the FAT dir entry has this structure.
  {\setlength{\tabcolsep}{2mm}
  \begin{tabular}{|c|c|p{6.9cm}|}
  \hline
  \textbf{Offset} & \textbf{Type} & \textbf{Description}
  \\\hline
  \$00 & asciiz & The long file name
  \\
  \$40 & byte & The length of long file name
  \\
  \$41 & ascii & The "8.3" file name. The name part is padded with spaces to make
               it exactly 8 bytes. The 3 bytes of the extension follow. There is
               no . between the name and the extension. There is no NULL byte.
  \\
  \$4E & dword & The cluster number where the file begins. For sub-directories,
               this is where the FAT dir entries start for that sub-directory.
  \\
  \$52 & dword & The length of file in bytes.
  \\
  \$56 & byte  & The type and attribute bits.
  \\\hline
  \end{tabular}
  }

  This is what the bits in the last byte mean. Bits 6 and 7 are undefined.
  {\setlength{\tabcolsep}{2mm}
  \begin{tabular}{|c|l|}
  \hline
  \textbf{Bit} & \textbf{Meaning if bit is set} \\
  \hline
  0 & Read only         \\
  1 & Hidden            \\
  2 & System            \\
  3 & Volume label      \\
  4 & Sub-directory     \\
  5 & Archive           \\
  \hline
  \end{tabular}
  }
\item [Postconditions:]
  Using \hypporef{readdir} again reads the next FAT dir entry in the directory.
\item [Errors:]
  \errordesc{08}{partition error}{An unspecified error occurred while handling
  the currently selected partition.}
  \errordesc{10}{invalid address}{The Y register is $>$ \$7E.}
  \errordesc{85}{invalid cluster}{An attempt was made to read past the end of
  the directory.}
\item [Remarks:]
  If the long file name in the FAT dir entry is too long to copy into the
  destination area, Hyppo skips the entry entirely.

  The file names in FAT are encoded as UTF-16. Hyppo only reads the LSB
  of each 16-bit character. Hyppo does not convert file names into
  PETSCII.

  See \hypporef{setup\_transfer\_area} for more details about the value for the
  Y register.
\item [History:]
  \availablefrom{1.2}
\item [Example:]
  \TODO
\end{hyppotrap}


% ******************************************************************************
% readfile
% ******************************************************************************
\newpage
\begin{hyppotrap}{readfile}{D640}{1A}
\item [Service:]
  Reads the next sector of the current file into the sector buffer.
\item [Preconditions:]
  There is a current file open. Files can be opened with \hypporef{openfile}.
\item [Outputs:]
  \register{X}{The LSB of the number of bytes read}
  \register{Y}{The MSB of the number of bytes read}
\item [Postconditions:]
  The next call to \hypporef{readfile} will read the next sector of the current
  file or signal the end of the file.
\item [Errors:]
  \TODO
\item [History:]
  \availablefrom{1.2}
\item [Remarks:]
  To access the data, you need to either:
  \begin{itemize}
    \item map the sector buffer into the 16-bit address space;
    \item use an enhanced DMA transfer to copy the sector buffer (where? \TODO)
          into a buffer already mapping into the 16-bit address space; or
    \item use 32-bit load instructions to access the sector buffer directly.
  \end{itemize}

  If a full sector was read, Y:X will be \$0200. For the last sector of the
  file, Y:X may be less than that. Any bytes in the sector buffer after Y:X are
  undefined and will not necessarily be zero.

  If you read past the end of the last sector, Y:X will be \$0000, the A
  register will be \$FF and the C flag will be set.

  While multiple files can be opened simultaneously, only the current file can
  be read. The current file is often the last file opened, but not always.
\item [Example:]
  Maps the sector buffer to \$DE00 and then reads each sector of the file
  calling proccesssector for each sector read. proccesssector is assumed to be
  defined elsewhere.
  \acmelisting{readfile.asm}
\end{hyppotrap}


% ******************************************************************************
% rename
% ******************************************************************************
\newpage
\begin{hyppotrap}{rename}{D640}{2A}
\item [Service:]
  Renames a file or sub-directory.
\item [Errors:]
  \errordesc{8D}{file exists}{A sub-directory or file already exists with the
  current Hyppo filename in the current working directory of the current
  drive.}
\notimplemented
\end{hyppotrap}


% ******************************************************************************
% rmdir
% ******************************************************************************
\begin{hyppotrap}{rmdir}{D640}{10}
\item [Service:]
  Removes a sub-directory.
\notimplemented
\end{hyppotrap}


% ******************************************************************************
% rmfile
% ******************************************************************************
\begin{hyppotrap}{rmfile}{D640}{26}
\item [Service:]
  Removes a files.
\notimplemented
\end{hyppotrap}


% ******************************************************************************
% seekfile
% ******************************************************************************
\begin{hyppotrap}{seekfile}{D640}{24}
\item [Service:]
  Seeks to a given sector in a file.
\notimplemented
\end{hyppotrap}


% ******************************************************************************
% selectdrive
% ******************************************************************************
\newpage
\begin{hyppotrap}{selectdrive}{D640}{06}
\item [Service:]
  Selects the currently selected drive (SD card partition).
\item [Preconditions:]
  Hyppo has assigned a drive number to the SD card partition.
\item [Inputs:]
  \register{X}{The drive number to become the new current drive}
\item [Postconditions:]
  \hypporef{getcurrentdrive} returns the value that was in the X register.

  Hyppo services operate on the newly selected drive.
\item [Errors:]
  \errordesc{80}{no such drive}{The drive in the X register does not exist.
  Hyppo only assigns drive numbers to the SD card partitions it can
  read.}
\item [History:]
  \availablefrom{1.2}
\item [Example:]
  Tests if drive 2 exists by trying to select it. Returns with the C flag set if
  drive 2 exists.
  \acmelisting{selectdrive.asm}
\end{hyppotrap}


% ******************************************************************************
% setname
% ******************************************************************************
\newpage
\begin{hyppotrap}{setname}{D640}{2E}
\item [Service:]
  Sets the current Hyppo filename.
\item [Preconditions:]
  The filename is stored in ASCII and ends with \$00 byte.

  The filename starts on a page boundary between \$0000 and \$7E00 and is less
  than 63 characters, excluding the \$00 byte.
\item [Inputs:]
  \register{Y}{The MSB of the filename address.}
\item [Postconditions:]
  Hyppo has copied the filename into it's own data area.

  The hyppo\_find* and hyppo\_load* services use this filename.
\item [Side effects:]
  Sets the transfer area to \$YY00.
\item [Errors:]
  \errordesc{10}{invalid address}{The Y register is $>$ \$7E.}
  \errordesc{81}{name too long}{The filename is longer than 63 characters.}
\item [History:]
  \availablefrom{1.2}
\item [Remarks:]
  The filename must be between \$0000 and \$7E00. It must also begin on a
  page boundary. That is, its address must end with \$00. The current memory
  mapping is used. However, it is good practice to place it in the bottom 32KB
  of bank 0.

  The filenames in FAT are encoded in UTF-16. Hyppo only reads the LSB
  of each 16-bit character. Hyppo does not convert between ASCII and
  PETSCII.

  Hyppo accesses the files in the FAT file system on the internal and
  external SD cards. It does not access files on disks in floppy drives or in
  disk images.
\item [Example:]
  See the examples in \hypporef{openfile}, \hypporef{loadfile},
  \hypporef{d81attach0}.
\end{hyppotrap}


% ******************************************************************************
% writefile
% ******************************************************************************
\newpage
\begin{hyppotrap}{writefile}{D640}{1C}
\item [Service:]
  Writes the sector buffer to the current file.
\notimplemented
\end{hyppotrap}



% ==============================================================================
% Disk Image Services
% ==============================================================================
\newpage
\section{Disk Image Services}

The 45IO27 multi-function I/O controller includes a F011-compatible floppy
controller. The internal floppy drive is attached to this as drive 0.

Hyppo can virtualise the F011 floppy controller so that disk images
can be attached instead of floppy drives. Once a disk image is attached,
Hyppo traps the F011's I/O registers and emulates the commands on the disk
image.

You can use BASIC, the Kernal and the F011 I/O registers to operate on a disk
image just as you would a physical disk. The virtualisation does not behave the
same as a floppy drive in all cases. If you intend for your program to work
with both disk images and physical disks, be sure to test it with both.


% ******************************************************************************
% d81attach0
% ******************************************************************************
\begin{hyppotrap}{d81attach0}{D640}{40}
\item [Service:]
  Attach a D81 disk image to virtualised F011 drive 0.
\item [Preconditions:]
  The current Hyppo filename has been set using \hypporef{setname}.
\item [History:]
  \availablefrom{1.2}
\item [Remarks:]
  Unless it's been changed, drive 0 of the virtualised F011 floppy controller is
  unit 8.
\item [Example:]
  Attaches the disk image MYPROGRAM.D81 to virtualised F011 drive 0.

  \TODO
\end{hyppotrap}


% ******************************************************************************
% d81attach1
% ******************************************************************************
\newpage
\begin{hyppotrap}{d81attach1}{D640}{46}
\item [Service:]
  Attach a D81 disk image to virtualised F011 drive 1.
\item [Preconditions:]
  The current Hyppo filename has been set using \hypporef{setname}.
\item [History:]
  \availablefrom{1.2}
\item [Remarks:]
  Unless it's been changed, drive 1 of the virtualised F011 floppy controller is
  unit 9.
\end{hyppotrap}


% ******************************************************************************
% d81detach
% ******************************************************************************
\newpage
\begin{hyppotrap}{d81detach}{D640}{42}
\item [Service:]
  Detaches any disk images from virtualised F011 drives 0 and 1.
\item [History:]
  \availablefrom{1.2}
\end{hyppotrap}


% ******************************************************************************
% d81write_en
%
% Jimbo - hyppo_d81attach* already write-enables the images.
%         See https://github.com/MEGA65/mega65-core/issues/494
% ******************************************************************************
\newpage
\begin{hyppotrap}{d81write\_en}{D640}{44}
\item [Service:]
  Enables writing to any disk images attached to virtualised F011 drives 0
  and 1.
\item [History:]
  \availablefrom{1.2}
\end{hyppotrap}



% ==============================================================================
% Task and Process Services
% ==============================================================================
\newpage
\section{Task and Process Services}


% ******************************************************************************
% create_task_c64
% ******************************************************************************
\begin{hyppotrap}{create\_task\_c64}{D640}{66}
\item [Service:]
  Creates a Hyppo task in the C64-like operating mode.
\notimplemented
\end{hyppotrap}


% ******************************************************************************
% create_task_c65
% ******************************************************************************
\begin{hyppotrap}{create\_task\_c65}{D640}{68}
\item [Service:]
  Creates a Hyppo task in the C65-like operating mode.
\notimplemented
\end{hyppotrap}


% ******************************************************************************
% create_task_native
% ******************************************************************************
\begin{hyppotrap}{create\_task\_native}{D640}{62}
\item [Service:]
  Creates a Hyppo task in the MEGA65 operating mode.
\notimplemented
\end{hyppotrap}


% ******************************************************************************
% exit_and_switch_to_task
% ******************************************************************************
\begin{hyppotrap}{exit\_and\_switch\_to\_task}{D640}{6A}
\item [Service:]
  Exits the current Hyppo task and switches context to another Hyppo task.
\notimplemented
\end{hyppotrap}


% ******************************************************************************
% exit_task
% ******************************************************************************
\begin{hyppotrap}{exit\_task}{D640}{6E}
\item [Service:]
  Exits the current Hyppo task. % Jimbo - And then what?
\notimplemented
\end{hyppotrap}


% ******************************************************************************
% get_mapping
% ******************************************************************************
\newpage
\begin{hyppotrap}{get\_mapping}{D640}{74}
\item [Service:]
  Copies the current 45GS02 memory mapping into a destination area.
\item [Preconditions:]
  The destination area starts on a page boundary between \$0000 and \$7E00 and
  is at least 6 bytes.
\item [Inputs:]
  \register{Y}{The MSB of the destination area.}
\item [Outputs:]
  Starting at \$YY00, the current mapping info has this structure.
  {\setlength{\tabcolsep}{2mm}
  \begin{tabular}{|c|c|p{6.9cm}|}
  \hline
  \textbf{Offset} & \textbf{Type} & \textbf{Description} \\
  \hline
  0 & word & MAPLO \\
  2 & word & MAPHI \\
  4 & byte & The megabyte offset for MAPLO \\
  5 & byte & The megabyte offset for MAPHI \\
  \hline
  \end{tabular}
  }
\item [Errors:]
  \errordesc{10}{invalid address}{The Y register is $>$ \$7E.}
\item [History:]
  \availablefrom{1.2}
\item [Remarks:]
  MAPLO is the mapping for \$0000 - \$7FFF.

  MAPHI is the mapping for \$8000 - \$FFFF.

  See \TODO{} for more information on MEGA65 memory mapping and banking.
\end{hyppotrap}


% ******************************************************************************
% get_proc_desc
% ******************************************************************************
\newpage
\begin{hyppotrap}{get\_proc\_desc}{D640}{48}
\item [Service:]
  Copies the current task block into a destination area.
\item [Preconditions:]
  The destination area starts on a page boundary between \$0000 and \$7E00 and
  is at least 256 bytes.
\item [Inputs:]
  \register{Y}{The MSB of the destination area.}
\item [Outputs:]
  Starting at \$YY00, the current task block has this structure.
  {\setlength{\tabcolsep}{2mm}
  \begin{tabular}{|c|c|p{6.9cm}|}
  \hline
  \textbf{Offset} & \textbf{Type} & \textbf{Description}
  \\\hline
  \$00 & byte & The ID of the current task.
  \\
  \$01 & text & The name of the current task. A maximum of 16 characters.
                Padded with \$00 bytes.
                If it's 16 characters, there are no trailing \$00 bytes.
  \\
  \$11 & byte & Flags for the D81 disk image attached to drive 0 of the
                virtualised F011 floppy controller.
  \\
  \$12 & byte & Same as above but for drive 1.
  \\
  \$13 & byte & The length of the D81 disk image filename attached to drive 0.
  \\
  \$14 & byte & Same as above but for drive 1.
  \\
  \$15 & text & The filename of the D81 disk image attached to drive 0.
                A maximum of 32 characters. Padded with \$20 bytes.
                There is no trailing \$00 byte.
  \\
  \$35 & text & Same as above but for drive 1.
  \\
  \$55 & & The meaning of these bytes are undefined and subject to change.
  \\
  \$80 & & File descriptor 0.
  \\
  \$A0 & & File descriptor 1.
  \\
  \$C0 & & File descriptor 2.
  \\
  \$E0 & & File descriptor 3.
  \\\hline
  \end{tabular}
  }

  Each of the file descriptors has this structure.
  {\setlength{\tabcolsep}{2mm}
  \begin{tabular}{|c|c|p{6.9cm}|}
  \hline
  \textbf{Offset} & \textbf{Type} & \textbf{Description}
  \\\hline
  \$00 & byte  & The number of the SD card partition where the file resides.
                 \$FF means the file descriptor is closed.
  \\\hline
  \$01 & dword & The cluster where the file starts
  \\\hline
  \$05 & dword & The current cluster
  \\\hline
  \$09 & byte  & The current sector within the current cluster
  \\\hline
  \$0A & dword & The length of the file (what unit? \TODO)
  \\\hline
  \$0E & dword & The current position within the file's buffer
  \\\hline
  \$12 & dword & The cluster of the directory in which the file resides
  \\\hline
  \$16 & word  & The index of the file within its directory
  \\\hline
  \$18 & dword & The absolute 32-bit address of the file's buffer
  \\\hline
  \$1C & word  & The number of bytes used in the file's buffer
  \\\hline
  \$1E & word  & The current offset within the file's buffer
  \\\hline
  \end{tabular}
  }
\item [Errors:]
  \errordesc{10}{invalid address}{The Y register is $>$ \$7E.}
\item [History:]
  \availablefrom{1.2}
\end{hyppotrap}


% ******************************************************************************
% gettasklist
% ******************************************************************************
\newpage
\begin{hyppotrap}{gettasklist}{D640}{50}
\item [Service:]
  Gets the list of tasks in Hyppo.
\notimplemented
\end{hyppotrap}


% ******************************************************************************
% load_into_task
% ******************************************************************************
\begin{hyppotrap}{load\_into\_task}{D640}{64}
\item [Service:]
  Loads a file from an SD card partition into the memory of a Hyppo task.
\notimplemented
\end{hyppotrap}


% ******************************************************************************
% readoutoftask
% ******************************************************************************
\begin{hyppotrap}{readoutoftask}{D640}{58}
\item [Service:]
  Reads from the memory of another Hyppo task.
\notimplemented
\end{hyppotrap}


% ******************************************************************************
% receivemessage
% ******************************************************************************
\begin{hyppotrap}{receivemessage}{D640}{54}
\item [Service:]
  Receives messages sent from other Hyppo tasks.
\notimplemented
\end{hyppotrap}


% ******************************************************************************
% reset
% ******************************************************************************
\newpage
\begin{hyppotrap}{reset}{D640}{7E}
\item [Service:]
  Warm boots the MEGA65.
\item [History:]
  \availablefrom{1.2}
\end{hyppotrap}


% ******************************************************************************
% rom_writeenable
% ******************************************************************************
\newpage
\begin{hyppotrap}{rom\_writeenable}{D641}{02}
\item [Service:]
  Changes \$20000 -- \$3FFFF to behave like RAM by disabling the
  write-protection.
\item [History:]
  \availablefrom{1.2}
\item [Remarks:]
  \$20000 -- \$3FFFF normally has the Kernal, BASIC, CBDOS, and font ROMs.

  \hypporef{rom\_writeprotect} enables the write-protection and blocks writes.
  \hypporef{toggle\_rom\_writeprotect} toggles the write-protection.
\end{hyppotrap}


% ******************************************************************************
% rom_writeprotect
% ******************************************************************************
\newpage
\begin{hyppotrap}{rom\_writeprotect}{D641}{00}
\item [Service:]
  Changes \$20000 -- \$3FFFF to behave like ROM by enabling the
  write-protection.
\item [History:]
  \availablefrom{1.2}
\item [Remarks:]
  \$20000 -- \$3FFFF normally has the Kernal, BASIC, CBDOS, and font ROMs.

  \hypporef{rom\_writeenable} disables the write-protection and allows writes.
  \hypporef{toggle\_rom\_writeprotect} toggles the write-protection.
\end{hyppotrap}


% ******************************************************************************
% sendmessage
% ******************************************************************************
\newpage
\begin{hyppotrap}{sendmessage}{D640}{52}
\item [Service:]
  Sends a message to another Hyppo task.
\notimplemented
\end{hyppotrap}


% ******************************************************************************
% serial_monitor_wait_and_write
% ******************************************************************************
\newpage
\begin{hyppotrap}{serial\_monitor\_wait\_and\_write}{D643}{xx}
\item [Service:]
  Waits for the serial monitor or Matrix Mode Debugger to be ready to receive
  and then writes a character to it.
\item [Inputs:]
  \register{A}{The ASCII character to write.}
\item [History:]
  \availablefrom{1.2}
\item [Remarks:]
  The service waits for the serial monitor to be ready to receive. This could
  slow down or hang your program. If you don't want this and you are happy for
  the character to be lost if the serial monitor is not ready to receive,
  use \hypporef{serial\_monitor\_write}.
\end{hyppotrap}


% ******************************************************************************
% serial_monitor_write
% ******************************************************************************
\newpage
\begin{hyppotrap}{serial\_monitor\_write}{D640}{7C}
\item [Service:]
  Writes a character to the serial monitor or the Matrix Mode Debugger.
\item [Preconditions:]
  The serial monitor is ready to receive.
\item [Inputs:]
  \register{Y}{The ASCII character to write.}
\item [History:]
  \availablefrom{1.2}
\item [Remarks:]
  The character will be lost if the serial monitor is not ready to receive,
  If you don't want the character to be lost, use
  \hypporef{serial\_monitor\_wait\_and\_write}.
\end{hyppotrap}


% ******************************************************************************
% set_mapping
% ******************************************************************************
\newpage
\begin{hyppotrap}{set\_mapping}{D640}{76}
\item [Service:]
  Copies the source area into the current 45GS02 memory mapping.
\item [Preconditions:]
  The source area starts on a page boundary between \$0000 and \$7E00 and
  is at least 6 bytes.
\item [Inputs:]
  \register{Y}{The MSB of the source area.}

  Starting at \$YY00, the current mapping info has this structure.
  {\setlength{\tabcolsep}{2mm}
  \begin{tabular}{|c|c|p{6.9cm}|}
  \hline
  \textbf{Offset} & \textbf{Type} & \textbf{Description} \\
  \hline
  0 & word & MAPLO \\
  2 & word & MAPHI \\
  4 & byte & The megabyte offset for MAPLO \\
  5 & byte & The megabyte offset for MAPHI \\
  \hline
  \end{tabular}
  }
\item [Postconditions:]
  The CPU continues execution with the new memory mapping.
\item [Errors:]
  \errordesc{10}{invalid address}{The Y register is $>$ \$7E.}
\item [History:]
  \availablefrom{1.2}
\item [Remarks:]
  You must take care when changing the memory mapping. Hyppo will not take any
  steps to ensure the instructions after the STA are executed regardless of the
  mapping. If you change the mapping of the block where the program counter
  points to, the CPU will resume with the instructions in the newly mapped
  block.

  MAPLO is the mapping for \$0000 - \$7FFF.

  MAPHI is the mapping for \$8000 - \$FFFF.

  See \TODO{} for more information on MEGA65 memory mapping and banking.
\end{hyppotrap}


% ******************************************************************************
% switch_to_task
% ******************************************************************************
\newpage
\begin{hyppotrap}{switch\_to\_task}{D640}{6C}
\item [Service:]
  Switches context to another Hyppo task.
\notimplemented
\end{hyppotrap}


% ******************************************************************************
% terminateothertask
% ******************************************************************************
\begin{hyppotrap}{terminateothertask}{D640}{60}
\item [Service:]
  Terminates another Hyppo task.
\notimplemented
\end{hyppotrap}


% ******************************************************************************
% toggle_force_4502
% ******************************************************************************
\newpage
\begin{hyppotrap}{toggle\_force\_4502}{D640}{72}
\item [Service:]
  Toggles the CPU personality between 45GS02 and 6502.
\item [Outputs:]
  \register{A}{If bit 5 is set, the CPU is in the 45GS02 personality. If bit 5
  is clear, the CPU is in the 6502 personality.}
\item [History:]
  \availablefrom{1.2}
\item [Remarks:]
  The others bits of the A register are undefined. Do not expect them to be
  zero.

  In the 6502 personality, none of the new opcodes of the 65C02, 65CE02, 4510
  or 45GS02 are available. These are replaced with the original --- and often
  strange --- behaviour of the undefined opcodes of the 6502.

  \textbf{Warning} This feature is incomplete and untested. Most undocumented
  6502 opcodes do not operate correctly when the 6502 personality is enabled.
\item [Example:] Enables the 45GS02 personality regardless of the CPU's current
  personality.

  \TODO
\end{hyppotrap}


% ******************************************************************************
% toggle_rom_writeprotect
% ******************************************************************************
\newpage
\begin{hyppotrap}{toggle\_rom\_writeprotect}{D640}{70}
\item [Service:]
  Toggles the write-protection for \$20000 -- \$3FFFF.
\item [Outputs:]
  \register{A}{If bit 2 is set, \$20000 -- \$3FFFF cannot be written to.}
\item [History:]
  \availablefrom{1.2}
\item [Remarks:]
  The others bits of the A register are undefined. Do not expect them to be
  zero.

  If you want to toggle but disable or enable the protection, you can use
  \hypporef{rom\_writeenable} and \hypporef{rom\_writeprotect}.
\end{hyppotrap}


% ******************************************************************************
% writeintotask
% ******************************************************************************
\newpage
\begin{hyppotrap}{writeintotask}{D640}{56}
\item [Service:]
  Writes into the memory of another Hyppo task.
\notimplemented
\end{hyppotrap}



% ==============================================================================
% System Partition
% ==============================================================================
\newpage
\section{System Partition Services}


% ******************************************************************************
% configsector_apply
% ******************************************************************************
\begin{hyppotrap}{configsector\_apply}{D642}{04}
\item [Service:]
  Applies the system configuration sector currently loading into memory.
\item [History:]
  \availablefrom{1.2}
\end{hyppotrap}


% ******************************************************************************
% configsector_read
% ******************************************************************************
\begin{hyppotrap}{configsector\_read}{D642}{00}
\item [Service:]
  Reads the system configuration sector into memory.
\item [History:]
  \availablefrom{1.2}
\end{hyppotrap}


% ******************************************************************************
% configsector_write
% ******************************************************************************
\begin{hyppotrap}{configsector\_write}{D642}{02}
\item [Service:]
  Writes the system configuration sector from memory.
\item [History:]
  \availablefrom{1.2}
\end{hyppotrap}


% ******************************************************************************
% dmagic_autoset
% ******************************************************************************
\begin{hyppotrap}{dmagic\_autoset}{D642}{06}
\item [Service:]
  Sets the DMAgic revision based on the loaded ROM.
\item [History:]
  \availablefrom{1.2}
\end{hyppotrap}



% ==============================================================================
% Freezer
% ==============================================================================
\newpage
\section{Freezer Services}


% ******************************************************************************
% freeze_self
% ******************************************************************************
\begin{hyppotrap}{freeze\_self}{D67F}{xx}
\item [Service:]
  Launches the freezer.
\item [History:]
  \availablefrom{1.2}
\end{hyppotrap}


% ******************************************************************************
% get_slot_count
% ******************************************************************************
\begin{hyppotrap}{get\_slot\_count}{D642}{16}
\item [Service:]
  Gets the number of freeze slots
\item [History:]
  \availablefrom{1.2}
\end{hyppotrap}


% ******************************************************************************
% locate_freeze_slot
% ******************************************************************************
\begin{hyppotrap}{locate\_freeze\_slot}{D642}{10}
\item [Service:]
  Locates the first sector of a freeze slot
\item [History:]
  \availablefrom{1.2}
\end{hyppotrap}


% ******************************************************************************
% read_freeze_region_list
% ******************************************************************************
\begin{hyppotrap}{read\_freeze\_region\_list}{D642}{14}
\item [Service:]
  Reads the freeze region list
\item [History:]
  \availablefrom{1.2}
\end{hyppotrap}


% ******************************************************************************
% unfreeze_from_slot
% ******************************************************************************
\begin{hyppotrap}{unfreeze\_from\_slot}{D642}{12}
\item [Service:]
  Unfreezes from a freeze slot
\item [History:]
  \availablefrom{1.2}
\end{hyppotrap}
