
\chapter{Special Keyboard Controls and Sequences}


\section{ASCII Codes and CHR\$}

\label{appendix:asciicodes}

You can use the PRINT CHR\$(X) statement to print a character.
Below is the full table of ASCII codes you can print by index.
For example, by using index 65 from the table below as:
PRINT CHR\$(65) you will print the letter 'A'.

You can also do the reverse with the ASC statement.
For example:
PRINT ASC("A")
Will output 65, which matches in the ASCII code table.


\begin{center}
\setlength{\def\arraystretch{1.5}\tabcolsep}{6pt}
\begin{longtable}{ c c | c c  | c c}
	\textbf{CHR\$} & \textbf{Prints} & \textbf{CHR\$} & \textbf{Prints} & \textbf{CHR\$} & \textbf{Prints}\\
  \hline
	\endhead

	 0	& 								&	17	&	\megakey{$\downarrow$}		& 34	&	" \\
	 1	&									&	18	& \specialkey{RVS ON}						& 35	&	\# \\
	 2	&									&	19	& \specialkey{CLR HOME}					& 36	& \$ \\
	 3	&									& 20	& \specialkey{INST DEL}					& 37	& \% \\
	 4	&									& 21	& 													& 38	& \& \\
	 5	& \megakey{WHT}		& 22	&														& 39	& ' \\
	 6	& 								& 23	&														& 40	& ( \\
	 7	&									& 24	&														& 41	& ) \\
	 8	&	\multicolumn{1}{r|}{\small{DISABLE} \specialkey{SHIFT}\megasymbolkey}								& 25	&														& 42	& * \\
	 9	&	\multicolumn{1}{r|}{\small{ENABLE} \specialkey{SHIFT}\megasymbolkey}								& 26	&														& 43	& + \\
	10	&									& 27	&														& 44	& , \\
	11	&									& 28	& \megakey{RED}							& 45	& - \\
	12	&									& 29	&	\megakey{$\rightarrow$}		&	46	& . \\
	13	&	\megakey{RETURN}& 30	& \megakey{GRN}							& 47	& / \\
	14	&	\small{LOWER CASE}			& 31	& \megakey{BLU}							& 48	& 0 \\
	15	&									& 32	& \megakey{SPACE}						& 49	& 1 \\
	16	&									& 33	& !													& 50	& 2 \\

\end{longtable}
\end{center}


\newpage



\begin{center}
\setlength{\def\arraystretch{1.5}\tabcolsep}{6pt}
\begin{longtable}{ c c | c c  | c c}
	\textbf{CHR\$} & \textbf{Prints} & \textbf{CHR\$} & \textbf{Prints} & \textbf{CHR\$} & \textbf{Prints}\\
  \hline
	\endhead

	51	&	3		&	75	&	K					&	99	& \graphicsymbol{C}\\
	52	&	4		&	76	&	L					&	100	& \graphicsymbol{D}\\
	53	&	5		&	77	&	M					&	101	& \graphicsymbol{E}\\
	54	&	6		&	78	&	N					&	102	& \graphicsymbol{F}\\
	55	&	7		&	79	&	O					&	103	& \graphicsymbol{G}\\
	56	&	8		&	80	& P						&	104	& \graphicsymbol{H}\\
	57	&	9		&	81	& Q						&	105	& \graphicsymbol{I}\\
	58	&	:		&	82	& R						&	106	& \graphicsymbol{J}\\
	59	&	;		&	83	& S						&	107	& \graphicsymbol{K}\\
	60	&	<		&	84	& T						&	108	& \graphicsymbol{L}\\
	61	&	=		&	85	& U						&	109	& \graphicsymbol{M}\\
	62	&	>		&	86	& V						&	110	& \graphicsymbol{N}\\
	63	&	?		&	87	& W						&	111	& \graphicsymbol{O}\\
	64	&	\@		&	88	& X						&	112	& \graphicsymbol{P}\\
	65	&	A		&	89	& Y						&	113	& \graphicsymbol{Q}\\
	66	&	B		&	90	& Z						&	114	& \graphicsymbol{R}\\
	67	&	C		&	91	& [						&	115	& \graphicsymbol{S}\\
	68	&	D		&	92	& \pounds				&	116	& \graphicsymbol{T}\\
	69	&	E		&	93	& ]						&	117	& \graphicsymbol{U}\\
	70	&	F		&	94	& $\uparrow$			&	118	& \graphicsymbol{V}\\
	71	&	G		&	95	& $\leftarrow$			&	119	& \graphicsymbol{W}\\
	72	&	H		&	96	& \graphicsymbol{C}		&	120	& \graphicsymbol{X}\\
	73	&	I		&	97	& \graphicsymbol{A}		&	121	& \graphicsymbol{Y}\\
	74	&	J		&	98	& \graphicsymbol{B}		&	122	& \graphicsymbol{Z}\\

\end{longtable}
\end{center}



\newpage



\begin{center}
\setlength{\def\arraystretch{1.5}\tabcolsep}{6pt}
\begin{longtable}{ c c | c c  | c c}
	\textbf{CHR\$} & \textbf{Prints} & \textbf{CHR\$} & \textbf{Prints} & \textbf{CHR\$} & \textbf{Prints}\\
  \hline
	\endhead
	123	& \graphicsymbol{+} 				& 146	&	\specialkey{RVS OFF}	&  169	& \graphicsymbol{?}\\
	124	& \graphicsymbol{-} 				& 147	&	\specialkey{CLR HOME}	&  170	& \graphicsymbol{v}\\
	125	& \graphicsymbol{B} 				& 148	&	\specialkey{INST DEL}	&  171	& \graphicsymbol{q}\\
	126	& \graphicsymbol{\textbackslash}	& 149	&	\graphicsymbol{U}	&  172	& \graphicsymbol{d}\\
	127	& \graphicsymbol{]} 				& 150	&	\graphicsymbol{V}	&  173	& \graphicsymbol{z}\\
	128	& 									& 151	&	\graphicsymbol{W}	& 174	& \graphicsymbol{s}\\
	129	& \megakey{ORG} 					& 152	&	\graphicsymbol{X}	& 175	& \graphicsymbol{n}\\
	130	&  									& 153	&	\graphicsymbol{Y}	& 176	& \graphicsymbol{a}\\
	131	&  									& 154	&	\graphicsymbol{Z}	& 177	& \graphicsymbol{e}\\
	132	&  									& 155	&	\graphicsymbol{+}	& 178	& \graphicsymbol{r}\\
	133	& F1 								& 156	& \megakey{PUR}			& 179	& \graphicsymbol{w}\\
	134	& F3 								& 157	& \megakey{$\leftarrow$}& 180	& \graphicsymbol{h}\\
	135	& F5 								& 158	& \megakey{YEL}			& 181	& \graphicsymbol{j}\\
	136	& F7 								& 159	& \megakey{CYN}			& 182	& \graphicsymbol{l}\\
	137	& F2 								& 160	& \megakey{Space}		& 183	& \graphicsymbol{y}\\
	138	& F4								& 161	& \graphicsymbol{k}		& 184	& \graphicsymbol{u}\\
	139	&	F6								& 162	& \graphicsymbol{i}		& 185	& \graphicsymbol{p}\\
	140	&	F8								& 163	& \graphicsymbol{t}		& 186	& \graphicsymbol{\{}\\
	141	&	\specialkey{SHIFT}\megakey{return}	& 164	& \graphicsymbol{[}		& 187	& \graphicsymbol{f}\\
	142	&	\small{UPPERCASE}	& 165	& \graphicsymbol{g}		& 188	& \graphicsymbol{c}\\
	143	&									& 166	& \graphicsymbol{=}		& 189	& \graphicsymbol{x}\\
	144	& \megakey{BLK}						& 167	& \graphicsymbol{m}		& 190	& \graphicsymbol{v}\\
	145	&	\megakey{$\uparrow$}			& 168	& \graphicsymbol{/}		& 191	& \graphicsymbol{b}\\

\end{longtable}
\end{center}

\newpage



\section{Control codes}
\label{appendix:controlcodes}

\begin{center}
\setlength{\def\arraystretch{1.5}\tabcolsep}{6pt}
\begin{longtable}{c|L{5.5cm}}
	\textbf{Keyboard Control} & \textbf{Function}\\
   \hline
	\endhead

\megakey{CTRL} + \megakey{1} to \megakey{8} &
Choose from the first range of colours.\\

\megakey{CTRL} + \megakey{T} &
Backspace the character immediately to the left and to shift all rightmost characters one position to the left. This is the same function as the Backspace key.\\

\megakey{CTRL} + \megakey{Z} &
Tabs the cursor to the left.\\

\megakey{CTRL} + \megakey{E} &
Restores the colour of the cursor back to the default white.\\

\megakey{CTRL} + \megakey{Q} &
moves the cursor down one line at a time. This is the same function produced by the Cursor Down key.\\

\megakey{CTRL} + \megakey{G} &
produces a bell tone.\\

\megakey{CTRL} + \megakey{J} &
is a line feed and moves the cursor down one row. This is the same function produced by the \megakey{$\downarrow$} key.\\

\megakey{CTRL} + \megakey{U} &
backs up to the start of the previous word, or unbroken string of characters. If there are no characters between the current cursor position and the start of the line, the cursor will move to the first column of the current line.\\

\megakey{CTRL} + \megakey{W} &
advances forward to the start of the next word, or unbroken string of characters. If there are no characters between the current cursor position and the end of the line, the cursor will move to the first column of the next line.\\

\megakey{CTRL} + \megakey{B} &
turns on underline text mode. Turn off underline mode by pressing \megakey{ESC} then \megakey{O}.\\

\megakey{CTRL} + \megakey{N} &
changes the text case mode from uppercase to lowercase.\\

\megakey{CTRL} + \megakey{M} &
is the carriage return. This is the same function as the \megakey{Return} key.\\

\megakey{CTRL} + \megakey{]} &
is the same function as \megakey{$\rightarrow$}.\\

\megakey{CTRL} + \megakey{I} &
tabs forward to the right.\\

\megakey{CTRL} + \megakey{X} &
sets or clears the current screen column as a tab position.
 \megakey{CTRL} + \megakey{I} or \megakey{Z} will jump to all positions set with \megakey{X}. When there are no more tab positions, the cursor will stay at the end of the line with \megakey{CTRL} and \megakey{I}, or move to the start of the line in the case of \megakey{CTRL} and \megakey{Z}.\\

\megakey{CTRL} + \megakey{K} &
locks the uppercase/lowercase mode switch usually performed with \megasymbolkey and \megakey{Shift} keys.\\

\megakey{CTRL} + \megakey{L} &
enables the uppercase/lowercase mode switch that is performed with the \megasymbolkey and \megakey{Shift} keys.\\

\megakey{CTRL} + \megakey{[} &
is the same as pressing the \megakey{Esc} key.\\

\megakey{CTRL} + \megakey{*} &
enters the Matrix Mode Debugger.\\


\end{longtable}
\end{center}



\newpage

\section{Shifted codes}
\label{appendix:shiftedcodes}

\begin{center}
\setlength{\def\arraystretch{1.5}\tabcolsep}{6pt}
\begin{longtable}{c|L{5.5cm}}
	\textbf{Keyboard Control} & \textbf{Function}\\
   \hline
	\endhead

\megakey{Shift} + \specialkey{INST DEL} &
Insert a character in the current cursor position and move all characters to the right by one position.\\

\megakey{Shift} + \specialkey{HOME} &
Clear home, clear the entire screen and move the cursor to the home position.\\


\end{longtable}
\end{center}


\newpage



\section{Escape Sequences}
\label{appendix:escapesequences}

To perform an Escape Sequence, press and release the \megakey{Esc} key. Then press one of the following keys to perform the sequence:

\begin{center}
\setlength{\def\arraystretch{1.5}\tabcolsep}{6pt}
\begin{longtable}{C{2cm}|L{5.5cm}}
	\textbf{Key} & \textbf{Sequence}\\
   \hline
	\endhead

\megakey{X} &
Clears the screen and toggles between 40 and 80 column modes.\\

\megakey{@} &
Clears the screen starting from the cursor to the end of the screen.\\

\megakey{A} &
Enables the auto-insert mode. Any keys pressed will insert before other characters.\\

\megakey{B} &
Sets the bottom-right window area of the screen at the cursor position. All typed characters and screen activity will be restricted to the area. Also see \megakey{ESC} then \megakey{T}.\\

\megakey{C} &
Disables auto-insert mode, going back to overwrite mode.\\

\megakey{D} &
Deletes the current line and moves other lines up one position.\\

\megakey{E} &
Sets the cursor to non-flashing mode.\\

\megakey{F} &
Sets the cursor to regular flashing mode.\\

\megakey{G} &
Enables the bell which can be sounded using \megakey{CTRL} and \megakey{G}.\\

\megakey{H} &
Disable the bell so that pressing \megakey{CTRL} and \megakey{G} will have no effect.\\

\megakey{I} &
Inserts an empty line in the current cursor position and moves all subsequent lines down one position.\\

\megakey{J} &
Moves the cursor to start of current line.\\

\megakey{K} &
Move to end of the last non-whitespace character on the current line.\\

\megakey{L} &
Enables scrolling when the cursor down key is pressed at the bottom of the screen.\\

\megakey{M} &
Disables scrolling. When pressing the cursor down key at the bottom on the screen, the cursor will move to the top of the screen. The cursor is restricted at the top of the screen with the Cursor up key.\\

\megakey{O} &
Cancels the quote, reverse, underline and flash modes.\\

\megakey{P} &
Erases all characters from the cursor to the start of current line.\\

\megakey{Q} &
Erases all characters from the cursor to the end of current line.\\

\megakey{S} &
Switches the VIC-IV to colour range 16-31. These colours can be accessed with \megakey{CTRL} and keys \megakey{1} to \megakey{8} or \megasymbolkey and keys \megakey{1} to \megakey{8}.\\

\megakey{T} &
Set top-left window area of the screen at the cursor position. All typed characters and screen activity will be restricted to the area. Also see \megakey{ESC} then \megakey{B}.\\

\megakey{U} &
Switches the VIC-IV to colour range 0-15. These colours can be accessed with \megakey{CTRL} and keys \megakey{1} to \megakey{8} or \megasymbolkey and keys \megakey{1} to \megakey{8}.\\

\megakey{V} &
Scrolls the entire screen up one line.\\

\megakey{W} &
Scrolls the entire screen down one line.\\

\megakey{X} &
Toggles the 40/80 column display. The screen will also clear home.\\

\megakey{Y} &
Set the default tab stops (every 8 spaces) for the entire screen.\\

\megakey{Z} &
Clears all the tab stops. Any tabbing with \megakey{CTRL} and \megakey{I} will move the cursor to the end of the line.\\

\megakey{1} to \megakey{8} &
Choose from the second range of colours.\\

\end{longtable}
\end{center}
