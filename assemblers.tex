\chapter{Assemblers}

The short answer is that we recommend that you use the ACME
assembler.  The mnemonics used by ACME are now considered the official
onces for the 45GS02.  KickAsm is also quite suitable. The long answer continues below.

There are any number of assemblers that can be used to develop for the
MEGA65.  However only very few support the MEGA65's advanced CPU
features, or even the C65 4510 instructions.

Four different
assemblers have been used for different parts of the MEGA65 software,
reflecting the rather haphazard history of developing assembler
support for the MEGA65.  The first assembler to support the MEGA65 was
{\bf Ophis}, as it seemed the easiest to patch.  {\bf KickAss} later gained
support, followed by ACME. {\bf ACME} added support at a convenient time, when
the extended instructions, their names and syntax had sufficiently stabilised.

From the perspective of the MEGA65 team, ACME has two advantages over KickAsm:
First, it is open source, which fits the ethos of the project, and ensures long-term
availability. Second, it is written in C, rather than Java, which means that
it may be possible to port to run natively on the MEGA65 at some point in
the future.

Our fork of {\bf CA65} (\url{https://github.com/mega65/cc65}), the assembler
used in CC65, also now has the ability 
to detect the MEGA65's CPU, but has no explicit support for the
processor's features.  This is an important fix, however, as otherwise
software using the CC65 processor detection routine thinks that the
MEGA65 has a 65816, which can result in the execution of incompatible
instructions. This causes problems with Synthmark64 0.2, for example.
