\chapter{GETTING STARTED}
\phantomsection
\section{Keyboard}
\label{cha:getting-started}

Now that you have everything connected, it's time to get familiar with the MEGA65 keyboard.

You may notice that the keyboard is a little different from the standard used on computers today. While most keys will be in familiar positions, there are some specialised keys, and some with special graphic symbols marked on the front.

Here's a brief description of how some of these special keys function.

\subsection{Command Keys}

The Command Keys are: \specialkey{RETURN}, \specialkey{SHIFT}, \specialkey{Ctrl}, \megasymbolkey and \widekey{RESTORE}.

\subsubsection{RETURN}

Pressing the \specialkey{RETURN} key enters the information you have typed into the MEGA65's memory. The computer will either act on a command, store some information, or return you an error if you made a mistake.

\subsubsection{SHIFT}

The two \specialkey{SHIFT} keys are located on the left and the right. They work very much like Shift on a regular keyboard, however they also perform some special functions too.

In upper case mode, holding down \specialkey{SHIFT} and pressing any key with a graphic symbol on the front produces the right hand symbol on that key. For example, \specialkey{SHIFT} and \megakey{J} prints the \graphicsymbol{J}character.

In lower case mode, pressing \specialkey{SHIFT} and a letter key prints the upper case letter on that key.

Finally, holding down the \specialkey{SHIFT} key and pressing a Function key accesses the function shown on the front of that key. For example: \specialkey{SHIFT} and \megakey{F1} activates \textbf{F2}.


\subsubsection{SHIFT LOCK}

In addition to the Shift key is \specialkey{SHIFT\\LOCK}. Press this key to lock down the Shift function. Now any key you press prints the character to the screen as if you were holding down \specialkey{SHIFT}. That includes special graphic characters.

\subsubsection{CTRL}

\specialkey{CTRL} is the Control key. Holding down \specialkey{CTRL} and pressing another key allows you to perform Control Functions. For example, holding down \specialkey{CTRL} and one of the number keys allows you to change text colours.

There are some examples of this in \bookvref{sec:screen-editor}, and all the Control Functions are listed in \bookvref{appendix:controlcodes}.

If a program is being listed to the screen, holding down \specialkey{CTRL} slows down the display of each line on the screen.

Holding \specialkey{CTRL} and pressing \megakey{*} enters the Matrix Mode Debugger.

\subsubsection{RUN/STOP}

Normally, pressing the \specialkey{RUN\\STOP} key stops execution of a program. Holding \specialkey{SHIFT} while pressing \specialkey{RUN\\STOP} loads the first program from disk.

Programs are able to disable the \specialkey{RUN STOP} key.

You can boot your machine into the machine code monitor by holding down \specialkey{RUN\\STOP} and pressing reset on the MEGA65.

\subsubsection{RESTORE}

The computer screen can be restored to a clean state without clearing the memory by holding down the \specialkey{RUN\\STOP} key and tapping \widekey{RESTORE}. This combination also resets operating system vectors and re-initialises the screen editor, which makes it a handy combination if the computer has become a little confused.

Programs are able to disable this key combination.

Enter the Freeze Menu by holding down \widekey{RESTORE} for more than one second. You can access the machine code monitor via the Freeze menu.

\newpage

\subsubsection{THE CURSOR KEYS}

At the bottom right hand of the keyboard are the cursor keys. These four directional keys allow you move the cursor to any position for onscreen editing.

The cursor moves in the direction indicated on the keys: \megakey{$\leftarrow$} \megakey{$\uparrow$} \megakey{$\rightarrow$} \megakey{$\downarrow$}

However, it is also possible to move the cursor up using \specialkey{SHIFT} and \megakey{$\downarrow$}. In the same way you can move the cursor left using \specialkey{SHIFT} and \megakey{$\rightarrow$}.

You don't have to keep pressing a cursor key over and over. When moving the cursor a long way, you can keep the key pressed in. When you are finished, release the key.

\subsubsection{INSerT/DELete}

This is the INSERT / DELETE key. When pressing \specialkey{INST\\DEL}, the character to the left is deleted, and all characters to the right are shifted one position to the left.

To insert a character, hold the \specialkey{SHIFT} key and press \specialkey{INST\\DEL}. All the characters are shifted to the right. This allows you to type a letter, number or any other character into the newly inserted space.


\subsubsection{CLeaR/HOME}

Pressing the \specialkey{CLR\\HOME} key returns the cursor into the top left-most position of the screen.

Holding down \specialkey{SHIFT} and pressing \specialkey{CLR\\HOME} clears the entire screen and places the cursor into the top left-most position of the screen.

\subsubsection{MEGA KEY}

The \megasymbolkey key or the MEGA key provides a number of different functions and special utilities.

Holding the \specialkey{SHIFT} key and pressing \megasymbolkey switches between lower and upper case character modes.

Holding \megasymbolkey and pressing any key with graphic symbols on the front prints the left-most graphic symbol to the screen.

Holding \megasymbolkey and pressing any key that shows a single graphic symbol on the front prints that graphic symbol to the screen.

Holding \megasymbolkey and pressing a number key switches to one of the colours in the second range.

Holding \megasymbolkey and pressing \specialkey{TAB} enters the Matrix Mode Debugger.

When switching on the MEGA65 or pressing the reset button on the side, while holding down \megasymbolkey switches the MEGA65 into C64 mode.

\subsubsection{NO SCROLL}
If a program is being listed to the screen, pressing \specialkey{NO\\SCROLL} freezes the screen output. Not available in C64 mode.


\subsection{Function Keys}

There are seven Function keys available for use by software applications, \megakey{F1} \megakey{F3} \megakey{F5} \megakey{F7} \megakey{F9} \megakey{F11} and \megakey{F13} to perform functions with a single press.

Hold \specialkey{SHIFT} to access \megakey{F2} through to \megakey{F14} as shown on the front of each Function key.

Only Function keys \megakey{F1} to \megakey{F8} are available in C64 mode.

\subsubsection{HELP}

The \specialkey{HELP} key can be used by software and acts as an \megakey{F15} / \megakey{F16} key.

\subsubsection{ALT}

Holding \specialkey{ALT} down while pressing other keys can be used by software to perform functions. Not available in C64 mode.

Holding \specialkey{ALT} down when switching the MEGA65 on activates the Utility Menu. You can format the SD card or enter the MEGA65 Configuration Utility to select the default video mode and other settings, or test your keyboard.

\subsubsection{CAPS LOCK}

The \specialkey{CAPS\\LOCK} works like \specialkey{SHIFT\\LOCK} in C65 and MEGA65 modes, but only modifies the alphabet keys.
Also, holding the \specialkey{CAPS\\LOCK} down forces the processor to run at the maximum speed. This can be used, for example,
to speed up loading from the internal disk drive or SD card, or to greatly speed up the de-packing process after a program is run.
This can reduce the loading and de-packing time from many seconds to as little as a 10th of a second.
