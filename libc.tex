\chapter{MEGA65 Standard C Library}

A C library is being developed for the MEGA65, and which already
includes a number of useful features. This library is available from
\url{http://github.com/mega65/mega65-libc}. The procedures,
functions and definitions it provides are documented in a separate
chapter.

The MEGA65 libc is currently available only for CC65, although we would
welcome someone maintaining a KickC port of it.

\section{Structure and Usage}

The MEGA65 libc is purposely provided in source-form only, and with groups
of functions in separate files, and with separate header files for including.
The idea is that you include only the header files that you require, and
add only the source files required to the list of source files of the programme
you are compiling.  This avoids the risk of the compiler including functions
in your compiled programme that are never used, and thus wasting precious memory
space.

Note that some library source files are written in C, and thus are present as
files with a \stw{.c} extension, while others are written in assembly language
either for efficiency or out of necessity, and have a \stw{.s} extension.

Typical usage is to either have the mega65-libc source code checked out in an
adjacent directory, or within the source directory of your own project.  In the
latter case, this can be done using the git submodule facility.

The following sections document each of the header files and the corresponding
functions that they provide.

\section{conio.h}


\subsection{conionit}
\index{conionit}
\begin{description}[leftmargin=2cm,style=nextline]
\item [Description:] {Initializes the library internal state}
\item [Syntax:] \stw{void conioinit(void)}
\item [Notes:] {This must be called before using any conio library function.}
\end{description}

\subsection{setscreenaddr}
\index{setscreenaddr}
\begin{description}[leftmargin=2cm,style=nextline]
\item [Description:] {Sets the screen RAM start address}
\item [Syntax:] \stw{void setscreenaddr(long addr);}
\item [Parameters:]
\begin{description}\item[]
\item [{\em addr}:] {The address to set as start of screen RAM}
\end{description}
\item [Notes:] {No bounds check is performed on the selected address}
\end{description}

\subsection{getscreenaddr}
\index{getscreenaddr}
\begin{description}[leftmargin=2cm,style=nextline]
\item [Description:] {Returns the screen RAM start address}
\item [Syntax:] \stw{long getscreenaddr(void);}
\item [Desription:] {The current screen RAM address start address.}
\end{description}

\subsection{clrscr}
\index{clrscr}
\begin{description}[leftmargin=2cm,style=nextline]
\item [Description:] {Clear the text screen. }
\item [Syntax:] \stw{void clrscr(void)}
\item [Notes:] {Color RAM will be cleared with current text color}
\end{description}

\subsection{getscreensize}
\index{getscreensize}
\begin{description}[leftmargin=2cm,style=nextline]
\item [Description:] {Returns the dimensions of the text screen}
\item [Syntax:] \stw{void getscreensize(unsigned char* width, unsigned char* height)}
\item [Parameters:]
\begin{description}\item[]
\item [{\em width}:] {Pointer to location where width will be returned}
\item [{\em width}:] {Pointer to location where height will be returned}
\end{description}
\end{description}

\subsection{setscreensize}
\index{setscreensize}
\begin{description}[leftmargin=2cm,style=nextline]
\item [Description:] {Sets the dimensions of the text screen}
\item [Syntax:] \stw{void setscreensize(unsigned char width, unsigned char height)}
\item [Parameters:]
\begin{description}\item[]
\item [{\em width}:] {The width in columns (40 or 80)}
\item [{\em width}:] {The height in rows (25 or 50)}
\end{description}
\item [Notes:] {Currently only 40/80 and 25/50 are accepted. Other values are ignored.}
\end{description}

\subsection{set16bitcharmode}
\index{set16bitcharmode}
\begin{description}[leftmargin=2cm,style=nextline]
\item [Description:] {Sets or clear the 16-bit character mode}
\item [Syntax:] \stw{void set16bitcharmode(unsigned char f)}
\item [Parameters:]
\begin{description}\item[]
\item [{\em f}:] {Set true to set the 16-bit character mode}
\end{description}
\end{description}

\subsection{setextendedattrib}
\index{setextendedattrib}
\begin{description}[leftmargin=2cm,style=nextline]
\item [Description:] {Sets or clear the VIC-III extended attributes mode to support blink, underline, bold and highlight.}
\item [Syntax:] \stw{void setextendedattrib(unsigned char f)}
\item [Parameters:]
\begin{description}\item[]
\item [{\em f}:] {Set true to set the extended attributes mode}
\end{description}
\end{description}

\subsection{togglecase}
\index{togglecase}
\begin{description}[leftmargin=2cm,style=nextline]
\item [Description:] {Toggle the current character set case}
\item [Syntax:] \stw{void togglecase(void)}
\end{description}

\subsection{bordercolor}
\index{bordercolor}
\begin{description}[leftmargin=2cm,style=nextline]
\item [Description:] {Sets the current border color}
\item [Syntax:] \stw{void bordercolor(unsigned char c)}
\item [Parameters:]
\begin{description}\item[]
\item [{\em c}:] {The color to set}
\end{description}
\end{description}

\subsection{bgcolor}
\index{bgcolor}
\begin{description}[leftmargin=2cm,style=nextline]
\item [Description:] {Sets the current screen (background) color}
\item [Syntax:] \stw{void bgcolor(unsigned char c)}
\item [Parameters:]
\begin{description}\item[]
\item [{\em c}:] {The color to set}
\end{description}
\end{description}

\subsection{textcolor}
\index{textcolor}
\begin{description}[leftmargin=2cm,style=nextline]
\item [Description:] {Sets the current text color}
\item [Syntax:] \stw{void textcolor(unsigned char c)}
\item [Parameters:]
\begin{description}\item[]
\item [{\em c}:] {The color to set}
\end{description}
\end{description}

\subsection{revers}
\index{revers}
\begin{description}[leftmargin=2cm,style=nextline]
\item [Description:] {Enable the reverse attribute}
\item [Syntax:] \stw{void revers(unsigned char c)}
\item [Parameters:]
\begin{description}\item[]
\item [{\em enable}:] {0 to disable, 1 to enable}
\end{description}
\item [Notes:] {Extended attributes mode must be active. See setextendedattrib.}
\end{description}

\subsection{highlight}
\index{highlight}
\begin{description}[leftmargin=2cm,style=nextline]
\item [Description:] {Enable the highlight attribute}
\item [Syntax:] \stw{void highlight(unsigned char c)}
\item [Parameters:]
\begin{description}\item[]
\item [{\em enable}:] {0 to disable, 1 to enable}
\end{description}
\item [Notes:] {Extended attributes mode must be active. See setextendedattrib.}
\end{description}

\subsection{blink}
\index{blink}
\begin{description}[leftmargin=2cm,style=nextline]
\item [Description:] {Enable the blink attribute}
\item [Syntax:] \stw{void blink(unsigned char c)}
\item [Parameters:]
\begin{description}\item[]
\item [{\em enable}:] {0 to disable, 1 to enable}
\end{description}
\item [Notes:] {Extended attributes mode must be active. See setextendedattrib.}
\end{description}

\subsection{underline}
\index{underline}
\begin{description}[leftmargin=2cm,style=nextline]
\item [Description:] {Enable the underline attribute}
\item [Syntax:] \stw{void underline(unsigned char c)}
\item [Parameters:]
\begin{description}\item[]
\item [{\em enable}:] {0 to disable, 1 to enable}
\end{description}
\item [Notes:] {Extended attributes mode must be active. See setextendedattrib.}
\end{description}

\subsection{cellcolor}
\index{cellcolor}
\begin{description}[leftmargin=2cm,style=nextline]
\item [Description:] {Sets the color of a character cell}
\item [Syntax:] \stw{void cellcolor(unsigned char x, unsigned char y, unsigned char c)}
\item [Parameters:]
\begin{description}\item[]
\item [{\em x}:] {The cell X-coordinate}
\item [{\em y}:] {The cell Y-coordinate}
\item [{\em c}:] {The color to set}
\end{description}
\item [Notes:] {No screen bounds checks are performed; out of screen behavior is undefined }
\end{description}

\subsection{fillrect}
\index{fillrect}
\begin{description}[leftmargin=2cm,style=nextline]
\item [Description:] {Fill a rectangular area with character and color value}
\item [Syntax:] \stw{void fillrect(const RECT *rc, unsigned char ch, unsigned char col)}
\item [Parameters:]
\begin{description}\item[]
\item [{\em rc}:] {A RECT structure specifying the box coordinates}
\item [{\em ch}:] {A char code to fill the rectangle}
\item [{\em col}:] {The color to fill}
\end{description}
\item [Notes:] {No screen bounds checks are performed; out of screen behavior is undefined }
\end{description}

\subsection{box}
\index{box}
\begin{description}[leftmargin=2cm,style=nextline]
\item [Description:] {Draws a box with graphic characters}
\item [Syntax:] \stw{void box(const RECT *rc, unsigned char color, unsigned char style, unsigned char clear, unsigned char shadow)}
\item [Parameters:]
\begin{description}\item[]
\item [{\em rc}:] {A RECT structure specifying the box coordinates}
\item [{\em color}:] {The color to use for the graphic characters}
\item [{\em style}:] {The style for the box borders. Can be set to BOX\_STYLE\_ROUNDED, BOX\_STYLE\_INNER, BOX\_STYLE\_OUTER, BOX\_STYLE\_MID }
\item [{\em clear}:] {Set to 1 to clear the box interior with the selected color}
\item [{\em shadow}:] {Set to 1 to draw a drop shadow}
\end{description}
\item [Notes:] {No screen bounds checks are performed; out of screen behavior is undefined }
\end{description}

\subsection{hline}
\index{hline}
\begin{description}[leftmargin=2cm,style=nextline]
\item [Description:] {Draws an horizontal line.}
\item [Syntax:] \stw{void hline(unsigned char x, unsigned char y, unsigned char len, unsigned char style)}
\item [Parameters:]
\begin{description}\item[]
\item [{\em x}:] {The line start X-coordinate}
\item [{\em y}:] {The line start Y-coordinate}
\item [{\em len}:] {The line length}
\item [{\em style}:] {The style for the line. See HLINE\_ constants for available styles. }
\end{description}
\item [Notes:] {No screen bounds checks are performed; out of screen behavior is undefined }
\end{description}

\subsection{vline}
\index{vline}
\begin{description}[leftmargin=2cm,style=nextline]
\item [Description:] {Draws a vertical line.}
\item [Syntax:] \stw{void vline(unsigned char x, unsigned char y, unsigned char len, unsigned char style)}
\item [Parameters:]
\begin{description}\item[]
\item [{\em x}:] {The line start X-coordinate}
\item [{\em y}:] {The line start Y-coordinate}
\item [{\em len}:] {The line length}
\item [{\em style}:] {The style for the line. See VLINE\_ constants for available styles. }
\end{description}
\item [Notes:] {No screen bounds checks are performed; out of screen behavior is undefined }
\end{description}

\subsection{gohome}
\index{gohome}
\begin{description}[leftmargin=2cm,style=nextline]
\item [Description:] {Set the current position at home (0,0 coordinate)}
\item [Syntax:] \stw{void gohome(void)}
\end{description}

\subsection{gohome}
\index{gohome}
\begin{description}[leftmargin=2cm,style=nextline]
\item [Description:] {Set the current position at X,Y coordinates}
\item [Syntax:] \stw{void gotoxy(unsigned char x, unsigned char y)}
\item [Parameters:]
\begin{description}\item[]
\item [{\em x}:] {The new X-coordinate}
\item [{\em y}:] {The new Y-coordinate}
\end{description}
\item [Notes:] {No screen bounds checks are performed; out of screen behavior is undefined }
\end{description}

\subsection{gotox}
\index{gotox}
\begin{description}[leftmargin=2cm,style=nextline]
\item [Description:] {Set the current position X-coordinate}
\item [Syntax:] \stw{void gotox(unsigned char x)}
\item [Parameters:]
\begin{description}\item[]
\item [{\em x}:] {The new X-coordinate}
\end{description}
\item [Notes:] {No screen bounds checks are performed; out of screen behavior is undefined }
\end{description}

\subsection{gotoy}
\index{gotoy}
\begin{description}[leftmargin=2cm,style=nextline]
\item [Description:] {Set the current position Y-coordinate}
\item [Syntax:] \stw{void gotoy(unsigned char y)}
\item [Parameters:]
\begin{description}\item[]
\item [{\em y}:] {The new Y-coordinate}
\end{description}
\item [Notes:] {No screen bounds checks are performed; out of screen behavior is undefined }
\end{description}

\subsection{moveup}
\index{moveup}
\begin{description}[leftmargin=2cm,style=nextline]
\item [Description:] {Move current position up}
\item [Syntax:] \stw{void moveup(unsigned char count)}
\item [Parameters:]
\begin{description}\item[]
\item [{\em count}:] {The number of positions to move}
\end{description}
\item [Notes:] {No screen bounds checks are performed; out of screen behavior is undefined }
\end{description}

\subsection{movedown}
\index{movedown}
\begin{description}[leftmargin=2cm,style=nextline]
\item [Description:] {Move current position down}
\item [Syntax:] \stw{void movedown(unsigned char count)}
\item [Parameters:]
\begin{description}\item[]
\item [{\em count}:] {The number of positions to move}
\end{description}
\item [Notes:] {No screen bounds checks are performed; out of screen behavior is undefined }
\end{description}

\subsection{moveleft}
\index{moveleft}
\begin{description}[leftmargin=2cm,style=nextline]
\item [Description:] {Move current position left}
\item [Syntax:] \stw{void moveleft(unsigned char count)}
\item [Parameters:]
\begin{description}\item[]
\item [{\em count}:] {The number of positions to move}
\end{description}
\item [Notes:] {No screen bounds checks are performed; out of screen behavior is undefined }
\end{description}

\subsection{moveright}
\index{moveright}
\begin{description}[leftmargin=2cm,style=nextline]
\item [Description:] {Move current position right}
\item [Syntax:] \stw{void moveright(unsigned char count)}
\item [Parameters:]
\begin{description}\item[]
\item [{\em count}:] {The number of positions to move}
\end{description}
\item [Notes:] {No screen bounds checks are performed; out of screen behavior is undefined }
\end{description}

\subsection{wherex}
\index{wherex}
\begin{description}[leftmargin=2cm,style=nextline]
\item [Description:] {Return the current position X coordinate}
\item [Syntax:] \stw{unsigned char wherex(void)}
\item [Desription:] {The current position X coordinate}
\end{description}

\subsection{wherey}
\index{wherey}
\begin{description}[leftmargin=2cm,style=nextline]
\item [Description:] {Return the current position Y coordinate}
\item [Syntax:] \stw{unsigned char wherey(void)}
\item [Desription:] {The current position Y coordinate}
\end{description}

\subsection{cputc}
\index{cputc}
\begin{description}[leftmargin=2cm,style=nextline]
\item [Description:] {Output a single character to screen at current position}
\item [Syntax:] \stw{void cputc(unsigned char c)}
\item [Parameters:]
\begin{description}\item[]
\item [{\em c}:] {The character to output}
\end{description}
\end{description}

\subsection{cputnc}
\index{cputnc}
\begin{description}[leftmargin=2cm,style=nextline]
\item [Description:] {Output N copies of a character at current position}
\item [Syntax:] \stw{void cputnc(unsigned char count, unsigned char c)}
\item [Parameters:]
\begin{description}\item[]
\item [{\em c}:] {The character to output}
\item [{\em count}:] {The count of characters to print}
\end{description}
\end{description}

\subsection{cputhex}
\index{cputhex}
\begin{description}[leftmargin=2cm,style=nextline]
\item [Description:] {Output an hex-formatted number at current position}
\item [Syntax:] \stw{void cputhex(long n, unsigned char prec)}
\item [Parameters:]
\begin{description}\item[]
\item [{\em n}:] {The number to write}
\item [{\em prec}:] {The precision of the hex number, in digits. Leading zeros will be printed accordingly}
\end{description}
\item [Notes:] {The \$ symbol will be automatically added at beginning of string}
\end{description}

\subsection{cputdec}
\index{cputdec}
\begin{description}[leftmargin=2cm,style=nextline]
\item [Description:] {Output a decimal number at current position}
\item [Syntax:] \stw{void cputdec(long n, unsigned char padding, unsigned char leadingZ)}
\item [Parameters:]
\begin{description}\item[]
\item [{\em n}:] {The number to write}
\item [{\em padding}:] {The padding space to add before number}
\item [{\em leadingZ}:] {The leading zeros to print}
\end{description}
\end{description}

\subsection{cputs}
\index{cputs}
\begin{description}[leftmargin=2cm,style=nextline]
\item [Description:] {Output a string at current position}
\item [Syntax:] \stw{void cputs(const char* s)}
\item [Parameters:]
\begin{description}\item[]
\item [{\em s}:] {The string to print}
\end{description}
\item [Notes:] {No pointer check is performed.  If s is null or invalid, behavior is undefined }
\end{description}

\subsection{cputsxy}
\index{cputsxy}
\begin{description}[leftmargin=2cm,style=nextline]
\item [Description:] {Output a string at X,Y coordinates}
\item [Syntax:] \stw{void cputsxy (unsigned char x, unsigned char y, const char* s)}
\item [Parameters:]
\begin{description}\item[]
\item [{\em s}:] {The string to print}
\item [{\em x}:] {The X coordinate where string will be printed}
\item [{\em y}:] {The Y coordinate where string will be printed}
\end{description}
\item [Notes:] {No pointer check is performed.  If s is null or invalid, behavior is undefined }
\end{description}

\subsection{cputcxy}
\index{cputcxy}
\begin{description}[leftmargin=2cm,style=nextline]
\item [Description:] {Output a single character at X,Y coordinates}
\item [Syntax:] \stw{void cputcxy (unsigned char x, unsigned char y, unsigned char c)}
\item [Parameters:]
\begin{description}\item[]
\item [{\em x}:] {The X coordinate where character will be printed}
\item [{\em y}:] {The Y coordinate where character will be printed}
\item [{\em c}:] {The character to print}
\end{description}
\end{description}

\subsection{cputncxy}
\index{cputncxy}
\begin{description}[leftmargin=2cm,style=nextline]
\item [Description:] {Output N copies of a single character at X,Y coordinates}
\item [Syntax:] \stw{void cputncxy (unsigned char x, unsigned char y, unsigned char c)}
\item [Parameters:]
\begin{description}\item[]
\item [{\em x}:] {The X coordinate where character will be printed}
\item [{\em y}:] {The Y coordinate where character will be printed}
\item [{\em count}:] {The number of characters to output}
\item [{\em c}:] {The character to print}
\end{description}
\end{description}

\subsection{cprintf}
\index{cprintf}
\begin{description}[leftmargin=2cm,style=nextline]
\item [Description:] {Prints formatted output.

    Escape strings can be used to modify attributes, move cursor,etc,
    similar to PRINT in CBM BASIC. Available escape codes:

    Cursor positioning

    \textbackslash t           Go to next tab position (multiple of 8s) \\
    \textbackslash r           Return \\
    \textbackslash n           New-line (assume \textbackslash r like in C printf)

    {clr}
}
\item [Syntax:] \stw{unsigned char cprintf (const unsigned char* format, ...)}
\item [Parameters:]
\begin{description}\item[]
\item [{\em format}:] {The string to output. See escape codes for formatting options.}
\end{description}
\item [Notes:] {Currently no argument replacement is done with the variable arguments.}
\end{description}

\subsection{cgetc}
\index{cgetc}
\begin{description}[leftmargin=2cm,style=nextline]
\item [Description:] { Waits until a character is in the keyboard buffer and returns it }
\item [Syntax:] \stw{unsigned char cgetc (void);}
\item [Desription:] {The last character in the keyboard buffer }
\item [Notes:] {Returned values are ASCII character codes}
\end{description}

\subsection{kbhit}
\index{kbhit}
\begin{description}[leftmargin=2cm,style=nextline]
\item [Description:] { Returns the character in the keyboard buffer }
\item [Syntax:] \stw{unsigned char kbhit (void);}
\item [Desription:] {The character code in the keyboard buffer,  0 otherwise. }
\item [Notes:] {Returned values are ASCII character codes}
\end{description}

\subsection{getkeymodstate}
\index{getkeymodstate}
\begin{description}[leftmargin=2cm,style=nextline]
\item [Description:] {
   Return the key modifiers state, where bits:

    Bit           Meaning             Constant
    ----------------------------------------------------
    0             Right SHIFT state   KEYMOD\_RSHIFT
    1             Left  SHIFT state   KEYMOD\_LSHIFT
    2             CTRL state          KEYMOD\_CTRL
    3             MEGA/C= state       KEYMOD\_MEGA
    4             ALT state           KEYMOD\_ALT
    5             NOSCRL state        KEYMOD\_NOSCRL
    6             CAPSLOCK state      KEYMOD\_CAPSLOCK
    7             Reserved            -
    }
\item [Syntax:] \stw{unsigned char getkeymodstate(void)}
\item [Desription:] {A byte with the key modifier state bits.}
\end{description}

\subsection{flushkeybuf}
\index{flushkeybuf}
\begin{description}[leftmargin=2cm,style=nextline]
\item [Description:] {Flush the keyboard buffer}
\item [Syntax:] \stw{void flushkeybuf(void)}
\end{description}

\subsection{cinput}
\index{cinput}
\begin{description}[leftmargin=2cm,style=nextline]
\item [Description:] {Get input from keyboard, printing incoming characters at current position.}
\item [Syntax:] \stw{unsigned char cinput(char* buffer, unsigned char buflen, unsigned char flags)}
\item [Parameters:]
\begin{description}\item[]
\item [{\em buffer}:] {Target character buffer preallocated by caller}
\item [{\em buflen}:] {Target buffer length in characters}
\item [{\em flags}:] {Flags for input:  (default is accept all printable characters)

            CINPUT\_ACCEPT\_NUMERIC
            CINPUT\_ACCEPT\_LETTER
            CINPUT\_ACCEPT\_SYM
            CINPUT\_ACCEPT\_ALL
            CINPUT\_ACCEPT\_ALPHA
  }
\end{description}
\item [Desription:] {Successfully read characters in buffer}
\end{description}



\subsection{VIC\_BASE}

{\em VIC\_BASE} is a pre-processor macro that provides the base address of the
VIC-IV chip, i.e., \$D000.

{\em IS\_H640} is a pre-processor macro that returns 0 if the current VIC-III/IV
video mode is set to 320 pixels accross (40 column mode), and non-zero if it is set to 640 pixels across (80 column mode).
