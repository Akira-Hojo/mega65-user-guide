\chapter{45GS02 \& 6502 Instruction Sets}

The 45GS02 CPU is able to operate in native mode, where it
is compatible with the CSG 4510, and in 6502 compatibility mode,
where 6502 undocumented instructions, also known as illegal
instructions, are supported for compatibility.  The instruction set and
timing information for both personalities are listed in the following
sections.

\section{45GS02 Instruction Set}

\subsection{Opcode Map}

\begin{center}
\rotatebox{270}{
  \input{4510-opcodes}
  }
\end{center}

\subsection{Instruction Timing}

The following table lists the base cycle count for each opcode.
Note that the number of cycles depends on the speed setting of the
processor: Some instructions take more or fewer cycles when the
processor is running at full-speed, or a C65 compatibility 3.5MHz speed,
or at C64 compatibility 1MHz/2MHz speed.  More detailed information on
this is listed under each each instruction's information, but the high-level
view is:

\begin{itemize}
\item When the processor is running at 1MHz, all instructions take at least
  two cycles, and dummy cycles are re-inserted into Read-Modify-Write instructions,
  so that all instructions take exactly the same number of cycles as on a 6502.
\item The Read-Modify-Write instructions and all instructions that read a value from
  memory all require an extra cycle when operating at full speed, to allow signals
  to propagate within the processor.
\item The Read-Modify-Write instructions require an additional cycle if the operand
  is \$D019, as the dummy write is performed in this case.
  This is to improve compatibility with C64 software that frequently uses this
  ``bug'' of the 6502 to more rapidly acknowledge VIC-II interrupts.
\item Page-crossing and branch-taking penalties do not apply when the processor is
  running at full speed.
\item Many instructions require fewer cycles when the processor is running at full
  speed, as generally most non-bus cycles are removed. For example, Pushing and Pulling
  values to and from the stack requires only 2 cycles, instead of the 4 that that the
  6502 requires for these instructions.
\end{itemize}

Note that it is possible that further changes to processor timing will occur.

Similar issues apply to when the processor is in 6502 mode.
  
\begin{center}
\rotatebox{270}{
\input{4510-cycles}
}
\end{center}

\input{instructionset-4510}

\section{6502 Instruction Set}

NOTE: The mechanisms for switching from 4510 to 6502 CPU personality
have yet to be finalised.

NOTE: Not all 6502 illegal opcodes are currently implemented.

\subsection{Opcode Map}

\begin{center}
\rotatebox{270}{
\input{6502-opcodes}
}
\end{center}

\subsection{Instruction Timing}

The following table summarises the base instruction timing for 6502 mode.
Please also read the information for 4510 mode, as it discusses a number
of important factors that affect these figures.

\begin{center}
\rotatebox{270}{
\input{6502-cycles}
}
\end{center}

\input{instructionset-6502}

