\begingroup
\setlength{\def\arraystretch{1.1}\tabcolsep}{1pt}
% \OPC{Instruction}{Addr-Mode}{Bytes}{Cycles}
\newcommand{\OPC}[4]{\makecell{\begin{tabular}{>{\raggedright\arraybackslash}p{0.4cm}>{\raggedleft\arraybackslash}p{0.8cm}}
\fontsize{8pt}{0pt}\selectfont #3 & \fontsize{8pt}{0pt}\selectfont {\em #4} \\[-3pt]
\multicolumn{2}{c}{\fontsize{10pt}{0pt}\selectfont #1} \\[-3pt]
\multicolumn{2}{c}{\fontsize{8pt}{0pt}\selectfont #2}
\end{tabular}}}
\newcommand{\OPCQ}[4]{\makecell{\begin{tabular}{>{\raggedright\arraybackslash}p{0.4cm}>{\raggedleft\arraybackslash}p{0.8cm}}
\fontsize{8pt}{0pt}\selectfont #3 & \fontsize{8pt}{0pt}\selectfont {\em #4} \\[-3pt]
\multicolumn{2}{c}{\fontsize{10pt}{0pt}\selectfont #1} \\[-3pt]
\multicolumn{2}{c}{\fontsize{8pt}{0pt}\selectfont #2} \\[-12pt]
\fontsize{8pt}{0pt}\selectfont Q
\end{tabular}}}
% cell colours
\newcommand{\OPill}{\cellcolor[rgb]{1,.8,.8}}
\newcommand{\OPfar}{\cellcolor[rgb]{.8,1,.8}}
\newcommand{\OPquad}{\cellcolor[rgb]{.8,.8,1}}
\newcommand{\OPfarq}{\cellcolor[rgb]{.8,1,1}}
% binary logic
\newcommand{\binand}{$\mathit{AND}$}
\newcommand{\binor}{$\mathit{OR}$}
\newcommand{\binxor}{$\mathit{XOR}$}
\newcommand{\binnot}{$\mathit{NOT}$}
\newcommand{\binneg}{$\mathit{NEG}$}

\chapter{45GS02 \& 6502 Instruction Sets}

\section{Introduction}

The 45GS02 CPU is able to operate in native mode, where it
is compatible with the CSG 4510, and in 6502 compatibility mode,
where 6502 undocumented instructions, also known as illegal
instructions, are supported for compatibility.

\begin{quote}
{\bf WARNING:} This feature is incomplete and untested.  Most undocumented
6502 opcodes do not operate correctly when the 6502 personality is enabled.
\end{quote}

When in 4510 compatibility mode, the 45GS02 also supports a number
of extensions through {\em compound instructions}. These work be prefixing
the desired instruction's opcode with one or more {\em prefix bytes}, which
represent sequences of instructions that should not normally occur.  For example,
two \stw{NEG} instructions in a row acts as a prefix to tell the 45GS02 that the
following instruction will operate on 32 bits of data, instead of the usual 8 bits
of data.  This means that a 45GS02 instruction stream can be readily decoded or disassembled,
without needing to set special instruction length flags, as is the case with the 65816
family of microprocessors. The trade-off is increased execution time, as the 45GS02 must
skip over the prefix bytes.

The remainder of this chapter introduces the addressing modes, instructions, opcodes and
instruction timing data of the 45GS02, beginning with 6502 compatibility mode, before
moving on to 4510 compatibility mode, and the 45GS02 extensions.

\section{Stack Operations}

The stack is a area of memory where you can push data onto or fetch (pop) the latest piece
of data from it. Every stack operation that puts data to the stack (push) also changes the Stack
Pointer (SP) downwards (the stack starts at the top of the area and grows down), and every
stack operation that takes data from the stack (pop) will change the SP upwards.

So if you find something like
\begin{quote}
  STACK $\leftarrow$ VALUE
\end{quote}
implies that VALUE is pushed onto the STACK and SP is reduced by the number of bytes VALUE is
big. When pushing more than one byte, the MSB is pushed first followed by the LSB.

And in the opposite direction a operation like
\begin{quote}
  MEMORY $or$ REGISTER $\leftarrow$ STACK
\end{quote}
will pop data from the STACK and increment SP for each byte removed.

\section{Addressing Modes}

The 45GS02 supports 34 different addressing modes, which are explained below.
Many of these are very similar to one another, being variations of the normal 6502
or 65CE02 addressing modes, except that they accept either
32-bit pointers, operate on 32-bits of data, or both.

\subsection{Implied}

In this mode, there are no operands, as the precise function of the instruction is
implied by the instruction itself.  For example, the \screentext{INX} instruction increments
the X Register.

\subsection{Accumulator}

In this mode, the Accumulator is the operand. This is typically used to shift,
rotate or modify the value of the Accumulator Register in some way.  For example,
\screentext{INC A} increments the value in the Accumulator Register.

\subsection{Q Pseudo Register}

In this mode, the Q Pseudo Register is the operand. This is typically used to shift,
rotate or modify the value of the Q Pseudo Register in some way.  For example,
\screentext{ASLQ} shifts the value in the Q Pseudo Register left one bit.

Remember that the Q Pseudo Register is simply the A, X, Y and Z registers acting together
as a virtual 32-bit register, where A contains the least significant bits, and Z the
most significant bits. If you modify Q, you will modify the true registers, and similarly,
if you modify a true register, this will change the respective part of the Q register.

There are some cases where using a Q mode instruction can be
helpful for operating on the four true registers, for example, being able to quickly
load or store all four registers.

\subsection{Immediate Mode}

In this mode, the argument to the instruction is a value that is used directly.
This is indicated by proceeding the value with a \# character. Most assemblers allow
values to be entered in decimal, or in hexadecimal by preceding the value with a \$ sign,
in binary, by preceding the value with a \% sign.  For example, to set the Accumulator
Register to the value 5, you could use the following:

\begin{screenoutput}
LDA #5
\end{screenoutput}

The immediate argument is encoded as a single byte following the instruction.  For the above
case, the instruction stream would contain \$A9, the opcode for LDA immediate mode, followed
by \$05, the immediate operand.

\subsection{Immediate Word Mode}

In this mode, the argument is a 16-bit value that is used directly. There is only one instruction
which uses this addressing mode, \screentext{PHW}.  For example, to push the word \$1234
onto the stack, you could use:

\begin{screenoutput}
PHW #$1234
\end{screenoutput}

The low byte of the immediate value follows the opcode of the instruction.  The high byte of the
immediate value then follows that.  For the above example, the instruction stream would thus
be \$F4 \$34 \$12.

\subsection{Base-Page Mode}
\label{Base-Page (Zero-Page) Mode}

In this mode, the argument is an 8-bit address.  The upper 8-bits of the address are taken from
the Base-Page Register.  On 6502 processors, there is no Base-Page Register, and instead, the
upper 8-bits are always set to zero -- hence the name of this mode on the 6502: Zero-Page. On
the 45GS02, it is possible to move this ``Zero-Page'' to any page in the processor's 64KB view
of memory by setting the Base-Page Register using the \screentext{TAB} instruction. Base-Page
Mode allows faster access to a 256 region of memory, and uses less instruction bytes to do so.

The argument is encoded as a single byte that immediately follows the instruction opcode. For
example,

\begin{screenoutput}
LDA #12
\end{screenoutput}

would read the value stored in location \$12 in the Base-Page,
and put it into the Accumulator Register.  The instruction byte stream for this would be
\$85 \$12.

\subsection{Base-Page Quad Mode}
\label{Base-Page (Zero-Page) Quad Mode}

This mode is identical to Base-Page Mode, except that it reads a 32-bit word starting at the
specified address.

The argument is encoded as a single byte that immediately follows the instruction opcode. For
example,

\begin{screenoutput}
LDQ $12
\end{screenoutput}

would read the value stored in locations \$12 -- \$15 in the Base-Page,
and put them into the Q Pseudo Register.
The instruction byte stream for this would be \$42 \$42 \$85 \$12.  Note that this is the same as for
the Base-Page (Zero-Page) Mode, with the addition of the two \$42 prefix bytes.  Opcode \$42 is normally
NEG (negate the value in the A register).  When executed twice in a row, this returns the A value to its
original value.  The 45GS02 processor has special logic to recognises this sequence, so that it knows
to execute the next instruction using the Q Pseudo Register for that instruction.

See the note on page \pageref{Base-Page (Zero-Page) Mode} for more information about Base-Page and Zero-Page.

\subsection{Base-Page X-Indexed Mode}

This mode is identical to Base-Page Mode, except that the address is formed by taking the
argument, and adding the value of the X Register t<o it.  In 6502 mode, the result will always
be in the Base-Page, that is, any carry due to the addition from the low byte into the high byte
of the address will be ignored.  The encoding for this addressing mode is identical to Base-Page
Mode.

The argument is encoded as a single byte that immediately follows the instruction opcode.
For example,

\begin{screenoutput}
LDA $12,X
\end{screenoutput}

would read the value stored in location (\$12 + X) in the Base-Page,
and put it into the A register.  The instruction byte stream for this would be \$B5 \$12.

See the note on page \pageref{Base-Page (Zero-Page) Mode} for more information about Base-Page and Zero-Page.

\subsection{Base-Page Quad X-Indexed Mode}

This mode is identical to Base-Page Quad Mode, except that the address is formed by taking the
argument, and adding the value of the X Register to it.  In 6502 mode, the result will always
be in the Base-Page, that is, any carry due to the addition from the low byte into the high byte
of the address will be ignored.  The encoding for this addressing mode is identical to Base-Page Quad
Mode.

The argument is encoded as a single byte that immediately follows the instruction opcode.
For example,

\begin{screenoutput}
DEQ $12,X
\end{screenoutput}

would increment the 32-bit word stored at (\$12 + X) through to (\$15 + X) in the Base-Page,
and put it into the X register.  The instruction byte stream for this would be \$42 \$42 \$D6 \$12.

Note that LDQ is not available in this addressing mode.

See the note on page \pageref{Base-Page (Zero-Page) Mode} for more information about Base-Page and Zero-Page.
See the note on page \pageref{Base-Page (Zero-Page) Quad Mode} for more information on Quad Mode instructions.

\subsection{Base-Page Y-Indexed Mode}

This mode is identical to Base-Page Mode, except that the address is formed by taking the
argument, and adding the value of the Y Register to it.  In 6502 mode, the result will always
be in the Base-Page, that is, any carry due to the addition from the low byte into the high byte
of the address will be ignored.  The encoding for this addressing mode is identical to Base-Page
Mode.

The argument is encoded as a single byte that immediately follows the instruction opcode.
For example,

\begin{screenoutput}
LDX $12,Y
\end{screenoutput}

would read the value stored in location (\$12 + Y) in the Base-Page,
and put it into the X register.  The instruction byte stream for this would be \$B6 \$12.

See the note on page \pageref{Base-Page (Zero-Page) Mode} for more information about Base-Page and Zero-Page.

\subsection{Base-Page Quad Y-Indexed Mode}

This mode is identical to Base-Page Quad Mode, except that the address is formed by taking the
argument, and adding the value of the Y Register to it.  In 6502 mode, the result will always
be in the Base-Page, that is, any carry due to the addition from the low byte into the high byte
of the address will be ignored.  The encoding for this addressing mode is identical to Base-Page Quad
Mode.

Note that no instructions currently offer this addressing mode.

See the note on page \pageref{Base-Page (Zero-Page) Mode} for more information about Base-Page and Zero-Page.

\subsection{Base-Page Z-Indexed Mode}

This mode is identical to Base-Page Mode, except that the address is formed by taking the
argument, and adding the value of the Z Register to it.  In 6502 mode, the result will always
be in the Base-Page, that is, any carry due to the addition from the low byte into the high byte
of the address will be ignored.  The encoding for this addressing mode is identical to Base-Page
Mode.

Note that no instructions currently offer this addressing mode.

See the note on page \pageref{Base-Page (Zero-Page) Mode} for more information about Base-Page and Zero-Page.

\subsection{Base-Page Quad Z-Indexed Mode}

This mode is identical to Base-Page Quad Mode, except that the address is formed by taking the
argument, and adding the value of the Z Register to it.  In 6502 mode, the result will always
be in the Base-Page, that is, any carry due to the addition from the low byte into the high byte
of the address will be ignored.  The encoding for this addressing mode is identical to Base-Page Quad
Mode.

Note that no instructions currently offer this addressing mode.

See the note on page \pageref{Base-Page (Zero-Page) Mode} for more information about Base-Page and Zero-Page.

\subsection{Absolute Mode}

In this mode, the argument is an 16-bit address.  The low 8-bits of the address are taken from
the byte immediately following the instruction opcode. The upper 8-bits are taken from the
byte following that.  For example, the instruction

\begin{screenoutput}
LDA $1234
\end{screenoutput}

would read the
memory location \$1234, and place the read value into the Accumulator Register.  This would
be encoded as \$AD \$34 \$12.

\subsection{Absolute Quad Mode}

In this mode, the argument is an 16-bit address.  The low 8-bits of the address are taken from
the byte immediately following the instruction opcode. The upper 8-bits are taken from the
byte following that.
For example, the instruction

\begin{screenoutput}
LDQ $1234
\end{screenoutput}

would read the
memory locations \$1234 -- \$1237, and place the read values into the Q Pseudo Register.  This would
be encoded as \$42 \$42 \$AD \$34 \$12.

See the note on page \pageref{Base-Page (Zero-Page) Quad Mode} for more information on Quad Mode instructions.

\subsection{Absolute X-Indexed Mode}

This mode is identical to Absolute Mode, except that the address is formed by taking the
argument, and adding the value of the X Register to it.  If the indexing causes the address
to cross a page boundary, i.e., if the upper byte of the address changes, this may incur a
1 cycle penalty, depending on the processor mode and speed setting.
The encoding for this addressing mode is identical to Absolute Mode.
For example, the instruction

\begin{screenoutput}
LDA $1234,X
\end{screenoutput}

would read the
memory location (\$1234 + X), and place the value read from there into the A Register.  This would
be encoded as \$BD \$34 \$12.

\subsection{Absolute Quad X-Indexed Mode}

This mode is identical to Absolute Quad Mode, except that the address is formed by taking the
argument, and adding the value of the X Register to it.  If the indexing causes the address
to cross a page boundary, i.e., if the upper byte of the address changes, this may incur a
1 cycle penalty, depending on the processor mode and speed setting.
The encoding for this addressing mode is identical to Absolute Quad Mode.

For example, the instruction

\begin{screenoutput}
ROLQ $1234,X
\end{screenoutput}

would rotate left the 32-bit value 
at memory locations (\$1234+X) -- (\$1237+X), and write the result back to these same memory locations.  This would
be encoded as \$42 \$42 \$3E \$34 \$12.

See the note on page \pageref{Base-Page (Zero-Page) Quad Mode} for more information on Quad Mode instructions.

\subsection{Absolute Y-Indexed Mode}

This mode is identical to Absolute Mode, except that the address is formed by taking the
argument, and adding the value of the Y Register to it.  If the indexing causes the address
to cross a page boundary, i.e., if the upper byte of the address changes, this may incur a
1 cycle penalty, depending on the processor mode and speed setting.
The encoding for this addressing mode is identical to Absolute Mode.
For example, the instruction

\begin{screenoutput}
LDA $1234,Y
\end{screenoutput}

would read the
memory location (\$1234 + Y), and place the value read from there into the A Register.  This would
be encoded as \$B9 \$34 \$12.

\subsection{Absolute Quad Y-Indexed Mode}

This mode is identical to Absolute Quad Mode, except that the address is formed by taking the
argument, and adding the value of the Y Register to it.  If the indexing causes the address
to cross a page boundary, i.e., if the upper byte of the address changes, this may incur a
1 cycle penalty, depending on the processor mode and speed setting.
The encoding for this addressing mode is identical to Absolute Quad Mode.

Note that no instructions currently offer this addressing mode.

See the note on page \pageref{Base-Page (Zero-Page) Quad Mode} for more information on Quad Mode instructions.

\subsection{Absolute Indirect Mode}

In this mode, the 16-bit argument is the address that points to, i.e., contains the
address of actual byte to read.  For example, if memory location \$1234 contains \$78
and memory location \$1235 contains \$56, then

\begin{screenoutput}
JMP (\$1234)
\end{screenoutput}

would jump
to address \$5678.  The encoding for this addressing mode is identical to Absolute Mode,
 and thus this instruction would be encoded as \$6C \$34 \$12.

\subsection{Absolute Indirect X-Indexed Mode}

In this mode, the 16-bit argument is the address that points to, i.e., contains the
address of actual byte to read. It is identical to Absolute Indirect Mode, except that
 the value of the X Register is added to the pointer address.
For example, if the X Register contains the value \$04, memory location \$1238 contains \$78
and memory location \$1239 contains \$56, then

\begin{screenoutput}
JMP ($1234,X)
\end{screenoutput}

would jump
to address \$5678.
The encoding for this addressing mode is identical to Absolute Mode, and thus this instruction
would be encoded as \$7C \$34 \$12.

\subsection{Base-Page Indirect X-Indexed Mode}

This addressing mode is identical to Absolute Indirect X-Indexed Mode, except that the address
of the pointer is formed from the Base-Page Register (high byte) and the 8-bit operand (low byte).
The encoding for this addressing mode is identical to Base-Page Mode.

For example, if the X Register contains the value \$04, and the memory locations \$16 and \$17 in the current
Base-Page contained \$34 and \$12, respectively,
then

\begin{screenoutput}
LDA ($12,X)
\end{screenoutput}

would read the contents of memory location \$1234,
and store the result in the A register. This instruction would be encoded as \$A1 \$12.

See the note on page \pageref{Base-Page (Zero-Page) Mode} for more information about Base-Page and Zero-Page.

\subsection{Base-Page Quad Indirect X-Indexed Mode}

This addressing mode is identical to Base-Page Indirect X-Indexed Mode, except that the address
of the pointer is formed from the Base-Page Register (high byte) and the 8-bit operand (low byte).
The encoding for this addressing mode is identical to Base-Page Quad Mode.

Note that no instructions currently offer this addressing mode.

See the note on page \pageref{Base-Page (Zero-Page) Mode} for more information about Base-Page and Zero-Page.
See the note on page \pageref{Base-Page (Zero-Page) Quad Mode} for more information on Quad Mode instructions.

\subsection{Base-Page Indirect Y-Indexed Mode}

This addressing mode differs from the X-Indexed Indirect modes, in that the Y Register is
added to the address that is read from the pointer, instead of being added to the pointer.
This is a very useful mode, that is frequently used because it effectively provides access to
``the Y-th byte of the memory at the address pointed to by the operand.'' That is, it de-references
a pointer.  
The encoding for this addressing mode is identical to Base-Page Mode.

For example, if the Y Register contains the value \$04, and the memory locations \$12 and \$13 in the current
Base-Page contained \$78 and \$56, respectively,
then

\begin{screenoutput}
LDA ($12),Y
\end{screenoutput}

would read the contents of memory location \$567C (i.e., \$5678 + Y),
and store the result in the A register. This instruction would be encoded as \$B1 \$12.

See the note on page \pageref{Base-Page (Zero-Page) Mode} for more information about Base-Page and Zero-Page.

\subsection{Base-Page Quad Indirect Y-Indexed Mode}

This addressing mode is identical to the Base-Page Indirect Y-Indexed Mode, except that
32-bits of data are operated on. The encoding for this addressing mode is identical to
Base-Page Mode, except that it is prefixed by \$42, \$42.

Note that no instructions currently offer this addressing mode.

See the note on page \pageref{Base-Page (Zero-Page) Mode} for more information about Base-Page and Zero-Page.
See the note on page \pageref{Base-Page (Zero-Page) Quad Mode} for more information on Quad Mode instructions.

\subsection{Base-Page Indirect Z-Indexed Mode}

This addressing mode differs from the X-Indexed Indirect modes, in that the Z Register is
added to the address that is read from the pointer, instead of being added to the pointer.
This is a very useful mode, that is frequently used because it effectively provides access to
``the Z-th byte of the memory at the address pointed to by the operand.'' That is, it de-references
a pointer.
The encoding for this addressing mode is identical to Base-Page Mode.

For example, if the Z Register contains the value \$04, and the memory locations \$12 and \$13 in the current
Base-Page contained \$78 and \$56, respectively,
then

\begin{screenoutput}
LDA ($12),Z
\end{screenoutput}

would read the contents of memory location \$567C (i.e., \$5678 + Z),
and store the result in the A register. This instruction would be encoded as \$B2 \$12.

That is, it is equivalent to the Base-Page Indirect Y-Indexed Mode, but using the Z register instead
of the Y register to calculate the offset.

See the note on page \pageref{Base-Page (Zero-Page) Mode} for more information about Base-Page and Zero-Page.

\subsection{Base-Page Quad Indirect Z-Indexed Mode}

This addressing mode is identical to the Base-Page Indirect Z-Indexed Mode, except that
32-bits of data are operated on. The encoding for this addressing mode is identical to
Base-Page Mode, except that it is prefixed by \$42, \$42.
For example, if the Z Register contains the value \$04, and the memory locations \$20, and \$21 in the current
Base-Page contained \$AB and \$CD, respectively,
then

\begin{screenoutput}
LDQ ($12),Z
\end{screenoutput}

would read the contents of memory location \$ABD1 (i.e., \$ABCD + Y) -- \$ABD4
and store the result in the Q Pseudo Register. This instruction would be encoded as \$42 \$42 \$B2 \$12.

Currently the only instruction that offers this mode is LDQ. 

See the note on page \pageref{Base-Page (Zero-Page) Mode} for more information about Base-Page and Zero-Page.
See the note on page \pageref{Base-Page (Zero-Page) Quad Mode} for more information on Quad Mode instructions.

\subsection{32-bit Base-Page Indirect Z-Indexed Mode}
\label{32-bit Base-Page (Zero-Page) Indirect Z-Indexed Mode}

This mode is formed by preceding a Base-Page Indirect Z-Indexed Mode instruction with
the {NOP} instruction (opcode \$EA).  This causes the 45GS02 to read a 32-bit address instead
of a 16-bit address from the Base-Page address indicated by its operand.  The Z index is added
to that pointer.  Importantly, the 32-bit address does not refer to the processor's current 64KB
view of memory, but rather to the 45GS02's true 28-bit address space. This allows easy access
to any memory, without requiring the use of complex bank-switching or DMA operations.

For example, if addresses \$12 to \$15 contained the bytes \$20, \$30, \$FD, \$0F, representing the 32-bit address \$FFD3020, i.e., the VIC-IV border colour register's natural address, and the
Z index contained the value \$01, the following instruction sequence would change the screen
colour to blue, because the screen colour register is at \$FFD3021, i.e., \$FFD3020 + Z:

\begin{screenoutput}
LDA #$06
LDZ #$01
STA [$12],Z
\end{screenoutput}

See the note on page \pageref{Base-Page (Zero-Page) Mode} for more information about Base-Page and Zero-Page.


\subsection{32-bit Base-Page (Zero-Page) Indirect Quad Z-Indexed Mode}

This addressing mode is identical to the 32-bit Base-Page Indirect Z-Indexed Mode,
except that it operates on 32-bits of data at the 32-bit address formed by the argument,
in comparison to 32-bit Base-Page Indirect Z-Indexed Mode which operates on only 8 bits
of data.   The encoding of this addressing mode is \$42, \$42, \$EA, followed by the
natural 6502 opcode for the instruction being performed.

It is also important to note that most of the time Z is actually {\em not} added to the
pointer, as it is part of the Q Pseudo Register. Only \screentext{LDQ (\$nn),Z} will add Z.
For example,

\begin{screenoutput}
LDQ [$12],Z
\end{screenoutput}

would read the memory locations at \$12 + Z through \$15 + Z from the Base-Page, and use those
values to form the 32-bit address from which to load the Q Pseudo Register. This instruction would be
encoded as \$42 \$42 \$EA \$B2 \$12.

See the note on page \pageref{Base-Page (Zero-Page) Mode} for more information about Base-Page and Zero-Page.
See the note on page \pageref{Base-Page (Zero-Page) Quad Mode} for more information on Quad Mode instructions.
See the note on page \pageref{32-bit Base-Page (Zero-Page) Indirect Z-Indexed Mode} for more information on 32-bit Base-Page Indirect addressing.

\subsection{Stack Relative Indirect, Y-Indexed}

This addressing mode is similar to Base-Page Indirect Y-Indexed Mode,
except that instead of providing the address of the pointer in the
Base-Page, the operand indicates the offset in the stack to find the
pointer. This addressing mode effectively de-references a pointer that
has been placed on the stack, e.g., as part of a function call from a
high-level language.  It is encoded identically to the Base-Page Mode.

For example,

\begin{screenoutput}
LDA ($12,SP),Y
\end{screenoutput}

This would use the contents of memory at the current stack pointer plus \$12 to compute the address of the pointer.
This pointer would then have the Y-Register value added to it to obtain the final address to read from, and to
store the value into the Accumulator. The instruction byte stream for this instruction would be \$E2 \$12.

If the Stack Pointer currently pointed to \$01E0, then the pointer address would
be read from addresses \$1F2 and \$1F3, i.e., the two bytes at \$1E0 + \$12 and \$1E0 + \$13.  If locations
\$1F2 and \$1F3 contained \$78 and \$56 respectively, and the Y-Register contained the value \$34, then
the final memory location that would be read would be \$5678 + \$34 = \$56AC, and the contents of that memory
location would be read into the Accumulator.

\subsection{Relative Addressing Mode}

In this addressing mode, the operand is an 8-bit signed offset to the
current value of the Program Counter (PC). It is used to allow branches
to encode the nearby address at which execution should proceed if the
branch is taken.

For example,

\begin{screenoutput}
BNE $2003
\end{screenoutput}

would jump to \$2003, if the Z flag of the processor was not set.  If this instruction were located at
address \$2000, it would be encoded as \$D0 \$01, i.e., branching to +1 bytes after the PC.  Branch
offsets greater than \$7F branch backwards, with \$FD branching to the byte immediately preceeding the branch
instruction, and lower values branching progressively further back.  In this way, a branch can effectively
be made between -125 and +127 bytes from the opcode byte of the branch instruction.  For longer branches,
the 45GS02 supports Relative Word Addressing Mode, where the offset is encoded using 2 bytes instead of 1.

\subsection{Relative Word Addressing Mode}

This addressing mode is identical to Relative Addressing Mode, except that
the address offset is a 16-bit value. This allows a relative branch or jump
to any location in the current 64KB memory view.  This makes it possible
to write software that is fully relocatable, by avoiding the need for absolute
addresses when calling routines.

For example,

\begin{screenoutput}
BNE $3000
\end{screenoutput}

would jump to \$3000 if the Z flag of the process was not set. If this instruction were located at
address \$2002, it would be encoded as \$D3 \$FC \$0F, i.e., branching to +\$FFC = 4,092 bytes following
the second byte of the instruction.  The fact that the instruction is 3 bytes long is ignored in this calculation.

\clearpage
\section{6502 Instruction Set}

NOTE: The mechanisms for switching from 4510 to 6502 CPU personality
have yet to be finalised.

\subsection{Official And Unintended Instructions}

The 6502 opcode matrix has a size of 16 x 16 = 256 possible opcodes.
Those, that are officially documented, form the set of the
\textcolor{blue}{legal} instructions.
All instructions of this legal set are headed by a blue coloured mnemonic.

The remaining opcodes form the set of the
\textcolor{red}{unintended} instructions
(sometimes called "illegal" instructions).
For the sake of completeness these are documented too.
All instructions of the unintended set are headed by a red coloured mnemonic.

NOTE: The unintended instructions are currently unimplemented, and are guaranteed
not to produce exactly the same results as on other
CPU's of the 65xx family. Many of these instructions are known to
be unstable, even running on old hardware.

\subsection{Opcode Table}

The Opcode Table lists all possible opcodes, their size, cycles and addressing mode
in a conscise format.

A cell with a light red background signifies an \textcolor{red}{unintended} instruction.

\begin{center}
\begin{tabular}{c|c|c|}
  \cline{2-3}
  & \$x0 & \$x1 \\\hline
  \multicolumn{1}{|c|}{\$0x} & \OPC{OPC}{mode}{size}{cyc} & \OPill\OPC{OPC}{mode}{size}{cyc} \\\hline
\end{tabular}
\end{center}

The letters attached to the cycle count have the following meaning:

\begin{center}
  \begin{tabular}{|p{2em}|l|}
  \cline{1-2}
  & {\bf Meaning} \\\hline
\multicolumn{1}{|c|}{$b$} & Add one cycle if branch is taken. \\
                          & Add one more cycle if branch taken crosses a page boundary. \\\hline
\multicolumn{1}{|c|}{$p$} & Add one cycle if indexing crosses a page boundary. \\\hline
  \end{tabular}
\end{center}

\input{opcodetable-6502}

\addtocontents{toc}{\protect\setcounter{tocdepth}{1}}
\input{instructionset-6502}
\addtocontents{toc}{\protect\setcounter{tocdepth}{5}}

\clearpage
\section{4510 Instruction Set}

\subsection{Instruction Timing}

Note that the number of cycles depends on the speed setting of the
processor: Some instructions take more or fewer cycles when the
processor is running at full-speed, or a C65 compatibility 3.5MHz speed,
or at C64 compatibility 1MHz/2MHz speed.  More detailed information on
this is listed under each each instruction's information, but the high-level
view is:

\begin{itemize}
\item When the processor is running at 1MHz, all instructions take at least
  two cycles, and dummy cycles are re-inserted into Read-Modify-Write instructions,
  so that all instructions take exactly the same number of cycles as on a 6502.
\item The Read-Modify-Write instructions and all instructions that read a value from
  memory all require an extra cycle when operating at full speed, to allow signals
  to propagate within the processor.
\item The Read-Modify-Write instructions require an additional cycle if the operand
  is \$D019, as the dummy write is performed in this case.
  This is to improve compatibility with C64 software that frequently uses this
  ``bug'' of the 6502 to more rapidly acknowledge VIC-II interrupts.
\item Page-crossing and branch-taking penalties do not apply when the processor is
  running at full speed.
\item Many instructions require fewer cycles when the processor is running at full
  speed, as generally most non-bus cycles are removed. For example, Pushing and Pulling
  values to and from the stack requires only 2 cycles, instead of the 4 that that the
  6502 requires for these instructions.
\end{itemize}

\subsection{Opcode Table}

The coloured cells indicate an extended 45GS02 Opcode. A Q pseudo register opcode is
marked blue, a base-page indirect Z indexed opcode that can use 32-bit pointers is cyan.

\begin{center}
  \begin{tabular}{c|c|c|c|}
    \cline{2-4}
    & \$x0 & \$x1 & \$x2 \\\hline
    \multicolumn{1}{|c|}{\$0x} & \OPC{OPC}{mode}{size}{cyc} & \OPquad\OPCQ{QOP}{mode}{size}{cyc} & \OPfarq\OPCQ{FARQ}{IbpZ}{size}{cyc} \\\hline
  \end{tabular}
\end{center}  

The letters attached to the cycle count have the following meaning:

\begin{center}
  \begin{tabular}{|p{2em}|l|}
  \cline{1-2}
    & {\bf Meaning} \\\hline
\multicolumn{1}{|c|}{b} & Add one cycle if branch is taken. \\
    & Add one more cycle if branch taken crosses a page boundary. \\\hline
\multicolumn{1}{|c|}{d} & Subtract one cycle when CPU is at 3.5MHz.  \\\hline
\multicolumn{1}{|c|}{i} & Add one cycle if clock speed is at 40 MHz. \\\hline
\multicolumn{1}{|c|}{m} & Subtract non-bus cycles when at 40MHz.  \\\hline
\multicolumn{1}{|c|}{p} & Add one cycle if indexing crosses a page boundary. \\\hline
\multicolumn{1}{|c|}{r} & Add one cycle if clock speed is at 40 MHz. \\\hline
\multicolumn{1}{|c|}{s} & Instruction requires 2 cycles when CPU is run at 1MHz or 2MHz. \\\hline
  \end{tabular}
\end{center}

\input{opcodetable-4510}

\addtocontents{toc}{\protect\setcounter{tocdepth}{1}}
\input{instructionset-4510}
\addtocontents{toc}{\protect\setcounter{tocdepth}{5}}

\section{45GS02 Compound Instructions}

As the 4510 has no unallocated opcodes, the 45GS02 uses compound instructions
to implement its extension.  These compound instructions consist of one or
more single byte instructions placed immediately before a conventional
instruction.  These prefixes instruct the 45GS02 to treat the following instruction
differently, as described in \bookvref{cha:cpu}.

You can find them highlighted in the 4510 Opcode Table (see \bookvref{sec:opctable4510})

% Opcodes can have multiple meanings, so no table

% No instruction timing table for compounds, as it can't
% be correctly drawn.

% Same goes for mode list.

\addtocontents{toc}{\protect\setcounter{tocdepth}{1}}
\input{instructionset-45GS02}
\addtocontents{toc}{\protect\setcounter{tocdepth}{5}}

\endgroup
