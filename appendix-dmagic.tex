\chapter{F018-Compatible Direct Memory Access (DMA) Controller}
\label{cha:dmagic}

The MEGA65 includes an F018/F018A backward-compatible DMA controller.
Unlike in the C65, where the DMA controller exists as a separate
chip, it is part of the 45GS02 processor in the MEGA65.  However, as the
use of the DMA controller is a logically separate topic, it is documented
separately in this appendix.
 
The MEGA65's DMA controller provides several important improvements over the
F018/F018A DMAgic chips of the C65:
 
\begin{itemize}
\item{\bf Speed} The MEGA65 performs DMA operations at 40MHz, allowing filling 40MiB or copying 20MiB
  per second.  For example, it is possible to copy a complete 8KiB C64-style bitmap display in
  about 200 micro-seconds, equivalent to less than four raster lines!
 \item{\bf Large Memory Access} The MEGA65's DMA controller allows access to all 256MiB of address space.
\item{\bf Texture Copying Support} The MEGA65's DMA controller can do fractional address calculations
  to support hardware texture scaling, as well as address striding, to make it possible in principle
  to simultaneously scale-and-draw a texture from memory to the screen. This would be useful, should
  anyone be crazy enough to try to implement a Wolfenstein or Doom style-game on the MEGA65.
\item{\bf Transparency/Mask Value Support} The MEGA65's DMA controller can be told to ignore a special value
   when copying memory, leaving the destination memory contents unchanged. This allows masking of transparent
   regions when performing a DMA copy, which considerably simplifies blitting of graphics shapes.
\i tem{\bf Per-Job Option List} A number of options can be configured for each job in a chained list of DMA
  jobs, for example, selecting F018 or F018B mode, changing the transparency value, fractional address stepping
  or the source or destination memory region.
\end{itemize}


\section{F018 ``DMAgic'' DMA Controller}

\input{regtable_DMA.C65}

\section{MEGA65 DMA Controller Extensions}

\input{regtable_DMA.MEGA65}

