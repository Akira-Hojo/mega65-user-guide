\chapter{F018-Compatible Direct Memory Access (DMA) Controller}
\label{cha:dmagic}

The MEGA65 includes an F018/F018A backward-compatible DMA controller.
Unlike in the C65, where the DMA controller exists as a separate
chip, it is part of the 45GS02 processor in the MEGA65.  However, as the
use of the DMA controller is a logically separate topic, it is documented
separately in this appendix.
 
The MEGA65's DMA controller provides several important improvements over the
F018/F018A DMAgic chips of the C65:
 
\begin{itemize}
\item{\bf Speed} The MEGA65 performs DMA operations at 40MHz, allowing filling 40MiB or copying 20MiB
  per second.  For example, it is possible to copy a complete 8KiB C64-style bitmap display in
  about 200 micro-seconds, equivalent to less than four raster lines!
 \item{\bf Large Memory Access} The MEGA65's DMA controller allows access to all 256MiB of address space.
\item{\bf Texture Copying Support} The MEGA65's DMA controller can do fractional address calculations
  to support hardware texture scaling, as well as address striding, to make it possible in principle
  to simultaneously scale-and-draw a texture from memory to the screen. This would be useful, should
  anyone be crazy enough to try to implement a Wolfenstein or Doom style-game on the MEGA65.
\item{\bf Transparency/Mask Value Support} The MEGA65's DMA controller can be told to ignore a special value
   when copying memory, leaving the destination memory contents unchanged. This allows masking of transparent
   regions when performing a DMA copy, which considerably simplifies blitting of graphics shapes.
\item{\bf Per-Job Option List} A number of options can be configured for each job in a chained list of DMA
  jobs, for example, selecting F018 or F018B mode, changing the transparency value, fractional address stepping
  or the source or destination memory region.

\item{\bf Background Audio DMA}
  The MEGA65 includes background audio DMA capabilities similar to the Amiga\texttm series of computers.
  Key differences are that the MEGA65 can use either 8 or 16 bit samples, supports very high sample rates
  up to approximately 1 MHz, has 256 volume settings per channel, and no inter-channel modulation.

\end{itemize}

\section{Audio DMA}

The MEGA65 includes four channels of DMA-driven audio playback that can be used in place of the direct digital
audio registers at \$D6F8-\$D6FB.  That is, you must select which of these two sources to feed to the audio
cross-bar mixer.  This is selected via the AUDEN signal (\$D711 bit 7), which simultaneously enables the audio DMA
function in the processor, as well as instructing the audio cross-bar mixer to use the audio from this instead
of the \$D6F8-\$D6FB digital audio registers. If you wish to have no
other audio than the audio DMA channels, the audio cross-bar mixer can
be bypassed, and the DMA audio played at full volume by setting the
NOMIX signal (\$D711 bit 4).  In that mode no audio from the SIDs, FM,
microphones or other sources will be available.
All other bits in
\$D711 should ordinarily be left clear, i.e., write \$80 to \$D711 to
enable audio DMA.

Two channels form the left digital audio channel, and the other two
channels form the right digital audio channel. It is these left and
right channels that are then fed into the MEGA65's audio cross-bar
mixer.   

As the DMA controller is part of the processor of the MEGA65, and the MEGA65 does not have reserved bus slots
  for multi-media operations, the MEGA65 uses idle CPU cycles to perform background DMA. This requires that the
  MEGA65 CPU be set to the ``full speed'' mode, i.e., approximately 40MHz.  In this mode, there is a wait-state
  whenever reading an operand from memory.  Thus each instruction that loads a byte from memory will create one
  implicit audio DMA slot.  This is rarely a problem in practice, except if the processor idles in a very tight
  loop.  To ensure that audio continues to play in the background, such loops should include a read instruction,
  such as:

\begin{screenoutput}
loop:   LDA $1234   // Ensure loop has at least one idle cycle for
                    // audio DMA
        JMP loop
\end{screenoutput}

Each of the four DMA channels is configured using a block of 16
registers at \$D720, \$D730, \$D740 and \$D750, respectively.
We will explain the registers for the first channel, channel 0, at
\$D720 -- \$D72F.

\subsection{Sample Address Management}

To play an audio sample you must first supply the start address of the
sample. This is a 24-bit address, and must be in the main chip memory
of the MEGA65. This is done by writing the address into \$D72A --
\$D72C.  This is the address of the first sample value that will be played.
You must then provide the end address of the sample in \$D727 --
\$D728.  But note that this is is only 16 bits. This is because the
MEGA65 compares only the bottom 16 bits of the address when checking
if it has reached the end of a sample.  In practice, this means that
samples cannot be more than 64KB in size.  If the sample contains a
section that should be repeated, then the start address of the
repeating part should be loaded into \$D721 -- \$D723, and the CH0LOOP
bit should be set (\$D720 bit 6).  

You can determine the current sample address at any time by reading
the registers at \$D72A -- \$D72C. But beware: These registers are not
latched, so it is possible that the values may be updated as you read
the registers, unless you stop the channel first by clearing the CH0EN
signal.

\subsection{Sample Playback frequency and Volume}

The MEGA65 controls the playback rate of audio DMA samples by using a
24-bit counter.  Whenever the 24-bit counter overflows, the next
sample is requested. Sample speed control is achieved by setting the
value added to this counter each CPU cycle.  Thus a value of
\$FFFFFF would result in a sample rate of almost 40.5 MHz.  In
practice, sample rates above a few megahertz are not possible, because
there are insufficient idle CPU cycles, and distorted audio will
result.  Even below this, care must be taken to ensure that idle
cycles come sufficiently often and dispersed throughout the
processor's instruction stream to prevent distortion.  At typical
sample rates below 16KHz and using 8 bit samples these effects are
typically negligible for normal instruction streams, and so no special
action is normally required for typical audio playback.

At the other end of the scale, sample rates as low as 40.5MHz/$2^{24}$
= 2.4 samples per second are possible.  This is sufficiently low
enough for even the most demanding infra-sound applications.

Volume is controlled by setting \$D629.  Maximum volume is obtained
with the value \$FF, while a value of \$00 will effectively mute the
channel.  The first two audio channels are normally allocated to the
left, and the second two to the right.  However, the MEGA65 includes
separate volume controls for the opposite channels. For example, to
play audio DMA channel 0 at full volume on both left and right sides
of the audio output, set both \$D729 and \$D71C to \$FF.  This allows
panning of the four audio DMA channels.

Both the frequency and volume can be freely adjusted while a sample is playing
to produce various effects.

\subsection{Pure Sine Wave}

Where it is necessary to produce a stable sine wave, especially at
higher frequencies, 
there is a special mode to support this. By setting the CH0SINE
signal, the audio channel will play a 32-byte 16-bit sine wave
pattern.  The sample addresses still need to be set, as though the
sine wave table were located in the bottom 64 bytes of memory, as the
normal address generation logic is used in this mode. However, no
audio DMA fetches are performed when a channel is in this mode, thus
avoiding all sources of distortion due to irregular spacing of idle
cycles in the processor's instruction stream.

This can be used to produce sine waves in both the audible range, as
well as well into the ultrasonic range, at frequencies exceeding
60,000Hz, provided that the MEGA65 is connected to an appropriately
speaker arrangement. 

\subsection{Sample playback control}

To begin a channel playing a sample, set the CH0EN signal (\$D720 bit
7). The sample will play until its completion, unless the CH0LOOP
signal has also been set. When a sample completes playing, the CH0STP
flag will be set.  The audio DMA subsystem cannot presently generate
interrupts.

Unlike on the Amiga\texttm, the MEGA65 audio DMA system supports both
8 and 16-bit samples.  It also supports packed 4-bit samples, playing
either the lower or upper nybl of each sample byte.  This allows two
separate samples to occupy the same byte, thus effectively halving the
amount of space required to store two equal length samples.

\section{MEGA65 Enhanced DMA Jobs}

The MEGA65's implementation of the DMAgic supports significantly
enhanced DMA jobs.  An enhanced DMA job is indicated by writing the
low-byte of the DMA list address to \$D705 instead of to \$D700.  The
MEGA65 will then look for one or more {\em job option tokens} at the
start of the DMA list.  Those tokens will be interpretted, before
executing the DMA job which immediately follows the {\em end of job
  options} token (\$00).  Job option tokens which take an argument have the
most-significant bit set, and always take a 1 byte option.  Job option
tokens that take no argument have the most-significant-bit clear.
Unsupported job option tokens are simply ignored.
This allows for future revisions of the DMAgic to add support for
additional options, without breaking backward compatibility.

The list of valid job option tokens is:

\begin{tabular}{|c|l|}
  \hline
  \$00 & End of job option list \\
  \$06 & Enable use of transparent value \\
  \$07 & Disable use of transparent value \\
  \$0A & Use 11-byte F011A DMA list format \\
  \$0B & Use 12-byte F011B DMA list format \\
  \$80 & Source address bits 20 -- 27 \\
  \$81 & Destination address bits 20 -- 27 \\
  \$82 & Source skip rate (256\textsuperscript{ths} of bytes) \\
  \$83 & Source skip rate (whole bytes) \\
  \$84 & Destination skip rate (256\textsuperscript{ths} of bytes) \\
  \$85 & Destination skip rate (whole bytes) \\
  \$86 & Transparent value (bytes with matching value are not written) \\
\end{tabular}




\section{F018 ``DMAgic'' DMA Controller}

\input{regtable_DMA.C65}

\section{MEGA65 DMA Controller Extensions}

\input{regtable_DMA.MEGA65}

\section{Unimplemented Functionality}

The MEGA65's DMAgic does not currently support either memory-swap or
mini-term operations.

Miniterms were intended for bitplane blitting,
which is not required for the MEGA65 which offers greatly advanced
character modes and stepped and fractional DMA address incrementing
which allows efficient texture copying and scaling. Also there exists
no known software which ever used this facility, and it remains
uncertain if it was ever implemented in any revision of the DMAgic
chip used in C65 prototypes. 

The memory-swap
operation is intended to be implemented, but can be worked around in
the meantime by copying the first region to a 3rd region that acts as
a temporary buffer, then copying the 2nd region to the 1st, and the
3rd to the 2nd.
