\chapter{Data Transfer and Debugging Tools}
\label{cha:transfer-and-debug-tools}

The key to effective cross-platform development is having quick and
easy means to deploy and test software on the MEGA65.  This is
especially true while the MEGA65 emulator continues to be developed.
In fact, even once the MEGA65 emulator is complete, it is unlikely
that it will be able to offer full compatibility at full speed,
because the MEGA65 is much more demanding to emulate than the C64.

There are a variety of tools that can be used for data transfer and
debugging.  These typically function using either the MEGA65's serial
monitor interface, or via the MEGA65's fast ethernet adapter.  The
serial monitor interface is available via the UART lines on the JB1
header.

If you do not have access to the serial monitor interface, there are
tools being developed for the fast ethernet port that provide some,
but not all, of the capabilities of the serial monitor
interface. These will be documented as they become available. The
remainder of this chapter focusses on methods that access the serial
monitor interface.

You can either connect a 3.3V UART adaptor to the appropriate
lines, or more conveniently, connect a TE-0790-03 JTAG debug module
onto this connector.  This gives you a USB connection that can be used
for injecting software, remote debugging and memory inspection, as
well as activating or flashing bitstreams.  With this connection,
there are the following tools:

\section{m65 command line tool}

The \url{https://github.com/mega65/mega65-tools} repository contains a
number of tools, utilities and example programs. These tools are mainly for Linux but can be used on Windows with Cygwin. One of those is
the \stw{m65} command line tool. This is rather a swiss-army
knife collection of utilities in one.  Common useful functions
include:

\subsection{Screenshots using m65 tool}

To take a screenshot of the MEGA65 use:

\begin{screenoutput}
m65 -S
\end{screenoutput}

This will create a file called \stw{mega65-screeen-000000.png},
or if that file already exists, the first non-used number will be used
in place of \stw{000000}.

Note that this screenshot function works by having \stw{m65} emulate the
function of the VIC-IV. Thus while it produces excellent looking
digital screenshots, it may not exactly match the real display of the
MEGA65.  At the time of writing it does not render sprites or
bitplanes, only text and bitmap-based video modes.

\subsection{Load and run a program on the MEGA65}

To load and run a program on the MEGA65, you can use a command like:

\begin{screenoutput}
m65 -F -4 -r foo.prg
\end{screenoutput}

The \stw{-F} option tells \stw{m65} to reset the MEGA65
before loading the program.

The \stw{-4} option tells \stw{m65} to switch the MEGA65
to C64 mode before loading the program. If this is left off, then it
will attempt to load the program in C65 mode.

The \stw{-r} option tells \stw{m65} to run the program
immediately after loading.

Note that this command works using the normal BASIC LOAD command, and
is thus limited to loading programs into the lower 64KB of RAM

\subsection{Reconfigure the FPGA to run a different bitstream}

To try out a different MEGA65 bitstream, a command like the following can be
used:

\begin{screenoutput}
m65 -b bitstream.bit
\end{screenoutput}

This will cause the named bitstream to be sent to the FPGA.  As the
FPGA will be reconfigured by this action, and program currently
running will not merely be stopped, but also main memory will be
cleared. For models of the MEGA65 that are fitted with 8MB or 16MB of
expansion memory, those expansion memories are implemented in external
chips, and so the contents of them will not be erased.

For non-MEGA65 bitstreams (such as zxunomega65 and gbc4mega65), use the '-q' argument instead:

\begin{screenoutput}
m65 -q bitstream.bit
\end{screenoutput}

\subsection{Remote keyboard entry}

The MEGA65's keyboard interface logic supports the injection of
synthetic key events using the registers \$D615 -- \$D617.
The \stw{m65} utility uses this to allow remote typing on the MEGA65
in a way that is transparent to software.  There are three ways to use
this:

\begin{screenoutput}
m65 -t sometext
\end{screenoutput}

This form types the supplied text, in this case {\em sometext}, but
does not simulate pressing the \specialkey{RETURN} key.  If you wish
to simulate the pressing of the \specialkey{RETURN} key, use \stw{-T}
instead of \stw{-t}, e.g.:

\begin{screenoutput}
m65 -T list
\end{screenoutput}

This would cause the LIST command to be typed and executed.

Finally, it is possible to begin general remote keyboard control via:

\begin{screenoutput}
m65 -t -
\end{screenoutput}

In this mode, any key pressed on the keyboard of the computer
where \stw{m65} is running will be relayed to the MEGA65.  Note that
not all special keys are supported, and that there is some latency, so
using key repeat can cause unexpected results.  But for general remote
control, it is a very helpful facility.

\subsection{Unit testing and logging support}

The \stw{m65} tool includes support to facilitate remote unit testing 
directly on MEGA65 hardware. When \textit{unit testing mode} is active, 
\stw{m65} waits for the MEGA65 to send certain byte sequences over the 
serial interface which signal the current state (started, passed, failed) 
of a given test. Additionally, it is possible to send log messages from 
the MEGA65 to the host computer.

Unit testing mode is entered by calling \stw{m65} with the \stw{-u} flag. 
To run a remote BASIC program in C65 mode and simultaneously put \stw{m65}
into \textit{unit testing mode}, the following command can be used:

\begin{screenoutput}
    m65 -Fur attic-ram.prg -w tests.log
\end{screenoutput}

The \stw{-F} and \stw{-r} options tell \stw{m65} to reset the MEGA65
before loading the program "attic-ram.prg" and then automatically run it. 
The \stw{-u} option then tells \stw{m65} go into \textit{unit testing mode} 
instead of exiting after launching the program.
The optional \stw{-w} option makes \stw{m65} append the test results to the 
file "test.log" (creating the file if it doesn't exist). 

Please note that \stw{m65} automatically exits from \textit{unit testing mode}
if no test state signals were received for over 10 seconds.

Support is provided for sending unit test signals to the host computer from
C and BASIC 65 programs:

\subsubsection{Using unit tests with C}

The MEGA65 libc contains support for unit testing via functions defined in 
\stw{tests.h} and \stw{tests.c}. 

To signal the start of a test, include \stw{tests.h} and use

\begin{verbatim}
    unit_test_setup("testName",issueNumber);
\end{verbatim}

where \textit{"testName"} is a human-readable name of the test (e.g. 
"VIC-II") and \textit{issueNumber} a reference to the corresponding
bug issue (for example, the issue number from github).

After starting a test, it's possible to signal passed tests with 
the unit\_test\_ok() function:

\begin{verbatim}
    unit_test_ok();
\end{verbatim}

A failed test is signalled with unit\_test\_fail():

\begin{verbatim}
    unit_test_fail("fail message");
\end{verbatim}

Each time the unit\_test\_ok() or unit\_test\_fail() 
functions are called, the \textit{sub issue} of the test (reported on 
the host computer) is incremented. This makes it easier to combine and 
identify multiple tests in one file.

You can send arbitrary log messages via unit\_test\_log():

\begin{verbatim}
    unit_test_log("hello world from mega65!");
\end{verbatim}

...and finally, when all is done, the end of unit testing is signalled
by the use of

\begin{verbatim}
    unit_test_done();
\end{verbatim}


\subsubsection{Using unit tests with BASIC 65}

\stw{b65support.bin} is a machine language module providing support for 
unit testing from BASIC 65, available in the \stw{bin65} folder of the mega65-tools
repository. This module works by redirecting the USR vector to perform the functions 
needed to communicate with the testing host.

In an automated test scenario, you may want to inject the \stw{b65support.bin} 
binary into MEGA65 RAM by using \stw{m65}:

    \begin{screenoutput}
        m65 -@ mega65-tools/bin65/b65support.bin@15fe
    \end{screenoutput}


Of course it's also possible to load \stw{b65support.bin} directly from the
MEGA65 by mounting the \stw{M65UTILS.D81} image from the freezer and issuing

\begin{screenoutput}
    BLOAD "B65SUPPORT.BIN"
\end{screenoutput}

After loading, \stw{b65support.bin} is initialized with

\begin{screenoutput}
    SYS $1600
\end{screenoutput}

Once initialized, the following functions are provided by \stw{b65support.bin}:

\begin{screenoutput}
    A=USR(<issueNum>)
\end{screenoutput}
prepares a new test with number <issueNum> and resets subissue number to 0

\begin{screenoutput}
    A=USR("=<testName>")
\end{screenoutput}
sets test name and sends test start signal; for example: \stw{A=USR("=VIC-III")} 
sets the test name to 'VIC-III' and signals the host computer that the test 
has started.

\begin{screenoutput}
    A=USR("/<logMessage>)
\end{screenoutput}
sends a log message to the host computer

\begin{screenoutput}
    A=USR("P")
\end{screenoutput}
sends the 'passed' signal to the host computer and increases the sub issue number

\begin{screenoutput}
    A=USR("F")
\end{screenoutput}
sends the 'test failed' signal to the host computer and increases 
the sub issue number

\begin{screenoutput}
    A=USR("D")
\end{screenoutput}
sends the 'test done' signal to the host computer 

All calls return the current sub issue number or \stw{?ILLEGAL QUANTITY ERROR} in 
case of calling an invalid command.

\subsubsection{BASIC 65 example}

The following is a complete BASIC 65 example showing how to use \stw{m65}'s
unit testing features:

\begin{screenoutput}
    100 rem attic ram cache test
    110 poke $bfffff2,$e0         : rem enable attic ram cache
    120 sys $1600                 : rem init test module
    130 a=usr(379)                : rem set issue number
    140 a=usr("=attic-ram-cache") : rem set test name
    150 bank128:poke0,65          : rem just to be sure
    160 b0=$8000000 : b1=$8000100 : rem attic ram areas to be tested
    170 for r=0 to $ff
    180   poke b0+r,0             : rem fill area 1 with 0
    190   poke b1+r,$ff           : rem fill area 2 with $ff
    200 next r
    210 for t=1 to 10             : rem 10 tries
    220   poke b0,32              : rem write to b0
    230   for x=0 to $ff:t1=b1+x
    240     a=peek(b0)            : rem read from b0
    250     b=peek(t1):b=peek(t1) : rem read twice from t1
    260     ifb<>255 thenf=t:t=11:x=256   : rem this shouldn't happen
    270   next x
    280 next t
    290 if f=0 then begin
    300   print "no faults detected after";t;"tries."
    310   a=usr("p")              : rem signal 'test passed' to host
    320 bend : else begin
    330   a=usr("f")              : rem signal 'test failed' to host
    340   print "hyper ram fault detected after";f;"tries."
    350   print "peek($";hex$(t1);") [t1] is";b;"but should be 255"
    360 bend
    370 a=usr("d")                : rem test done    
\end{screenoutput}

\section{M65Connect}

This is a cross-platform graphical tool available for Windows, Linux
and MacOSX, which allows access to most of the functions of
the \stw{m65} command-line tool, without needing to use a
command line, or being able to compile the tool for your preferred
operating system.

The repository for M65Connect is: \url{https://github.com/MEGA65/m65connect}

The latest binary version is available from \url{https://files.mega65.org}.

With the MEGA65 or Nexys FPGA switched off, connect a USB cable from your computer to the MEGA65 or Nexys FPGA board. Run the \textit{M65Connect} executable and follow the prompts to connect. The program will help you identify which USB Serial Port to communicate over.

With this tool you can easily transfer PRG programs and a variety of other files. M65Connect can handle the transfer, switching to C64 mode, and execution of programs.

\section{mega65\_ftp}

The \stw{mega65}\_\stw{ftp} utility from
the \url{https://github.com/mega65/mega65-tools} repository is a
little misleadingly named: While it 
is a File Transfer Program, it does not use the File Transfer
Protocol (FTP).  Rather, it uses the serial monitor interface to take
remote control of a MEGA65, and directly access its SD card to enable
copying of files between the MEGA65 and the host computer.

Note that it does not perfectly restore the MEGA65's state on exit,
and thus should only be used when the MEGA65 is at the READY prompt,
so that any running software doesn't go haywire. In particular, you
should avoid using it when a sensitive program is running, such as
the Freeze Menu, MEGA65 Configuration Utility, or the MEGA65
Format/FDISK utility.  (This problem could be solved with a little
effort, if someone has the time and interest to fix it).

When run, it provides an FTP-like interface that supports
the \stw{get}, \stw{put}, \stw{rename} and \stw{dir} commands.
Note that when putting a file, you should make sure that it is given a
name that is all capitals and has o DOS-compatible 8.3 character file
name.  This is due to limitations in both \stw{mega65}\_\stw{ftp} and the
MEGA65's Hypervisor's VFAT32 file system code. Again, these problems
could be fixed with a modest amount of effort on the part of a
motivated member of the community.

Finally, the \stw{mega65}\_\stw{ftp} program is {\em very} slow to push
new files to the MEGA65, typically yielding speeds of around 5KB/sec.
This is partly because the serial monitor interface is capable of
transferring data at only 40KB/sec (when set to 4,000,000 bits per
second), and partly because the remote control process results in a
lot of round-trips where helper routines are executed on the MEGA65 to
read, write and verify sectors on the SD card.  It would be quite
feasible to improve this to reach close to 40KB/sec, and potentially
faster using either some combination of data compression,
de-duplication of identical sectors (especially when uploading disk
images) and other techniques. Again, this would be a very welcome
contribution that someone in the community could contribute to
everyone's benefit.

\section{TFTP Server}

Work on a true TFTP server for the MEGA65 that supports fast
TFTP transfers over the 100mbit ethernet has begun, and can be used to
very quickly read files from the MEGA65. Speeds of close to 1MB/sec
are possible, depending on SD card performance.  Rather than using
DHCP, this utility will respond to {\em any} IP address that ends in
.65. It always uses the MAC address 40:40:40:40:40:40. True DHCP
support as well as using the MEGA65's configured ethernet MAC address
may be added in the future. 

More importantly, support for writing
files to the SD card is not yet complete, and is blocked by the need
for the implementation of the necessary functions in the MEGA65's Hypervisor for creating and
growing files.  A particular challenge is enabling the creation of
files with contiguous clusters as is required for D81 disk images: If
a D81 file is fragmented, then it cannot be mounted, because the
mounting mechanism requires a pointer to the contiguous block of the
SD card containing the disk image.
In the interim, \stw{mega65}\_\stw{ftp} can be used as a substitute.

\section{Converting a BASIC text file listing into a PRG file}

If you have a untokenised BASIC program in plain text format sourced from somewhere like an internet post, and you wish to try it on the MEGA65 without typing it in, it is possible to convert it to a PRG.

\textit{C64List} is a Windows-based command-line tool that will allow you to make the conversion. Once you have a \textit{.PRG} file, you can use a tool like M65Connect to upload it to the MEGA65 or Nexys FPGA. 

C64List is available for download from \url{http://www.commodoreserver.com/Downloads.asp}

Ensure you have a program listing saved to a file on your local computer (for example, \textit{program.txt}) encoded as ANSI or UTF8.

Use C64List to convert the file to a PRG file using:
    
\begin{tcolorbox}[colback=black,coltext=white]
    \begin{verbatim}
        C64List program.txt -prg
    \end{verbatim}
\end{tcolorbox}

Now you can upload your newly converted program to the MEGA65 with M65Connect or one of the other tools described previously.

It is worth noting that this method will not be 100\% effective on listings with special PETSCII characters. Programs with PETSCII will require some editing on the MEGA65 itself before saving to disk.
