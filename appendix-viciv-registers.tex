\chapter{VIC-IV Video Interface Controller}
\label{cha:viciv}

\section{Features}
The VIC-IV is a fourth generation Video Interface Controller developed
especially for the MEGA65, and featuring very good backwards compatibility
with the VIC-II that was used in the C64, and the VIC-III that
was used in the C65.  The VIC-IV can be programmed as though it were either
of those predecessor systems.  In addition it supports a number of new
features. It is easy to mix older VIC-II/III features with the new VIC-IV
features, making it easy to transition from the VIC-II or VIC-III to the VIC-IV,
just as the VIC-III made it easy to transition from the VIC-II.  Some of the new
features and enhancements of the VIC-IV include:

\begin{itemize}
\item {\bf Direct access to 384KB RAM} (up from 16KB/64KB with the VIC-II and 128KB
  with the VIC-IV).
\item Support for {\bf 32KB of 8-bit Colour/Attribute RAM} (up from 2KB on the VIC-III), to
  support very large screens.
\item {\bf HDTV 720$\times$576 / 800$\times$600 native resolution} at both 50Hz and 60Hz for {\bf PAL and NTSC}, with {\bf VGA and digital video} output.
\item {\bf 81MHz pixel clock} (up from ~8MHz with the VIC-II/III), which enables a wide range of new features.
\item New 16-colour (16$\times$8 pixels per character cell) and 256-colour (8$\times$8 pixels per character cell) {\bf full-colour character modes}.
\item Support for upto {\bf 8,192 unique characters in a character set}.
\item {\bf Four 256-colour palette banks} (versus the VIC-III's single palette bank), each supporting {\bf 23-bit colour depth} (versus the VIC-III's 12-bit colour depth), and which can be rapidly alternated to create even more colourful graphics than is possible with the VIC-III.
\item Screen, bitmap, colour and character data can be positioned at any {\bf address with byte-level granularity} (compared with fixed 1KB -- 16KB boundaries with the VIC-II/III)
\item {\bf Virtual screen dimensioning}, which combined with byte-level data position granularity provides effective {\bf hardware support for scrolling and panning in both X and Y directions}.
\item {\bf New sprite modes}: Bitplane modification, {\bf full-colour} (15 foreground colours + transparency) and tiled modes, allowing a wide variety of new and exciting sprite-based effects
  \item The ability to stack sprites in a bit-planar manner to produce {\bf sprites with upto 256 colours}.
n\item Sprites can use 64 bits of data per raster line, allowing {\bf sprites to be 64 pixels wide} when using VIC-II/III mono/multi-colour mode, or 16 pixels wide when using the new VIC-IV full-colour sprite mode.
\item {\bf Sprite tile mode}, which allows a sprite to be repeated horizontally across an entire raster line, allowing sprites to be used to create  animated backrounds in a memory-efficient manner.
  \item Sprites can be configured to use a {\bf separate 256-colour palette} to that used to draw other text and graphics, allowing for a more colourful display.
  \item {\bf Super-extended attribute mode} which uses two screen RAM bytes and two colour RAM bytes per character mode, which supports a wide variety of new features including {\bf alpha-blending/anti-aliasing}, {\bf hardware kerning/variable-width characters}, hardware horizontal/vertical flipping, alternate palette selection and other powerful features that make it easy to create highly dynamic and colourful displays.
  \item {\bf Raster-Rewrite Buffer} which allows {\bf hardware-generated pseuso-sprites}, similar to ``bobs'' on Amiga(tm) computers, but with the advantage that they are rended in the display pipeline, and thus do not need to be undrawn and redrawn to animate them.
    \item {\bf Multiple 8-bit colour playfields} are also possible using the Raster-Rewrite Buffer. 

      In short, the VIC-IV is a powerful evolution of the VIC-II/III, while retaining the character and disctinctiveness of the VIC-series of
      video controllers.

      For a full description of the additional registers that the VIC-IV provides, as well as documentation of the legacy VIC-II and VIC-III registers, refer to the corresponding sections of this appendix. The remainder of the appendix will focus on describing the capabilitie and use of many of the VIC-IV's new features.
\end{itemize}

\section{VIC-II/III/IV Mode Selection}
Because the new features of the VIC-IV are all extensions to the existing VIC-II/III designs, there is no concept of having to select which mode in which the VIC-IV will operate: It is always in VIC-IV mode. However, for backwards compatibility with softwre, the many additional registers of the VIC-IV can be hidden, so that it appears to be either a VIC-II or VIC-III. This is done in the same manner that the VIC-III uses to hide its new features from legacy VIC-II software.

 The mechanism is the VIC-III write-only KEY register (\$D02F, 53295 decimal).  The VIC-III by default conceals its new features until a ``knock'' sequence is performed.  This consists of writing two special values one after the other to \$D02F.  The following table summarises the kock sequences supported by the VIC-IV, and indicates which are VIC-IV specific, and which are supported by the VIC-III:

\setlength{\tabcolsep}{3pt}
\begin{longtable}{|L{2.4cm}|L{2.4cm}|L{3.5cm}|L{2cm}|}
\hline  
{\bf{First Value Hex (Decimal)}} & {\bf{Second Value Hex (Decimal)}} & {\bf{Effect}} & {\bf{VIC-IV Specific? }} \\
\hline  
\endfirsthead
\multicolumn{3}{l@{}}{\ldots continued}\\
\hline
{\bf{First Value Hex (Decimal)}} & {\bf{Second Value Hex (Decimal)}} & {\bf{Effect}} & {\bf{VIC-IV Specific? }} \\
\endhead
\multicolumn{3}{l@{}}{continued \ldots}\\
 \endfoot
 \hline
\endlastfoot
\small \$00 (0) & \small \$00 (0) & Only VIC-II registers visibiel (all VIC-III and VIC-IV new registers are hidden) & No \\
 \hline 
\small \$A5 (165) & \small \$96 (150) & VIC-III new registers visible & No \\
 \hline  
\small \$47 (71) & \small \$53 (83) & Both VIC-III and VIC-IV new registers visible & Yes \\
 \hline  
\small \$45 (69)  & \small \$54 (84) & No VIC-II/III/IV registers visible. 45E100 ethernet controller buffers are visible instead & Yes \\
 \hline 
   \end{longtable}


 \subsection{Detecting VIC-II/III/IV}
  
 Detecting which generation of the VIC-II/III/IV a machine is fitted with can be important for programs that support only particular generations, or that wish to vary their graphical display based on the capabilities of the machine.  While there are many possibilities for this, the following is a simple and effective method.  It relies on the fact that the VIC-III and VIC-IV do not repeat the VIC-II registers throughout the IO address space.  Thus while \$D000 and \$D100 are synonymous when a VIC-II is present (or a VIC-III/IV is hiding their additional registers), this is not the case when a VIC-III or VIC-IV is making all of its registers visible.  Therefore presence of a VIC-III/IV can be determined by testing whether these two locations are aliases for the same register, or represent separate registers.
 The detection sequence consists of using the KEY register to attempt to make either VIC-IV or VIC-III additional registers visible. If either succeeds, then we can assume that the corresponding generation of VIC is installed. As the VIC-IV supports the VIC-III KEY knocks, we must firt test for the presence of a VIC-IV.  Thus the test can be done in BASIC from either C64 or C65 mode as follows:

\begin{screenoutput}
10 POKE53248,1:POKE53295,71:POKE53295,83
20 POKE53248+256,0:IFPEEK(54248)=1THENPRINT"VIC-IV PRESENT":END
30 POKE53248,1:POKE53295,165:POKE53295,150
40 POKE53248+256,0:IFPEEK(54248)=1THENPRINT"VIC-III PRESENT":END
50 PRINT "VIC-II PRESENT"
\end{screenoutput}
   
As the MEGA65 is the only C64-class computer that is fitted with a VIC-IV, this can be used as a {\em de facto} test for the presence
of a MEGA65 computer. Detection of a VIC-III can be similarily be reasonably assumed to indicate the presence of a C65.

\section{Video Output Formats, Timing and Compatability}

The VIC-IV was designed for use in the MEGA65 and related systems, including the MEGAphone family of portable devices.
The VIC-IV supports both VGA and digital video output, using a connector for the digital video that accepts most cables intended
for connecting HDMI(tm) compatible devices.  It also supports parallel digital video output suitable for driving LCD display
panels.  Considerable care has been taken to create a common video front-end that supports these three output modes.

For simplicity and accuracy of frame timing for legacy software, the video format is normally based on the HDTV PAL and NTSC 720$\times$576/480 (576p and 480p) modes using a 27MHz output pixel clock.  This is ideal for digital video and LCD display panels. However not all VGA displays support
these modes, especially 720$\times$576 at 50Hz.

In terms of VIC-II and VIC-III backwards compatability, this display format has several effects that do not cause problems for most programs, but can cause some differences in behaviour:

\begin{enumerate}
\item Because VIC-IV display is progressive rather than interlaced, two physical raster lines are produced for each logical VIC-II or VIC-III raster line.  This can cause some minor visual artefacts, when programs make assumptions about where on a horizontal line the VIC is drawing when, for example, the border or screen colour is changed.
\item The VIC-IV does not follow the behaviour of the VIC-III, which allowed changes in video modes, e.g., between text and bitmap mode, on characters.  Nor does it follow the VIC-II's policy of having such changes take effect immediately.  Instead, the VIC-IV applies changes at the start of each raster line.  This can cause some minor artefacts.
\item The VIC-IV uses a single-raster rendering buffer which is populated using the VIC-IV's internal 81MHz pixel clock, before being displayed using the 27MHz output pixel clock.  This means that a raster lines display content tends to be rendered much earlier in a raster line than on either the VIC-II or VIC-III.  This can cause some artefacts with displays, particularly in demos that rely on specific behaviour of the VIC-II at particular cycles in a raster line, for example for effects such as VSP or FLI.  At present, such effects are unlikely to display correctly on the current revision of the VIC-IV.  Improved support for these features is planned for a future revision of the VIC-IV.
  \item The 1280$\times$200 and 1280$\times$400 display modes of the VIC-III are not currently supported, as they cannot be meaningfully displayed on any modern monitor, and no software is known to support or use this feature.
\end{enumerate}

\section{Memory Interface}

The VIC-IV supports upto 64KB of colour RAM and, in principle, 16MB of direct access RAM for video data.  However in typical installations
32KB of colour RAM and either 384KB of addressable RAM is present. In MEGA65 systems, the second 128KB of RAM is typically used to hold a C65-compatible ROM, leaving 256KB available, unless software is written to avoid the need to use C65 ROM routines, in which case all 384KB can be used.

The VIC-IV supports all legacy VIC-II and VIC-III methods for accessing this RAM, including the VIC-II's use of 16KB banks, and the VIC-III's Display Address Translator (DAT).  This additional memory can be used for character and bitmap displays, as well as for sprites.  However, the VIC-III bitplane modes remain limited to using only the first 128KB of RAM, as the VIC-IV does not enhance the bitplane mode.

To use the additional memory for screen RAM, the screen RAM start address can be adjusted to any location in memory with byte-level granularity by setting the SCRNPTR registers (\$D060 -- \$D063, 53344 -- 53347 decimal).  For example, to set the screen memory to address $12345:

\begin{screenoutput}
  POKE53295,71:POKE53295,83: REM MAKE VIC-IV REGISTERS VISIBLE
  POKE53344+0,69:POKE53344+1,35:POKE53344+2,1
\end{screenoutput}

The location of the character generator data can also be set with byte-level precision via the CHARPTR registers.
The area of colour RAM being used can be similarly set using the COLPTR registers. That is, the value is an offset from the start of the colour RAM.

The location of the sprite pointers can also be moved, and sprites can be made to have there data anywhere in first 4MB of memory.
This is accomplished by first setting the location of the sprite pointers by setting the SPRPTRADR registers.  This allows the list of
eight sprite pointers to be moved from the end of screen RAM to an arbitary location in the first 8MB of RAM.  To allow sprites themselves
to be located anywhere in the first 4MB of RAM, the SPRPTR16 bit in \$D06E must be set. In this mode, two bytes are used to indicate the
location of each sprite, instead of one.  The location remains a multiple of 64 bytes, thus allowing for upto 65,536 unique sprite images
to be used at any point in time, if the system is equipped with sufficient RAM (4MB or more).  In this mode, the VIC-II 16KB banking is ignored, and the location of sprite data is simply 64 $\times$ the pointer value.  For example, to have the data for a sprite at \$C000 (49152 decimal), this would be sprite location $49,152\divide64 = 768$.  As 768 = 256$\times$3, this would require the two sprite pointer bytes to be 0 and 3.


\section{VIC-II / C64 Registers}

\input{regtable_VIC-II.C64}

\section{VIC-III / C65 Registers}

\input{regtable_VIC-III.C65}

\section{VIC-IV / MEGA65 Specific Registers}

\input{regtable_VIC-IV.MEGA65}
