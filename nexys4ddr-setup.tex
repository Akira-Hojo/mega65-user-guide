\chapter{Using a Nexys4DDR as a MEGA65}

\section{Building your own MEGA65 Compatible Computer}

You can build your own MEGA65-compatible computer by using a Nexys4DDR FPGA development board.
This appendix describes the process to setup a Nexys4DDR board for this purpose.
The older non-DDR Nexys4 board is also supported, and the instructions are the same, except that
you must use a bitstream designed for that board.
Using a Nexys4DDR bitstream on a non-DDR Nexys4 board, or vice versa, may cause irrepairable damage to your board, so make sure
you have the correct bitstream to suit your board.


DISCLAIMER: M.E.G.A cannot take any responsibility for any damage that may occur to your Nexys4DDR board.

\newpage

\section{Power, Jumpers, Switches and Buttons}

The MEGA65 core consumes too much power for the Nexys4DDR board to by powered only from a standard USB port.
In particular, writing to the SD card will hang or do other annoying things.
Therefore you should use a 5V power supply.
Digilent sell a power supply for the Nexys4DDR board, and we recommend you use this to ensure you avoid the risk of damage
to your Nexys4DDR board.

You must then set the following jumpers on your Nexys4DDR board to the following positions:

\begin{itemize}
\item{JP1} - USB/SD
\item{JP2} - SD
\item {JP3} - Wall
\end{itemize}

% XXX - Image of board highlighting the jumpers

All 16 switches on the lower edge of the board must be set to the off position.

The ``CPU RESET'' button will reset the MEGA65 when pressed, while the ``PROG'' button will cause the FPGA itself to reload the MEGA65
core.  The main difference between the two is that CPU RESET is faster, and does not clear the contents of memory, while the FPGA button
is slower, and does reset the contents of memory.

Two of the five buttons in the cross arrangement can also be used:  BTNU acts as though you have pressed the \megakey{RESTORE} key, while BTNC will trigger an IRQ, as though the IRQ line had been pulled to ground.

\section{Keyboard}

Only USB keyboards that lack a USB hub will work with the Nexys4DDR board.  Generally extremely cheap keyboards will work, while more expensive keyboards tend to have a USB hub integrated, and will not work.  You may need to try several keyboards, before you find one that works.

The keyboard layout is positional rather than logical.
This means that keys in similar positions to the keys on a C65 keyboard will have similar function.
This relationship assumes that your USB keyboard uses a US keyboard layout.

The \megakey{RESTORE} key is mapped to the PAGE UP key.

\newpage

\section{Preparing microSDHC card}

The MEGA65 requires an SDHC card of between 4GB and 64GB capacity.  Some SDXC cards may work, however, this is not officially supported.

To prepare your SD card, you will need the following two separate files from the MEGA65 web site:

\begin{itemize}
\item{Bitstream} \url{http://mega65.org/assets/bitstream.7z}
\item{Support Files} \url{http://mega65.org/assets/fileset.7z}
\end{itemize}

In addition, you will also need a C65 ROM.  There were many different versions created during the development of the Commodore 65,
an the MEGA65 can use any of them.  However, we recommend you use 911001.bin, as this has the most complete BASIC and DOS implementations.

The steps are:

\begin{itemize}
  \item{Format the SD card}  in a convenient computer using the FAT32 filesystem.  The MEGA65 and Nexys4DDR boards do not understand other
file systems, especially the exFAT file system.
\item{Unzip} the bitstream.7z file, and copy the file with name ending in ``.bit'' onto the SD card.
\item{Insert} the SD card into the SD card slot on the under-side of the Nexys4DDR board.
\item{Turn on} the Nexys4DDR board.
\item{Enter the Utility Menu} by holding the left and right SHIFT keys down on the USB keyboard you have connected to the Nexys4DDR board.
\item{Enter the FDISK/FORMAT tool} by pressing 2 when the option appears on the MEGA65 boot screen.
\item{Follow the prompts} in the FDISK/FORMAT program to again format the SD card for use by the MEGA65.
  (The FDISK tool will partition your SD card into two partitions and format them.
  One is type \$41 = MEGA65 System Partition, where the save slots, configuration data etc. live.
  (This partition is invisible in i.e. Win PCs).
  and the other partition with type \$0C = VFAT32, where KERNEL,support files, Games, and so on, will be copied to later.
  (This Partition is visible on i.e. Win PCs).
\item{Once formatting is complete} switch off the Nexys4DDR board and remove the microSDHC card from the Nexys Board and put it back into your PC
\item{Again copy} the Bitstream back onto the SD card, as well as all the files from the other 7z file downloaded from the MEGA65 website.
\item{Copy the 911001.bin} ROM file onto the SD card, and rename it to MEGA65.ROM
\item{If you have any .D81 files}, copy them onto the SD card. Make sure that they have names that fit the old DOS 8.3 character limit, and are upper case.  This restriction will be removed in a future release.
\item{Remove the SD card} and reinsert it into your Nexys4DDR board
\item{Power the Nexys4DDR} board back on.  The MEGA65 should boot within 15 seconds.
\end{itemize}


\quote{Congratulations your Mega65 has been set up and is Ready to use !!!}



\section{Useful Tips}

The following are some useful tips for getting familiar with the MEGA65

\begin{itemize}

\item{Press \& hold \megasymbolkey (or the \megakey{C=} if using a Commodore 64 or 65 keyboard) during boot to start up in C64 mode instead of C65 mode}
 \item{Press \& hold \specialkey{RUN STOP} during boot to enter the machine language monitor, instead of starting BASIC.}
\item{Press the \megakey{RESTORE} key for approximately 1/2 - 1 second to enter the MEGA65 Freeze Menu.  From this menu
  you have conveient tools to change the CPU speed, switch between PAL \& NTSC video mode, change Audio settings, manage freeze-states,
   select D81 disk images, examine and modify memory of the frozen program, among other features.  This is in many ways the heart of the MEGA65, so it is well worth exploring and familiarising  yourself with!}
\item{Press the \megakey{RESTORE} for about 2 seconds to reset the MEGA65.  This is a temporary feature that will be removed when the MEGA65 desktop computers with built-in reset button become available.}
\item{Type \screentext{POKE0,65} in C64 mode to switch to the CPU to full speed (40MHz). Some software may behave incorrectly in this mode, while other software will work very well, and run many times faster than on a C64.}
\item{Type \screentext{POKE0,64} in C64 mode to switch the CPU to 1MHz.}
\item{Type \screentext{SYS58552} in C64 mode to switch to C65 mode.}
\item{Type \screentext{GO64} in C65 mode and confirm, by pressing \screentext{Y}, to switch to C64 mode, just like on a C128.}
\item{The C65 ROM makes device 8 the default, so you can normally leave off the ,8 from the end of LOAD and SAVE commands.}
\item{Pressing \megakey{SHIFT} + \specialkey{RUN STOP} from either C64 or C65 mode will attempt to boot from disk.}
\item{Have fun! The MEGA65 has been loving crafted over many years for your enjoyment. We hope you have as much fun using it, as we have had creating it!}
\item{The MEGA Museum of Electronic Games and Art welcomes your feedback, suggestions and contributions to this open-source digital heritage preservation project.}
\end{itemize}
