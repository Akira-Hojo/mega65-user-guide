\chapter{Getting Started in BASIC}
\label{cha:basic-getting-started}

It is possible to code on the MEGA65 in many languages,
however most people start with BASIC.  That makes sense,
because BASIC stands for Beginner's All-purpose Symbolic
Instruction Code: It was made for people like you to get
started with in the world of coding!

A few short words before we dive in: BASIC is a programming
language, and like spoken language it has conventions, grammar
and vocabulary.  Fortunately, it is much quicker and easier
to learn than our complex human languages. But if you pay
attention, you might notice some of these structures, and that
can help you along your path in the world of coding.

If you haven't already read Chapter \ref{cha:getting-started},
it might be a good idea to do so. This will help you be able to
more confidently interact with the MEGA65 computer.

It's also great to remember that if you really confuse the MEGA65,
you can always get back to the READY. prompt by just pressing the
reset button on the left-hand side of the keyboard, or if that
doesn't help, then by turning it off
and on again using the power switch on the left-hand side of the keyboard.
You don't have to worry about shutting the computer
down properly or any of that nonsense.  The only thing to remember
is that if you had any unsaved work, it will be lost when you turn
the computer off and on again or press the reset button.

Finally, if you don't understand all of the descriptions and information
with an example -- don't worry! We have provided as much information
as we can, so that it is there in case you have questions, encounter problems are
just curious to discover more.  Feel free to skip ahead to the examples
and try things out, and then you can go back and re-read it when you are motivated
to find something out, or help you work though a problem.  And if you don't find
the answer to your problem, send us a message!  There are support forums for the
MEGA65 at \url{https://mega65.net}, and you can
report problems with this guide at:

\url{https://github.com/mega65/mega65-user-guide}

We hope you have as much fun learning to programme the MEGA65 as
we have had making it!

\section{Your first BASIC programmes}

The MEGA65 was designed to be programmed! When you turn it on,
it takes a couple of seconds to get its house in order, and then
it quickly shows you a ``READY.'' prompt and flashing block called
the cursor.  When the cursor is blinking, it tells you that the
computer is waiting for input.  The ``READY.'' message tells you
that the BASIC programming language is running and ready for you to
start programming.  You don't even need to load any programmes --
you can just get started.

\needspace{4cm} % Dont allow following paragraph to separate from
                % following element
Try typing the following into the computer and see what happens:

\begin{screenoutput}
HELLO COMPUTER
\end{screenoutput}

\needspace{4cm} % Dont allow following paragraph to separate from
                % following screenshot

To do this, just type the letters as you see them above.  The computer
will already be in upper-case mode, so you don't need to hold the \specialkey{SHIFT}
or \specialkey{CAPS\\LOCK} key down.  When you have typed ``HELLO COMPUTER'' press
  the \specialkey{RETURN} key.  This tells the computer you want it to accept the
  line of input you have typed.  When you do this, you should see a message something
  like the following:

\screenshotwrap{images/syntax-error.png}
  
  If you saw a \screentextwide{SYNTAX ERROR} message something like that one, then congratulations:
  You have succeeded in communicating with the computer!\index{Errors!Syntax}\index{SYNTAX ERROR}
  Error messages sound much nastier than they are.  The MEGA65 uses them, especially
  the syntax error to tell you when it is having trouble understanding what you have
  typed, or what you have put in a programme.  They are nothing to be afraid of, and
  experienced programmers get them all the time.

  In this case, the computer was confused because it doesn't understand the word
  ``hello'' or the word ``computer''.  That is, it didn't know what you wanted it to
  do.  In this regard, computers are quite stupid. They know only a few words, and
  aren't particularly imaginative about how they interpret them.

\needspace{4cm} % Dont allow following paragraph to separate from
                % following screenshot

So let's try that again in a way that the computer will understand.  Try typing
  the following in.  You can just type it right away. It doesn't matter that the
  syntax error message can still be seen on the screen.  The computer has already
  forgotten about that by the time it told you \screentextwide{READY.} again.

\begin{screenoutput}
PRINT "HELLO COMPUTER"
\end{screenoutput}

Again, make sure you don't use shift or shift-lock while typing it in.  The symbols around
the words \screentextwide{HELLO COMPUTER} are double-quotes.  If you are used to an Australian or American
keyboard, you might discover that they double-quote key is in a rather different place to
where you are used to:  Double-quotes can be typed on the MEGA65 by holding down the
\specialkey{SHIFT} key, and then pressing 2.  Don't forget to press the \specialkey{RETURN}
key when you are done, so that the computer knows you want it to do something with your input.

If you make a mistake while typing, you can use the \specialkey{INST\\DEL} to rub out the mistake
and fix it up.  You can also use the cursor keys to move back and forth on the line while
you edit the line you are typing, but there is a bit of a trick if you have already typed
a double-quote: If you try to use the cursor keys, it will print a funny reversed symbol
instead of moving the cursor.  This is because the computer thinks you want to record
moving the cursor in the text itself, which can be really useful and fun, and which you can
read more about in Chapter \ref{cha:getting-started}. But for now, if you
make a mistake just press the \specialkey{RETURN} key and type the messed up line again.

\needspace{4cm} % Dont allow following paragraph to separate from
                % following screenshot
Hopefully now you will see something like the following:

\screenshotwrap{images/print-hello-computer.png}

  This time no new \screentextwide{SYNTAX ERROR} message should appear. But if some kind
  of error message has appeared, just try typing in the command again, after
  taking a close look to work out where the mistake might be.

  Instead of an error, we should see \screentextwide{HELLO COMPUTER} repeated underneath
  the line you typed in.  The reason this happened is that the computer
  does understand the word \screentextwide{PRINT}.  It knows that whatever comes after
  the word \screentextwide{PRINT} should be printed to the screen.  We had to put \screentextwide{HELLO
  COMPUTER} inside double-quotes to tell the compute that we want it to be
  printed literally.

  If we hadn't put the double-quotes in, the computer would have thought
  that \screentextwide{HELLO COMPUTER} was the name of a stored piece of information.
  But because we haven't stored any piece of information in such a place,
  the computer will have zero there, so the computer will print the number
  zero, like this. If the computer prints zero or some other number when
  you expected a message of some sort, this can be the reason.

\needspace{4cm} % Dont allow following paragraph to separate from
                % following screenshot
  You can try it, if you like, and you should see something like the following:

  \screenshotwrap{images/print-hello-computer-no-quotes.png}

  In the above examples we typed commands in directly, and the computer executed
  them immediately after you pressed the \specialkey{RETURN} key.  This is why
  typing commands in this way is often called {\em direct mode} or {\em immediate mode}.
  
  But we can also tell the computer to remember a list of commands to execute one
  after the other.   This is done using the rather unimaginatively named {\em non-direct mode}.
  To use non-direct mode, we just put a number between 0 and 63999 at the start of
  the command.  The computer will then remember that command.  Unlike when we executed
  a direct-mode command, the computer doesn't print \screentextwide{READY.} again. Instead the cursor
  just reappears on the next line, ready for us to type in more commands.

\needspace{4cm} % Dont allow following paragraph to separate from
                % following screenshot
  Let's try that out with a simple little programme.  Type in the following three lines of
  input:

\begin{screenoutput}
1 FOR I = 1 TO 10 STEP 1
2 PRINT I
3 NEXT I
\end{screenoutput}
\index{FOR}
\index{BASIC 10 Commands!FOR}

\needspace{4cm} % Dont allow following paragraph to separate from
                % following screenshot
When you have done this, the screen should show something like this:

\screenshotwrap{images/first-steps-for-loop-programme-1.png}

If it doesn't you
can try again. Don't forget, if you feel that the computer is getting all muddled up,
you can just press the reset button or flip the power switch off and on on the left side of the
computer to reboot it. This only takes a couple of seconds, and doesn't hurt the MEGA65
in anyway.

We have told the computer to remember three commands, that is, \screentextwide{FOR I = 1 TO 10 STEP 1},
\screentextwide{PRINT I}
and \screentextwide{NEXT I}.  We have also told the computer which order we would like to run them in: The
computer will start with the command with the lowest number, and execute each command that
has the next higher number in turn, until it reaches the end of the list.  So it's a bit like
a reminder list for the computer. This is what we call a programme, a bit like the programme at
a concert or the theatre, it tells us what is coming up, and in what order.
So let's tell the computer to execute this programme.

But first, let's try to guess what will happen.  Let's start with the middle command, \screentextwide{PRINT I}.
We've seen the \screentextwide{PRINT} command, and we know it tells the computer to print things to the screen.
The thing it will try to print is \screentextwide{I}.  Just like before, because there are no double-quotes
around the \screentextwide{I}, it will try to print a piece of stored information.  The piece of information
it will try to print will be the piece associated with the thing \screentextwide{I}.

When we give a piece of
information like this a name, we call it a {\em variable}\index{variable}.  They are called
variables because they can vary.  That is, we can replace the piece of information associated
with the variable called I with another piece of information.  The old piece will be forgotten
as a result.  So if we gave a command like \screentextwide{LET I = 3}, this would replace whatever was stored
in the variable called \screentextwide{I} with the number 3.

Back to our programme, we now know that the 2\textsuperscript{nd} command will try to print the piece of information
stored in the variable \screentextwide{I}.  So lets look at the first command: \screentextwide{FOR I = 1 TO 10 STEP 1}.  Although
we haven't seen the \screentextwide{FOR} command before, we can take a bit of a guess at how it works. It looks like
it is going to put something into the variable \screentextwide{I}.  That something seems to have something to do
with the range of number 1 through 10, and a step or interval of 1.  What do you think it will do?

If you guessed
that it will put the values 1, 2, 3, 4, 5, 6, 7, 8, 9 and then 10 into the variable \screentextwide{I}, then you
can give yourself a pat on the back, because that's exactly what it does.  It also helps us to
understand the 3\textsuperscript{rd} command, \screentextwide{NEXT I}: That command tells the computer to put the next value into
the variable \screentextwide{I}.  And here is a little bit of magic: When the computer does that, it goes back
up the list of commands, and continues again from the command after the \screentextwide{FOR} command.

So lets pull that together: When the computer executes the first command, it discovers that it has
to put 10 different values into the variable \screentextwide{I}. It starts by putting the first value in there, which
in this case will be the number 1.
The computer then continues to the second command, which tells the computer to print the piece of
information that is currently stored in the variable called \screentextwide{I}. That will be the number 1, since
that was the last thing the computer was told to put there.  Then the computer proceeds to the
third command, which tells it that it is time to put the next value into the variable \screentextwide{I}.  So the
computer will throw away the number 1 that is currently in the variable \screentextwide{I}, and put the number 2 in
there, since that is the next number in the list.  It will then continue from the 2\textsuperscript{nd} command,
which will cause the computer to print out the contents of the variable \screentextwide{I} again.  Except that this
time \screentextwide{I} has had the number 2 stored in it most recently, so the computer will print the number 2.
This process will repeat, until the computer has printed all ten values that the \screentextwide{FOR} command
indicated it to do.   

\needspace{4cm} % Dont allow following paragraph to separate from
                % following screenshot
To see this in action, we need to tell the computer to execute the programme of commands we typed in.
We do this by using the \screentextwide{RUN} command. Because we want it to run the programme immediately, we
should use immediate mode (remember, this is another name for direct mode).
So just type in the word \screentextwide{RUN} and press the \specialkey{RETURN} key.  You should then see a display
that looks something like the following:

\screenshotwrap{images/first-steps-for-loop-programme-1-running.png}

  You might notice a couple of things here:

  First, the computer has told us it is \screentextwide{READY.} again
  as soon as it finished running the programme. This just makes it easier for us to know when we
  can start giving commands to the computer again.

  Second, when the computer got to the bottom of the screen
  it automatically scrolled the display up to make space.  This is quite normal.  What is important
  to remember, is that the computer forgets everything that scrolls off the top.  The only exception
  is if you have told the computer to remember a command by putting a number in front of it.  So
  our programme is quite safe for now. We can see that this is the case by typing the \screentextwide{RUN} command a
  couple more times: The programme listing will have scrolled off the top of the screen, but we can
  still RUN the programme, because the computer has remembered it.  Give it a try!
  Did it work?

\needspace{4cm} % Dont allow following paragraph to separate from
                % following screenshot
  If you wish to see the programme of remembered commands, you can use the \screentextwide{LIST}\index{LIST}\index{BASIC 10 Commands!LIST}
  command.  This commands causes the computer to display the remembered programme of commands to the screen, like in the display here.
  If you would like to replace any of the commands in the programme, you can type a new line that has the same number as the one you
  wish to change. 

\screenshotwrap{images/first-steps-for-loop-programme-1-listing.png}

\needspace{4cm} % Dont allow following paragraph to separate from
                % following screenshot
  For example, to print the results all on one line, we could modify the second line of the programme to \screentextwide{PRINT I;} by
  typing the following line of input and pressing the \specialkey{RETURN} key:


  
\begin{screenoutput}
2 PRINT I;
\end{screenoutput}

\index{PRINT}
\index{BASIC 10 Commands!PRINT}
%\end{tcolorbox}

\needspace{4cm} % Dont allow following paragraph to separate from
                % following screenshot

You can make sure that the change has been remembered by running the \screentextwide{LIST} command again, as we can see here.
You can then use the \screentextwide{RUN} command to run the modified
programme, like this:

\screenshotwrap{images/first-steps-for-loop-programme-1-modified.png}

It is quite easy to modify your programmes in this way.  As you become more comfortable with the process, there are two
additional helpful tricks:

First, you can give the \screentextwide{LIST} command the number of a command, or line as they are referred to, and it will display only
that line of the programme.  Alternatively, you can give a range separated by a minus sign to display only a section of the programme,
e.g., \screentextwide{LIST 1 - 2} to list the first two lines of our programme.

Second, you can use the cursor keys to move the cursor to a line which has already been remembered and is displayed on the screen. If you
modify what you see on the screen, and then press the \specialkey{RETURN} key while the cursor is on that line, the BASIC interpreter will
read in the modified line and replace the old version of it.  It is important to note that if you modify multiple lines of the programme
at the same time, you must press the \specialkey{RETURN} key on each line that has been modified. It is good practice to check that the
programme has been correctly modified. Use the \specialkey{LIST}\index{LIST}\index{BASIC 10 Commands!LIST} command again to achieve this.
  
  
  \subsubsection{Exercises to try}

  {\bf 1. Can you make it count to a higher or lower number?}

  At the moment it counts from 1 to 10.  Can you change it to count to 20 instead?  Or to count from 3 to 17?
  Or how about from 14.5 to 21.5? What do you think you would need to reverse the order in which it counts?

  {\em Clue:} You will need to modify the \screentextwide{FOR} command.  

  {\bf 2. Can you change the counting step?}

  At the moment it counts by ones, i.e., each number is one more than the last.  Can you change it to count by twos
  instead? Or by halves, so that it counts 1, 1.5, 2, 2.5, 3, \ldots?
  
  {\em Clue:} You will need to modify the \screentextwide{STEP} clause of the \screentextwide{FOR} command.\index{STEP}\index{BASIC10 Commands!STEP}
  
  
  {\bf 3. Can you make it print out one of the times tables?}
  
  At the moment it prints the answers to the 1 times tables, because it counts by ones.
  Can you make it count by threes, and show the three times tables?
  
  {\em Clue:} You will need to modify the \screentextwide{FOR} command.
  
  {\bf 4. Can you make it print out the times tables from 1$\times$1 to 10$\times$10?}
  
  {\em Clue:} You might like to use ; on the end of \screentextwide{PRINT} statements, so that you can have
  more than one entry per line on the screen.\\
  {\em Clue:} The \screentextwide{PRINT} statement without any argument will just advance to the start of the next line.\\
  {\em Clue:} You might need to have multiple \screentextwide{FOR} loops, one inside the other.
  
\section{First steps with text and numbers}

In the last section we started to use both numbers and text.  Text on computers is made by stringing individual letters
and other symbols together.  For this reason they are called {\em strings}.  We also call the individual letters and
symbols {\em characters}.  The name character comes from the printing industry where each of the symbols that can be
printed on a page. For computers, it has much the same meaning, and the set of characters that a computer can display
is rather unimaginatively called a {\em character set}.\index{character}\index{character set}\index{string}.

When the MEGA65 expects some for of input, it is typically looking for one of four things:

\begin{enumerate}
\item {\em a keyword} like \screentextwide{PRINT} or \screentextwide{STEP}, which are words that have a special meaning to the computer;
\item {\em a variable name} like \screentextwide{I} or \screentextwide{A\$} that it will then use to either store or retrieve a piece of information;
\item {\em a number} like \screentextwide{42} or \screentextwide{-30.3137}; or
\item {\em a string} like \screentextwide{"HELLO COMPUTER"} or \screentextwide{"23 KILOMETRES"}.
\end{enumerate}

\needspace{4cm} % Dont allow following paragraph to separate from
                % following screenshot
Sometimes you have a choice of which sort of thing you can provide, while other times you have less choice. What
sort of thing the computer will accept depends on what you are doing at the time.  For example, in the previous
section we discovered that when the computer tells us that it is \screentextwide{READY}, that we can give it
a keyword or a number.  Do you think that the computer will accept all four kinds of thing when it says
\screentextwide{READY.}?  We already know that keywords and numbers and keywords can be entered, but what about
variable names or strings?  Let's try typing in a variable name, say \screentextwide{N}, and pressing the \specialkey{RETURN} key,
and see what happens.  And then lets try with a string, say \screentextwide{"THIS IS A STRING"}.

\screenshotwrap{images/typing-variable-name-or-string}

You should get a syntax error each time, telling you that the computer doesn't understand the input you have given it.
Let's start with when you typed the variable: If you just tell the computer the name of a stored piece of information,
it doesn't have the foggiest idea what you are wanting it to do.  It's the same when you give it a piece of information,
like a string, without telling the computer what to do with it.

But as we discovered in the last section, we can tell the computer that we want to see the piece of information that is
stored in a variable using the \screentextwide{PRINT} command.  So we could instead type in \screentextwide{PRINT N}, and
the computer would know what to do, and will print the piece of information stored in the variable called \screentextwide{N}.

In fact, using the \screentextwide{PRINT} command is so common, that programmers got annoying having to type in the \screentextwide{PRINT}
command all the time, that they made a short cut: If you type a question mark character, i.e., a \screentextwide{?}, the computer
knows that you mean \screentextwide{PRINT}.  So for example if you type \screentextwide{? N}, it will do the same as typing
\screentextwide{PRINT N}.  Of course, you have to press the \specialkey{RETURN} key after each command to tell the computer
you want it to process what you typed.  From here on, we will assume that you can remember to do that, without being reminded.

\needspace{4cm} % Dont allow following paragraph to separate from
                % following screenshot
The \screentextwide{?} shortcut also works if you are telling the computer to remember a command as part of a programme.
So if you type \screentextwide{1 ? N}, and then \screentextwide{LIST}, you will see \screentextwide{1 PRINT N}, as we can see
in the following screen-shot:

\screenshotwrap{images/print-question-mark}

Like we saw in the last section, the variable \screentextwide{N} has not had a value stored in it, so when the computer looks for
what is there, it finds nothing.  Because \screentextwide{N} is a {\em numeric variable}\index{variable!numeric}, when there is
nothing there, this means zero.  If it was a {\em string variable}\index{variable!string}, then it would have found literally nothing.
We can try that, but first we have to explain how we tell the computer we are talking about a string variable.  We do that by
putting a dollar sign character, i.e., a \screentextwide{\$}, on the end of the variable name. So if we put a \screentextwide{\$} on
the end of the variable name \screentextwide{N}, it will refer to a string variable called \screentextwide{N\$}.

\needspace{4cm} % Dont allow following paragraph to separate from
                % following screenshot
We can experiment with these variables by using the hopefully now familiar
\screentextwide{PRINT} statement (or the \screentextwide{?} shortcut)
to see what is in the variables. But we need a convenient way to put
values into them.  Fortunately we aren't the first people to want to
put values into variables, and so the
\screentextwide{LET}\index{LET}\index{BASIC 10 Commands!LET} exists.
The \screentextwide{LET} command is used to put a value into a
variable.  For example, we can tell the computer:

\begin{screenoutput}
  LET N = 5.3
\end{screenoutput}

\needspace{4cm} % Dont allow following paragraph to separate from
                % following screenshot
This tells the computer to put the value 5.3 into the variable
\screentextwide{N}.  We can then use the \screentextwide{PRINT}
statement to check that it worked.  Similarly, we can put a value into
the variable \screentextwide{N\$} with something like:

\begin{screenoutput}
  LET N$ = "THE KING OF THE POTATO PEOPLE"
\end{screenoutput}

\needspace{4cm} % Dont allow following paragraph to separate from
                % following screenshot
If we try those, we will see something like the following:

\screenshotwrap{images/let-command-examples}

\needspace{4cm} % Dont allow following paragraph to separate from
                % following screenshot
We mentioned just before that \screentextwide{N} is a numeric
variable and that \screentextwide{N\$} is a string variable. This
means that we can only put numbers into \screentextwide{N} and
strings into \screentextwide{N\$}.  If we try to put the wrong kind
of information into a variable, the computer will tell us that we have
mis-matched the kind of information with the place we are trying to
put it by giving us a \screentextwide{TYPE MISMATCH
  ERROR}\index{Errors!Type mismatch}\index{Type mismatch error} like
this:

\screenshotwrap{images/type-mismatch-errors}

This leads us to a rather important point: \screentextwide{N} and
\screentextwide{N\$} are separate variables, even though they have
similar names.  This applies to all possible variable names: If the
variable name has a \screentextwide{\$} character on the end, it
means it is a string variable quite separate from the similarly named
numeric variable.  To use a bit of jargon, this means that each {\em type}
of variable has their own separate {\em name
  spaces}\index{name spaces}.

(There are also four other variable name
spaces that we haven't talked about yet: integer variables, identified
by having a \screentextwide{\%} character at the end of their name,
e.g., \screentextwide{N\%}, and arrays of numeric, string or integer
variables. But don't worry about those for now.
We'll talk about those a bit later on.)

So far we have only given values to variables in direct mode, or
by using constructions like \screentextwide{FOR} loops.  But we
haven't seen how we can get information from the user when a programme
is running.  One way that we can do this, is with the
\screentextwide{INPUT}\index{BASIC 10 Commands!INPUT}\index{INPUT}
command.

\needspace{4cm} % Dont allow following paragraph to separate from
                % following screenshot
\screentextwide{INPUT} is quite easy to use: We just have to say which
variable we would like the input to go into.  For example, to tell the
computer to ask for the user to provide something to put into the
variable \screentextwide{A\$}, we could use something like
\screentextwide{INPUT A\$}.  The only trick with the \stw{INPUT}
command is that it cannot be used in direct mode\index{Direct Mode}.
If you try it, the computer will tell you \stw{ILLEGAL DIURECT
  ERROR}\index{Errors!Illegal Direct}\index{Illegal Direct Error}.
Try it, and you should see something like the following

\screenshotwrap{images/illegal-direct-error}

\needspace{4cm}
This means that the \stw{INPUT} command can only be used as part of a
programme.  So we can instead do something like the following:

\begin{screenoutput}
1 INPUT A$
2 PRINT "YOU TYPED "; A$
RUN
\end{screenoutput}

\needspace{4cm}
What do you think that this will do?  The first line will ask the
computer for something to put into the variable \stw{A\$}, and the
second line will print the string \stw{"YOU TYPED"}, followed by
what the \stw{INPUT} command read from the user.  Let's try it out:

\screenshotwrap{images/input-example-1}

Did you expect that to happen? What is this question mark doing there?
The \stw{?} here is the computer's way of telling you that a
programme is waiting for some input from you.  This means that the
computer uses the same symbol, \stw{?}, to mean two different things:
If you type it as part of a programme or in direct mode, then it is a
short-cut for the \stw{PRINT} command. That's when you type it. But if
the computer shows it to you, it has this other meaning, that the
computer is waiting for you to type something in. There is also a
third way that the computer uses the \stw{?} character. Have you
noticed what it is?  It is to indicate the start of an error
message. For example, a Syntax Error is indicated by \stw{?SYNTAX
 ERROR}. When a character or something has different meanings in
different situations or contexts, we say that it its {\em context
  dependent}\index{context dependent}.

\needspace{4cm}
But returning to our example,  if we now type
something in, and press the \specialkey{RETURN} key to tell the
computer that you are done, the programme will continue, like this:

\screenshotwrap{images/input-example-2}

\needspace{4cm}
Of course, we didn't really know what to type in, because the programme
didn't give any hints to the user as to what the programmer wanted
them to do. So we should try to provide some instructions.  For
example, if we wanted the user to type their name, we could print a
message asking them to type their name, like this:

\begin{screenoutput}
  1 PRINT "WHAT IS YOUR NAME"
  2 INPUT A$
  3 PRINT "HELLO "; A$
\end{screenoutput}

\needspace{4cm}
Now if we run this programme, the user will get a clue as to what we
expect them to do, and the whole experience will make a lot more sense
for them:

\screenshotwrap{images/input-example-3}

When we run the programme, we first see the \stw{WHAT IS YOUR NAME} message
from line 1.  The computer doesn't print the double-quote symbols,
because they only told the computer that the piece of information
between them is a string.  The string itself is only the part in
between.

After this we see the \stw{?} character again and the blinking cursor
telling us that the computer is waiting for some input from us.  The
rest of the progammed is {\em blocked}\index{IO!blocking}\index{blocked} from continuing until it we type the
piece of information.  Once we type the piece of input, the computer
stores it into the variable \stw{A\$}, and can continue.  Thus when it
reaches line 3 of the programme, it has everything it needs, and
prints out both the \stw{HELLO} message, as well as the information
stored in the variable called \stw{A\$}.

Notice that the word \stw{LISTER} doesn't appear anywhere in the
programme.  It exists only in the variable.  This ability to process
information that is not part of a programme is one of the things that
makes computer programmes so powerful and able to be used for so many
purposes. All we have to do is to change the input, and we can get
different output.


\needspace{4cm}
For example, with our programme we run it again and again, and give it
different input each time, and the
programme will adapt its output to what we type. Pretty nifty, right?
Let's have the rest of the crew try it out:

\screenshotwrap{images/input-extra-ignored-1}

We can see that each time the programme prints out the message
customised with the input that you typed in\ldots Until we get to
\stw{RIMMER, BSC}. As always, Mr. Rimmer is causing trouble.  In this
case, he couldn't resist putting his Bronze Swimming Certificate
qualification on the end of his name.

We see that the computer has
given us a kind of error message, \stw{?EXTRA IGNORED}\index{Extra
  Ignored}\index{Errors!Extra Ignored}\index{Warnings!Extra Ignored}.
The error is not written in red, and doesn't have the word \stw{ERROR}
on the end.  This means that it is a warning, rather than an error.
Because it is only a warning, the programme continues.  But something
has happened: The computer has ignored Mr. Rimmer's \stw{BSC}, that
is, it has ignored the extra input.  This
is because the \stw{INPUT} command doesn't really read a whole line
of input. Rather, it reads {\em one piece of information}.  The
\stw{INPUT} command thinks that a piece of information ends at the end
of a line of input, or when it encounters a comma (\stw{,}) or colon
(\stw{:}) character.

\needspace{4cm}
If you want to include one of those symbols, you need to surround the
whole piece of information in double-quotes.  So, if Mr. Rimmer had
read this guide instead of obsessing over the Space Core Directives,
he would have known to type \stw{"RIMMER, BSC"} (complete with the
double-quotes), to have the programme 
  run correctly.  It is important that the quotes go around the whole
  piece of information, as otherwise the computer will think that the
  first quote marks the start of a new piece of information.  We can
  see the difference it makes below:

  \screenshotwrap{images/input-quoting-1}

\needspace{1.5cm}
While this can all be a bit annoying at times, it has a purpose: The
\stw{INPUT} command can be used to read more than one piece of
information.  We do this by putting more than one variable after the
\stw{INPUT} command, each separated by a comma.  The \stw{INPUT}
command will then expect multiple pieces of information.  For example,
we could ask for someone's name and age, with a programme like this:

\begin{screenoutput}
  1 PRINT "WHAT IS YOUR NAME AND AGE"
  2 INPUT A$, A
  3 PRINT "HELLO "; A$
  4 PRINT "YOU ARE"; A; " YEARS OLD."
\end{screenoutput}

\needspace{4cm}
If we run this programme, we can provide the two pieces of information
on the one line when the computer presents us with the \stw{?} prompt,
for example \stw{LISTER, 3000000}. Note the comma that separates the
two pieces of information, \stw{LISTER} and \stw{3000000}.  It's also
worth noticing that we haven't put any thousands separators into the
number 3,000,000.  If we did, the computer would think we meant three
separate pieces of information, \stw{3}, \stw{000} and \stw{000},
which is not what we meant.  So let's see what it looks like when we
give \stw{LISTER, 3000000} as input to the programme:

  \screenshotwrap{images/input-multiple-1}

In this case, the \stw{INPUT}\index{INPUT}\index{BASIC 10
  Commands!INPUT} command reads the two pieces of
information, and places the first into the variable \stw{A\$}, and the second
into the variable \stw{A}. When the programme reaches line 3 it prints
\stw{HELLO} followed by the first piece of information.
Then when it gets to line 4, it prints the string \stw{YOU ARE},
followed by the contents of the variable \stw{A}, which is the number
3,000,000, and finally the string \stw{YEARS OLD}.

It's also possible to just give one piece of information at a time.
In that case, the \stw{INPUT} command will ask for the second piece
of information with a double question-mark prompt, i.e., \stw{??}.
Once it has the second piece of information.  (If we had more than
two variables on the \stw{INPUT} command, it will still present the
same \stw{??} prompt, rather than printing more and more
question-marks.)

\needspace{4cm}
So if we try this with our programme, we can see ths \stw{?} and
\stw{??} prompts, and how the first piece of information ends up in
\stw{A\$} because it is the first variable in the \stw{INPUT}
statement.
The second piece of information ends up in \stw{A} because \stw{A} is
the second variable after the \stw{INPUT} statement. Here's how it
looks if we give this input to our programme:

  
\screenshotwrap{images/input-multiple-2}

Until now we have been asking the user to input information by using a
\stw{PRINT} command to display the message, and then an \stw{INPUT}
command to tell the computerr which variables we would like to have
some information input into.  But, like with the \stw{PRINT} command,
this is something that happens often enough, that there is a shortcut
for it. It also has the advantage that it looks nicer when
running, and makes the programme a little shorter. The short cut is to
put the message to show after the \stw{INPUT} command, but before the
first variable.

We can change our programme to use this approach.  First, we can
change line 3 to include the prompt after the \stw{INPUT} command.  We
can do this one of two ways: First, we could just type in a new line
3. The computer will automatically replace the old line 3 with the new
one.

But, as we have mentioned a few times now, programmers are lazy
beasts, and so there is a short-cut: If you can see the line on the
screen that you want to change, you can use the cursor keys to
navigate to that line, edit it on the screen, and then press the
\specialkey{RETURN} key to tell the computer to accept the new version
of the line.\index{Programmes!editing}\index{Programmes!replacing
  lines}\index{Lines!editing}\index{Lines!replacing}

\needspace{4cm}
Either way, you
can check that the changes succeeded by typing the \stw{LIST} command
on any line of the screen that is blank.  This will show the revised
version of the programme.  For example:

\screenshotwrap{images/replacing-line-1}

\needspace{3cm}
We still have a little problem, though: Line 1 will print the message
\stw{WHAT IS YOUR NAME AND AGE}, and then Line 2 will print it again!
We only want the message to appear once. Thus we would like to change
line 1 so that it doesn't do this any more.  Because there is no other
command on line 1 that we want to keep, that line can just become
empty. So we can type in something like this:

\begin{screenoutput}
1
\end{screenoutput}

\needspace{4cm}
We can confirm that the contents of the line have been deleted by
running the \stw{LIST} command again, like this:

\screenshotwrap{images/deleting-line-1}

Did you notice something interesting? When we told the computer to
make line 1 of the programme empty, it deleted it completely!
That's because the computer thinks that an empty line is of no use.
It also makes sure that your programmes don't get all cluttered up
with empty lines if you make lots of changes to your programmes.

With that out the way, let's run our programme and see what happens.
As usual, just type in the \stw{RUN} command and hit the
\specialkey{RETURN} key.  You should see something like this:

\screenshotwrap{images/input-comma}

  \subsubsection{Exercises to try}

  {\bf 1. Can you make the programme ask someone for their name, and
    then for their favourite colour?}

  At the moment it asks for their name and age. Can you change the
  programme so that it reports on their favourite colour instead of
  their age?

  {\em Clue:} What type of information is age? Is it numeric or a
  string? Is it the same type of information as the name of a colour?



\section{Making simple decisions}

\section{Random numbers and chance}
